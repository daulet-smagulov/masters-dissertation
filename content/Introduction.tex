\anonsection{Введение}
%#
При построении системы измерения риска инвестиций существует определенная свобода выбора методологии для определения фактического и потенциального риска.
%#
В данной работе рассчитываются  меры риска, широко используемые в риск-менеджменте:  стоимостная мера риска \textit{Value-at-Risk} (VaR) и условная  стоимостная мера риска \textit{Conditional-Value-at-Risk} (CVaR), которые можно определить для разных уровней значимости $\alpha$.  
В статье О.~Крицкого и М.~Ульяновой \cite{Kritski2007} показано, что при наличии корреляции в динамике активов портфеля оценка одномерных величин VaR и CVaR оказывается неадекватной по отношению к риску портфеля. 
Поэтому для оценки риска портфеля необходимо использовать $d$-мерные величины, определяемые с помощью многомерной зависимости случайных величин.

Существует много различных подходов, которые активно применяются для представления многомерной зависимости случайных величин, например, метод главных компонент, байесовские сети, нечёткая логика, факторный анализ и использование совместной многомерной функции распределения \cite{Huynh2014, Kole2007}. 
Зависимость между случайными величинами $X_1, X_2, \ldots, X_d$ полностью определяется совместной функцией распределения $F(X_1, X_2, \ldots, X_d)$. 
Идея представления этой функции через две составляющие части, одна из которых определяет структуру зависимости, а другая --  частные (в англоязычной литературе -- маргинальные) распределения для каждой из рассматриваемых случайных величин
$X_1, X_2, \ldots, X_d$ по отдельности, приводит к понятию \boldit{копула-функции} или просто \boldit{копулы}.

В 1959 году Абе Шкляр \cite{Sklar1959} сформулировал и впервые доказал теорему о том, что набор маргинальных распределений можно записать через одно многомерное распределение посредством копулы. 
Копула содержит всю информацию о структуре зависимости рассматриваемых переменных. 
%#
В статье \cite{Penikas2010} автор представил концепты копулярных моделей и их применение в различных финансовых вопросах, включая риск-менеджмент.

В настоящее время копулы активно используются при решении задач из
различных предметных областей: эконометрика, финансовая и
актуарная математика \cite{Penikas2010, Penikas2014, Travkin2013, Knyazev2016, Atskanov2016, Antonov2016, Shemyakin2017}, задачах построения систем
безопасности \cite{Sundaresan2011}, биостатистике, гидрологии,
климатологии (см. ссылки в работе \cite{Fantazzini2011}).
Детальному описанию копул-функций и их свойств посвящены
монографии \cite{Joe2014, Nelsen1999}. В русскоязычной литературе
базовые элементы теории и приложения копул изложены в работах
\cite{Fantazzini2011, Blago2012}.

Сегодня существует множество способов описания финансовых данных с использованием нормального (Гауссового) распределения. 
Хорошо известно, что \boldit{Гауссова копула}, т.~е. копула, построенная нормальными маргинальными распределениями, используется в качестве одного из способов описания портфеля в теории Марковица. 
С другой стороны, многие эмпирические исследования показали, что распределение Гаусса имеет множество недостатков в описании зависимости финансовых данных \cite{Limp2011, Rachev2005, Wilmott2007}. 
Более того, стандарт~EBA \cite{EBA2015} не рекомендует использовать Гауссовы копулы для моделирования финансового риска. 
Например, в большинстве случаев более подходящей оказывается копула \boldit{t\,--\,Стьюдента} с несколькими целыми степенями свободы (обычно три или четыре).
Аналитическое решение для нахождения меры чувствительности риска -- CVaR, -- было представлено в статье \cite{Stoyanov2013}.

Исследования \cite{Ane2003, Kole2007, Lourme2016, Xu2008} рыночного риска в рамках управления портфелем во многом сходны друг с другом и отличаются зачастую только используемыми данными и незначительными деталями в оценке параметров копулярных моделей. 
Среди комплексных исследований выделим статью Тьерри Ане и Сесиль Харуби \cite{Ane2003} -- одну из первых, в которой авторы использовали копулу Клейтона в качестве инструмента для определения структуры зависимости между акциями международного фондового индекса.
Александр Лурме и др.\,\cite{Lourme2016} фокусируются на тестировании возможностей использования Гауссовой и $t$-копул для решения задач риск-менеджмента. 
Они предлагают $d$-мерную компактную Гауссову и $t$\,--\,Стьюдента  доверительную область, внутри которой вектор из выборки многомерной случайной величины, равномерно распределённой на отрезке $[0, 1]$, попадает с вероятностью~$\alpha$.
Результаты показывают, что копула $t$\,--\,Стьюдента, использованная для построения VaR-модели, является более консервативной и адекватной по сравнению c Гауссовой.
Портфель акций, облигаций и сделок на недвижимость был рассмотрен в работе \cite{Kole2007}. %# для определения важность выбора правильной копулы для риск-менеджмента. 
Были опробованы копулы Гауссовы, Стьюдента и Гумбеля для моделирования зависимости дневных доходностей, которые аппроксимируют данные для указанных выше активов. 
Затем по расчетам VaR было установлено, что гауссова копула слишком оптимистична в отношении преимуществ диверсификации активов, тогда как копула Гумбеля слишком пессимистична.

При моделировании копулы важной задачей является оценка параметров модели.
% Для оценки параметов копулярных моделей существуют различные методы:
В настоящее время разработано множество алгоритмов для оценки параметров и построения копул:
полный параметрический \cite{Patton2006}, полупараметрический \cite {Chen2006, Lourme2016} и непараметрический метод \cite{Fermanian2003, Kim2007}. 
Полный параметрический метод реализуется посредством двухэтапной оценки максимального правдоподобия (Maximum Likelihood Estimation -- MLE), предложенный Гарри Джо \cite{Joe1997, Joe2014}.
Параметры копулы оцениваниваются с использованием двухступенчатого параметрического подхода MLE, также называемого методом оценки частных распределений (Inference Functions for Margins -- IFM). 
Данный метод включает два этапа: (1) оценка параметров маргинальных распределений и затем (2) оценка параметров копулы.
%# добавить по 2-3 предложения про полупараметрический  и непараметрический методы

Для двумерного случая среди основных семейств копул выделяют: эллиптические (Гауссова, $t$\,--\,Стьюдента), архимедовы (Клейтона, Франка, Джо) и экстремальные (Гумбеля, Коши).
В диссертационном исследовании \cite{Xu2008} был использован двухэтапный MLE-метод, в ходе которого автор использует все возможные комбинации различных маргинальных распределений (нормальное, Стьюдента с асимметрией и без нее), а также различные архимедовы копулы. %# для оценки и тестирования. 
Решение о выборе маргинального распределение определяется после второго этапа MLE. 
Для этой цели была предложена модификация теста Хансена \cite{Hansen2005}; данный тест позволяет определить наиболее адекватную копулу.

Многомерные копулы, основанные на одном распределении (например, Гауссова или $t$\,--\,Стьюдента) или созданные из так называемых функций-генераторов, не обладают необходимой гибкостью для проведения моделирования зависимости между большим числом переменных \cite{Brechmann2013}. Эти недостатки предопределили направление дальнейших исследований, в результате которых Гарри Джо \cite{Joe1996} предложил  концепцию \boldit{регулярных вайн-копул} (R-vine), дальнейшее развитие этой концепции представлено в работах \cite{Brechmann2013, Cooke2015}.
R-vine копула представляют собой гибкую графическую модель для описания многомерных копул, построенных с использованием каскада двумерных копул (двумерных функций). 
Такую копулу легче интерпретировать и визуализировать, и сегодня существует множество различных методов для работы с R-vine копулой \cite{Cooke2015, Czado2010, Dissmann2013}.
Например, в исследовании \cite{Dissmann2013} описаны новые алгоритмы оценки параметров вайн-копулы и моделирования с использованием специальных R-vine копул.
Выбор R-vine древовидной структуры определяется на основе алгоритма максимального покрывающего дерева (Maximum Spanning Tree, MST), в котором весовые коэффициенты ребер дерева учитывают степень корреляционной зависимости исходных временных рядов данных.

Цель данной работы -- построить систему оценивания риска с
использованием копулярных моделей -- эффективных, последовательных
и достаточно чувствительных к риску. %# провести  сравнение значения оценки финансового риска в рамках управления портфелем путём определения параметров моделей копула в в соответствии с динамикой и корреляцией цен составляющих его активов. 
%# Иными словами, задачей исследования было протестировать модели копула путём сравнения риск метрик.


Для достижения поставленной цели необходимо последовательно решить следующие задачи:

1. Проанализовать сущестующие методы и выбрать метод для оценки параметров различных копулярных моделей. 

2. С использованием копулярных моделей вычислить и сравнить поведение различных мер риска (VaR, CVaR) для инвестиционного портфеля.


%# использовать текст в заключении
%# В данной работе постоена копулярная модель для портфеля, состоящего из четырёх активов -- ежедневные цены на фьючерсные контракты. Был использован полный параметрический MLE-метод оценки копулы, описанный Г. Джо \cite{Joe1997, Joe2014}.

%# Моя работа содержит три нововведения. Во-первых, я показываю, что Гауссовы, Стьюдента и вайн-копулы могут быть использованы для представления многомерной зависимости в коротких временных рядах (только 253 наблюдения), в то время как применение копул изучалось только по отношению к длительным наблюдениям \cite{Dissmann2013, Kole2007, Lourme2016}.
%# Во-вторых, я использую маргинальные распределения, отличные от нормального: гиперболическое~\cite{Barndoff1983}, устойчивое~\cite{Rachev2005} и Мейкснера~\cite{Schoutens2002}. 
%# Стоянов и др.~\cite{Stoyanov2013} рассматривают подобную задачу, используя только симметричные варианты распределений Стьюдента и устойчивого, а также обобщенное нормальное распределение. 
%# В-третьих, при моделировании R-vine копулы я использую нецелые степени свободы для двух-параметрических копул, которые существенно увеличивают гибкость и расширяют возможности копул в рамках риск-менеджмента.

%# Чтобы сравнить показатели различных мер риска (VaR, CVaR), мы предлагаем использовать Монте-Карло симуляции для моделирования оптимального CVaR-портфеля.
%# Преимущество оптимизации портфеля CVaR заключается в том, что оптимизация CVaR-портфеля производится путём решения задачи линейного программирования \cite{Rock2000}.
%# Мы заимствуем и сравниваем исторические P\&L с кривыми VaR и CVaR, оцениваемых с помощью моделей копул, на разных уровнях значимости.

%# Раздел~\ref{section:object} посвящён объекту и методам исследования. В разделе~\ref{section:methodology} описана методология модели. В разделе~\ref{section:results} описаны исходные данные, приведены результаты калибровки модели: оценённые параметры и результаты оптимизации портфеля, а также финальные значения риск\-метрик.


\section{Обзор литературы}
\label{section:literature}




\section{Объект и методы исследования}
\label{section:object}

Здесь будут перечислены объект и методы исследования.