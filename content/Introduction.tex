\anonsection{Введение}

При построении системы измерения риска инвестиций существует определенная свобода выбора методологии для определения фактического и потенциального риска.
В данной работе рассчитываются  меры риска, широко используемые в риск-менеджменте:  стоимостная мера риска \textit{Value-at-Risk} (VaR) и условная  стоимостная мера риска \textit{Conditional-Value-at-Risk} (CVaR), которые можно определить для разных уровней значимости $\alpha$.
В статье О.~Крицкого и М.~Ульяновой~\cite{Kritski2007} показано, что при наличии корреляции в динамике активов портфеля оценка одномерных величин VaR и CVaR оказывается неадекватной по отношению к риску портфеля. 
Поэтому для оценки риска портфеля целесобразно использовать $d$-мерные случайные величины, определяемые с помощью многомерных функций совместного распределений.

Зависимость между случайными величинами $X_1, X_2, \ldots, X_d$ может быть полностью определена через совместную функцию распределения $F(X_1, X_2, \ldots, X_d)$.
Идея представления этой функции через две составляющие части, одна из которых определяет структуру зависимости, а другая~---  частные (в англоязычной литературе~--- маргинальные) распределения для каждой из рассматриваемых случайных величин $X_1, X_2, \ldots, X_d$ по отдельности, приводит к понятию \boldit{копула-функции}, или просто \boldit{копулы}.
% Копула содержит информацию о структуре зависимости рассматриваемых переменных.

Сегодня существует множество способов описания финансовых временных данных с использованием нормального (Гауссового) распределения.
Хорошо известно, что \boldit{Гауссова копула}, т.~е. копула, для которой в качестве частного (маргинального) распределения выбрано нормальное распределение, используется в качестве одного из способов описания портфеля в теории Марковица. 
С другой стороны, многие эмпирические исследования показали, что распределение Гаусса имеет множество недостатков в описании зависимости финансовых временных рядов~\cite{Limp2011, Rachev2005, Wilmott2007}. 
Заметим, что стандарт EBA~\cite{EBA2015} не рекомендует использовать Гауссовы копулы для моделирования финансового риска, а в большинстве случаев более подходящей оказывается двухпараметрическая копула \boldit{t\,--\,Стьюдента}, для которой степень свободы $\nu$ (параметр модели) обычно выбирается равным трем или четырем.

Общая проблема использования копулярных моделей, как и любых других моделей в реальной практике,~--- требование нахождения оценок неизвестных параметров с наилучшими статистическими свойствами. 
При этом качество оценок может определяться по\-разному: несмещенная оценка, оценка с минимальной дисперсией; состоятельная оценка; оценка с хорошим асимптотическим поведением; удобная для вычислительной работы оценка и прочее.

%При моделировании копулы важной задачей является оценка параметров модели.
В настоящее время предложено множество подходов для оценки параметров и построения копуляпных моделей:
полный параметрический~\cite{Patton2006}, полупараметрический~\cite{Chen2006, Lourme2016} и непараметрический метод~\cite{Fermanian2003, Kim2007}.

Полный параметрический метод реализуется посредством двухэтапной оценки максимального правдоподобия (Maximum Likelihood Estimation, MLE), предложенный Гарри Джо~\cite{Joe1997, Joe2014}.
Параметры копулы оцениваниваются с использованием двухступенчатого параметрического подхода MLE, также называемого методом оценки частных распределений (Inference Functions for Margins, IFM). 
Данный метод включает два этапа: (1) оценка параметров маргинальных распределений и затем (2) оценка параметров копулы.
Полупараметрический метод предполагает эти же два этапа, в первом из которых вместо маргинальных рассчитываются эмпирические распределения.
Непараметрический метод на обоих этапах предполагает оценку эмпирических функций распределения: в первом~--- для маргиналов. во втором~--- для копулы.

Для двумерного случая среди основных семейств копул выделяют: эллиптические (Гауссова, $t$\,--\,Стьюдента), архимедовы (Клейтона, Франка, Джо) и экстремальные (Гумбеля, Коши).
В диссертационном исследовании~\cite{Xu2008} для оценки параметров и тестирования был использован двухэтапный параметрический MLE-метод, в ходе которого автор использует все возможные комбинации различных маргинальных распределений (нормальное, Стьюдента с асимметрией и без нее), а также различные архимедовы копулы. 
Решение о выборе маргинального распределение определяется после второго этапа MLE. 
Для этой цели в работе~\cite{Hansen2005} была предложена модификация теста Хансена, данный тест позволяет определить наиболее адекватную копулу.

Многомерные копулы, основанные на одном распределении (например, Гауссова или $t$\,--\,Стьюдента) или созданные из так называемых функций-генераторов, не обладают необходимой гибкостью для проведения моделирования зависимости между большим числом переменных~\cite{Brechmann2013}. Эти недостатки предопределили направление дальнейших исследований, в результате которых Гарри Джо~\cite{Joe1996} предложил  концепцию \boldit{регулярных иерархических копул} (regular vine copula, R-vine), дальнейшее развитие этой концепции представлено в работах~\cite{Brechmann2013, Cooke2015}.
R-vine копула представляют собой достаточно гибкую математичическую модель для описания многомерных законов распределения с использованием каскада двумерных копул (двумерных функций распределения). 
В настоящее время активно развиваются различные подходы для работы с R-vine копулой, такие копулы легче интерпретировать и визуализировать (например, в виде древовидной структуры или матриц). 
Отметим работы~\cite{Cooke2015, Czado2010, Dissmann2013}, в которых авторы уделяют большое внимание разработке новых алгоритмов для оценки параметров копулярных моделей, а также  приложениям R-vine копул для моделирования временных рядов из разных предметных областей.
Авторы используют древовидную и матричные структуры для визуализации процесса выбора и объединения двухмерных функций распределений в R-vine копуле, при этом они используют алгоритм построения максимального покрывающего дерева (Maximum Spanning Tree, MST), в котором вес ребер отражает степень корреляционной зависимости исходных временных рядов данных.

%Заметим, что построенные оценки параметров позволяют использовать полученные копулярные модели для различных приложений, таких как ... 

%Получение теоретических характеристик процедур разладки часто затруднено, а теоретическая проверка на робастность к небольшим отклонениям в модели для различных зашумлений либо крайне громоздка, либо невозможна. Поэтому большое значение для практики имеет численное исследование построенных процедур для нахождения их характеристик.


Цель данной работы~--- построить систему оценивания риска с использованием копулярных моделей.
% -- эффективных, последовательных и достаточно чувствительных к риску.
% провести  сравнение значения оценки финансового риска в рамках управления портфелем путём определения параметров моделей копула в в соответствии с динамикой и корреляцией цен составляющих его активов. 
% Иными словами, задачей исследования было протестировать модели копула путём сравнения риск метрик.

Для достижения поставленной цели необходимо последовательно решить следующие задачи:
\begin{enumerate}[label=\arabic*)]
    \item проанализовать сущестующие подходы и выбрать метод для оценки параметров копулярных моделей. 
    \item с использованием копулярных моделей вычислить и сравнить поведение различных мер риска (VaR, CVaR) для инвестиционного портфеля.
\end{enumerate}


\section{Обзор литературы}
\label{section:literature}

Существует много различных подходов, которые активно применяются для представления многомерной зависимости случайных величин, например, метод главных компонент, байесовские сети, нечёткая логика, факторный анализ~\cite{Huynh2014, Kole2007}. 
В 1959 году Абе Шкляр~\cite{Sklar1959} сформулировал и впервые доказал теорему о том, что набор частных (маргинальных) распределений можно записать через одно многомерное распределение посредством копулы.

Несмотря на то что теория копула-моделей исследована относительно полно, проблема оценивания и статистические выводы для копула-моделей, в определенном контексте, все
еще требуют дальнейших исследований~\cite{Fantazzini2011}.

В настоящее время копулы активно используются при решении задач из различных предметных областей: эконометрика, финансовая и актуарная математика~\cite{Antonov2016,Atskanov2016,Knyazev2016,  Penikas2014, Travkin2013,  Penikas2010,  Shemyakin2017}, задачах построения систем безопасности~\cite{Sundaresan2011}, биостатистике, гидрологии, климатологии (см. ссылки в работе~\cite{Fantazzini2011}). В статье~\cite{Penikas2010} автор представил концептуальные подходы применения копулярных моделей для решения задач управления финансами, включая риск-менеджмент.
Детальному описанию копула\-функций и их свойств посвящены монографии~\cite{Joe2014, Nelsen1999}.

За последние годы предложены различные методы оценивания параметров копула\-функций, среди которых выделяют параметрические~\cite{Patton2006}, полупараметрические~\cite{Chen2006, Lourme2016}, а также непараметрическими методами~\cite{Fermanian2003, Kim2007}. 
Более того, в работe~\cite{Cherubini2004} авторы предлагают использовать <<смеси>> методов~\cite{Fantazzini2011}, которые позволили бы сэкономить время на вычислениях.

Исследования~\cite{Ane2003, Kole2007, Lourme2016, Xu2008} рыночного риска в рамках управления портфелем во многом сходны друг с другом и отличаются зачастую только используемыми данными и незначительными деталями в оценке параметров копулярных моделей. 
Среди комплексных исследований выделим статью Тьерри Ане и Сесиль Харуби~\cite{Ane2003}~--- одну из первых, в которой авторы использовали копулу Клейтона в качестве инструмента для определения структуры зависимости между акциями международного фондового индекса.
Александр Лурме и др.\,\cite{Lourme2016} фокусируются на тестировании возможностей использования Гауссовой и $t$-копул для решения задач риск-менеджмента. 
Они предлагают $d$-мерную компактную Гауссову и $t$\,--\,Стьюдента  доверительную область, внутри которой вектор из выборки многомерной случайной величины, равномерно распределённой на отрезке $[0, 1]$, попадает с вероятностью~$\alpha$.
Результаты исследований показали, что копула $t$\,--\,Стьюдента, использованная для построения VaR-модели, является более консервативной и адекватной по сравнению c Гауссовой.

Важность выбора правильной копулы для риск\-менеджмента была показана в работе~\cite{Kole2007}, в которой рассчитывался портфель акций, облигаций и сделок на недвижимость.
Были протестированы копулы Гауссовы, Стьюдента и Гумбеля для моделирования зависимости дневных доходностей, которые аппроксимируют данные для указанных выше активов. 
Затем по расчетам VaR было установлено, что гауссова копула слишком оптимистична в отношении преимуществ диверсификации активов, тогда как копула Гумбеля слишком пессимистична.

Аналитическое решение для нахождения меры чувствительности риска~--- CVaR~--- было представлено в статье~\cite{Stoyanov2013}, где авторы приводят формулы для вычислений в предположении, что многомерная случайная величина подчинается одному из следующих распределений -- нормального, устойчивого и Стьюдета.

В данном обзоре отметим группу ученых под руководством профессора С.\,Czado \cite{Czado2010, Dissmann2013, Klup2017}, которые активно развивают как математический аппарат копула-функций, так и расширяют функционал пакетов программ для работы с ними.

В русскоязычной периодической литературе существенное развитие аппарата и приложений копула-функций получило после цикла публикаций Д.\,Фантаццини~\cite{Fantazzini2011}. Кратко укажем  направления использования копула-моделей: моделирование инвестиционного портфеля~\cite{Penikas2010, Penikas2014, Travkin2013}, моделирование совместного распределения биржевых индексов~\cite{Knyazev2016}, оптимизации
инвестиционного портфеля~\cite{Atskanov2016}, прогнозирование курсов валют~\cite{Antonov2016}.

\section{Объект и методы исследования}
\label{section:object}

В данной работе исследуется модель оценивания инвестиционного риска посредством применения копулярных моделей.
В качестве исходных данных были использованы временные ряды дневных цен на фьючерсные контракты.
При моделировании в данной работе были использованы следующие копула-функции:
\begin{itemize}
    \item Гауссова (нормальная) копула;
    \item копула Стьюдента;
    \item R\-иерархическая копула (R-vine).
\end{itemize}

Для описания логарифмических доходностей финансовых временных рядов в качестве кандитатов были рассмотрены следующие четырехпараметрические распределения:
\begin{itemize}
    \item гиперболическое;
    \item устойчивое;
    \item Мейкснера.
\end{itemize}
%
Параметры этих частных (маргинальные) распределений, а также копула-функций будем оценивать с использованием двухступенчатого параметрического подхода.
Для представления копулярных фукнций будем использовать матричный способ, а также графовые модели.
При моделировании иерархических копул мы допускаем использование в качестве  степени свободы вещественное число.

Меры риска~--- VaR~и CVaR~--- будем рассчитывать применительно к оптимальному портфелю из выбранных фьючерсных контрактов.
Для определения долей оптимального портфеля сформулируем и решим оптимизационную задачу линейного программирования. В качестве оптимального портфеля будем использовать портфель, который имеет минимильное значение CVaR при заданных ограничениях.

Для вычисления точечных и интервальных оценок и характеристик мер риска VaR~и CVaR будем использовать метод Монте-Карло, примененный к оптимальному портфелю.
