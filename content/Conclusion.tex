\anonsection{Заключение}

В данной работе мы использовали математический аппарат копула-функций для исследования многомерной зависимости между финансовыми временными рядами и сравнения риск-метрик VaR и CVaR для управления портфелем.

В качестве исходных данных мы использовали четыре временных ряда из дневных цен закрытия фьючерсов на индекс РТС (RTS) и обыкновенные акции ПАО <<НорНикеля>> (GMKR), ПАО <<Сбербанка>> (SBEF) и ПАО <<Газпрома>> (GAZP).
В работе использовались наблюдения за два года: с 16 декабря 2015 по 16 декабря 2017 (504 наблюдения).

Для оценки параметров копулярных моделей применен двухэтапный полный параметрический %MLE-
метод. %\cite{Joe1997, Joe2014}.
Для каждого временного ряда была выполненена оценка параметров слудуюхщих четырехпараметрических распределений-кандидатов: гиперболического, устойчивого и Мейкснера.
Стастическое тестирование оценок распределений показало удовлетворительные результаты качества полученных значений параметоров для всех распределений.
%На основе результатов для дальнейшего моделирования %всех временных рядов было выбрано  
Распределение Мейкснера выбрано в качестве частного (маргинального) на основе результатов статистического критерия Крамера\,--\,фон~Мизеса.

Далее в работе была осуществлена оценка параметров для трех копулярных моделей: Гауссовой, $t$\,--\,Стьюдента и R-иерархической.
Оценка параметров для трех выбранных копула-моделей была произведена методом <<инверсии $\tau$ Кендалла>>.
Выбор семейств для иерархической копулы выполнялся на основе тестов Вуонга и Кларка.

%Статистическое тестирование качества оценки параметров копул показало, что модель $t$-копулы и R-vine копулы оказались наименее и наиболее адеватными соответственно.

Для выбранных активов RTS, SBRF, GAZR, GMKR был найдет оптимальный mean-CVaR-портфель с долями $\textbf{w}=\{0,050; 0,114; 0,384; 0,452\}$.
%Преимущество этой задачи заключается в том, что минимизация CVaR производится путём решения задачи линейного программирования \cite{Rock2000}.

Предложен алгоритм для расчёта точечных и интервальных оценок риск-метрик VaR и CVaR с использованием копулярных моделей, основанный на Монте-Карло моделировании.
Данный алгоритм был также усовершенстован за счёт проведения бутстрап-процедуры, благодаря которой были получены несмещенные оценки риск-метрик, а также доверительный интервал для них.
В результате было установлено, что исследуемые копула-модели перестают быть консервативными на уровне выше 99.0\%.

Как показали результаты моделирования для кривых VaR и CVaR на уровне 95\%, все используемые копула-модели являются консервативными, причём более точно оценивает риск иерархическая копула, в то время как $t$-Стьюдента копула его переоценивает.
Такое поведение двухпараметрической $t$-Стьюдента копулы было ожидаемо, так она более чувствительная, чем другие рассматриваемые копулы к объему выборки.
%Данные результаты были ожидаемы исходя из результатов тестов адекватности моделей копула.

Данная работа содержит три нововведения.
Во-первых, мы показали, что Гауссовы, Стьюдента и иерархические копулы могут быть использованы для представления многомерной зависимости в коротких временных рядах (только 504 наблюдения), в то время как применение копул изучалось только по отношению к длительным наблюдениям~\cite{Dissmann2013, Kole2007, Lourme2016}.
Во-вторых, мы использовали маргинальные распределения, отличные от нормального: гиперболическое~\cite{Barndoff1983}, устойчивое~\cite{Rachev2005} и Мейкснера~\cite{Schoutens2002}. Стоянов и др.~\cite{Stoyanov2013} рассматривают подобную задачу, используя только симметричные варианты распределений Стьюдента и устойчивого, а также обобщённое нормальное распределение.
В-третьих, при моделировании R-vine копула-функций мы допускаем использование в качестве  степени свободы -- вещественное число. 
%для двухпараметрических копул, которые существенно увеличивают гибкость и расширяют возможности использования копулярных моделей в рамках риск-менеджмента.

%# Мы заимствуем и сравниваем исторические P\&L с кривыми VaR и CVaR, оцениваемых с помощью моделей копул, на разных уровнях значимости.

%# Раздел~\ref{section:object} посвящён объекту и методам исследования. В разделе~\ref{section:methodology} описана методология модели. В разделе~\ref{section:results} описаны исходные данные, приведены результаты калибровки модели: оценённые параметры и результаты оптимизации портфеля, а также финальные значения риск\-метрик.