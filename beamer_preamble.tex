% Подписи к рисункам, таблицам
\setbeamertemplate{caption}[numbered]
\usepackage[labelsep=endash]{caption}
\captionsetup[figure]{justification=centering,name={Рисунок}}
\captionsetup[table]{justification=raggedright}

% Изменить размеры заголовков
\setbeamerfont{frametitle}{size=\Large}
\setbeamerfont{framesubtitle}{size=\LARGE}

% \mode<presentation>
% {
%   \usetheme{default}      % or try Darmstadt, Madrid, Warsaw, ...
%   \usecolortheme{default} % or try albatross, beaver, crane, ...
%   \usefonttheme{default}  % or try serif, structurebold, ...
%   \setbeamertemplate{navigation symbols}{}
%   \setbeamertemplate{caption}[numbered]
% } 

% Шрифт
\usefonttheme{serif}

% Рисунки
% \usepackage[final]{graphicx}
\graphicspath{{images/}}

\bibliographystyle{apalike}

% Списки
% \usepackage{enumitem}
% \setitemize{label=\usebeamerfont*{itemize item}
%   \usebeamercolor[fg]{itemize item}
%   \usebeamertemplate{itemize item}}
% % Уравнение
% \newbool{firstitem}
% \newenvironment{where}
% {\begin{itemize}[
%     label=\ifbool{firstitem}{{где}\global\boolfalse{firstitem}}{},
%     before=\booltrue{firstitem},
%     leftmargin=0pt,
%     itemindent=0.8cm,
%     labelwidth=\itemindent,
%     labelsep = 0pt,
%     align=left]}
% {\end{itemize}}

% Алгоритм
\usepackage{algorithm}
\usepackage{algpseudocode}
\renewcommand{\algorithmicrequire}{\textbf{Вход:}}
\renewcommand{\algorithmicensure}{\textbf{Выход:}}
\floatname{algorithm}{Алгоритм}
\newcounter{Algorithm}
\setcounter{Algorithm}{1}

\usepackage{tabularx}
\newcolumntype{Y}{>{\centering\arraybackslash}X}
\newcolumntype{R}{>{\raggedleft\arraybackslash}X}
\newcolumntype{L}{>{\raggedright\arraybackslash}X}
\usepackage{adjustbox}
\usepackage{multirow}
\usepackage{booktabs}
\usepackage{bigstrut}
\usepackage{color, colortbl}
\usepackage{xcolor}