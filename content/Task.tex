\setcounter{page}{2}
\begingroup

\small
\centering
\singlespacing
\renewcommand\tabularxcolumn[1]{p{#1}}

\MakeUppercase{\textbf{Задание для раздела}} \\
\MakeUppercase{\textbf{<<Финансовый менеджмент, ресурсоэффективность}} \\
\MakeUppercase{\textbf{и ресурсосбережение>>}}

\bigskip

\begin{tabularx}{\textwidth}{|>{\hsize=0.5\hsize}Y|>{\hsize=1.5\hsize}Y|}
    \multicolumn{2}{l}{Студенту:} \\
    \hline
    \footnotesize \textbf{Группа} & 
    \footnotesize \textbf{ФИО} \\
    \hline 
    0ВМ61\bigstrut 
    & Смагулову Даулету Серикбаевичу \\ \hline 
\end{tabularx}

\vspace{2ex}

\begin{tabularx}{\textwidth}
{|>{\hsize=0.14\hsize}L|>{\hsize=0.16\hsize}L|>{\hsize=0.3\hsize}L|>{\hsize=0.4\hsize}L|}
    \hline
    \footnotesize \textbf{Школа} & ИЯТШ & 
    \footnotesize \textbf{Отделение школы (НОЦ)} & экспериментальной физики
    \bigstrut \\ \hline
    \footnotesize \textbf{Уровень образования} & магистратура &
    \footnotesize \textbf{Направление\,/\,специальность} & 01.04.02 <<Прикладная математика и информатика>>
    \bigstrut \\ \hline
\end{tabularx}

% \vspace{1ex}

\begin{longtable}
{|p{0.5\textwidth-2\tabcolsep}|p{0.5\textwidth-2\tabcolsep}|}
\endfirsthead
\endhead
\endfoot
\endlastfoot
\hline
\multicolumn{2}{|l|}{\textbf{Исходные данные к разделу <<Финансовый менеджмент, ресурсоэффективность}} \\
\multicolumn{2}{|l|}{\textbf{и ресурсосбережение>>}}
\\ \hline
\begin{itshape}
\begin{tabular}{p{0.1cm}p{\hsize-0.6cm}}
1. & Стоимость ресурсов научного исследования (НИ): материально\-технических, энергетических, финансовых, информационных и человеческих
\end{tabular}
\end{itshape}
\vspace{\fill}
&
\begin{itshape}
    \begin{tabularx}{\hsize-2\tabcolsep}{>{\hsize=1.1\hsize}L >{\hsize=0.9\hsize}L}
        \multicolumn{2}{l}{Заработная плата} \\
        руководителя & 33 664 руб\,/\,мес; \\
        инженера & 9893 руб\,/\,мес; \\ 
        Интернет & 350 руб\,/\,мес; \bigstrut \\ 
        Электроэнергия & 5.8 руб\,/кВт\,$\cdot$\,ч ;
    \end{tabularx}
\end{itshape}
\\ \hline
% \begin{itshape}
% \begin{tabular}{p{0.1cm}p{\hsize-0.6cm}}
% 2. & Нормы и нормативы расходования ресурсов
% \end{tabular}
% \end{itshape}
% &
% \\ \hline
\begin{itshape}
\begin{tabular}{p{0.1cm}p{\hsize-0.6cm}}
2. & Используемая система налогообложения, ставки налогов, отчислений, дисконтирования и кредитования
\end{tabular}
\end{itshape}
&
\begin{itshape}
    \begin{tabularx}{\hsize}{>{\hsize=1.5\hsize}L >{\hsize=0.5\hsize}L}
        Районный коэффициент\bigstrut & 1.3 \\
        Коэффициент отчислений во внебюджетные фонды & 0.271
    \end{tabularx}
\end{itshape}
\\ \hline
\multicolumn{2}{|l|}{\textbf{Перечень вопросов, подлежащих исследованию, проектированию и разработке:}}
\\ \hline
\begin{itshape}
\begin{tabular}{p{0.1cm}p{\hsize-0.6cm}}
1. & Оценка коммерческого и инновационного потенциала НТИ
\end{tabular}
\end{itshape}
&
\textit{См. главу~\ref{F:comm}}
\\ \hline
% \begin{itshape}
% \begin{tabular}{p{0.1cm}p{\hsize-0.6cm}}
% 2. & Разработка устава научно-технического проекта
% \end{tabular}
% \end{itshape}
% &
% \\ \hline
\begin{itshape}
\begin{tabular}{p{0.1cm}p{\hsize-0.6cm}}
2. & Планирование процесса управления НТИ: структура и график проведения, бюджет, риски и организация закупок
\end{tabular}
\end{itshape}
&
\textit{См. главу~\ref{F:plan}}
% \\ \hline
% \begin{itshape}
% \begin{tabular}{p{0.1cm}p{\hsize-0.6cm}}
% 4. & Определение ресурсной, финансовой, экономической эффективности
% \end{tabular}
% \end{itshape}
% &
\\ \hline
\multicolumn{2}{|l|}{\textbf{Перечень графического материала}}
\\ \hline
\multicolumn{2}{|l|}{\parbox{\hsize-2\tabcolsep}{%
\begin{itshape}
\begin{enumerate}[labelsep=4pt,topsep=0pt,leftmargin=4pt]
    \item Сегментирование рынка (табл.~\ref{tab:F:segments})
    \item Сравнение конкурентных технических решений (табл.~\ref{tab:F:competitors})
    \item Матрица SWOT (табл.~\ref{tab:F:swot})
    \item Оценка готовности проекта к коммерциализации (табл.~\ref{tab:F:comm})
    \item Рабочая группа проекта (табл.~\ref{tab:F:group})
    \item Календарный план проекта (табл.~\ref{tab:F:plan})
    \item Диаграмма Ганта проведения НТИ (табл.~\ref{tab:F:gantt})
    \item Расчёт бюджета исследования (табл.~\ref{tab:F:bTime}\,--\,\ref{tab:F:budget})
    \item Реестр рисков (табл.~\ref{tab:F:risk})
\end{enumerate}
\end{itshape}
}}
\\ \hline

\end{longtable}

% \vspace{1ex}

\begin{tabularx}{\textwidth}
{|>{\hsize=1.5\hsize}L|>{\hsize=0.5\hsize}L|} \hline
\textbf{Дата выдачи задания для раздела по линейному графику}\bigstrut &
\\ \hline
\end{tabularx}

\vspace{2ex}

\begin{tabularx}{\textwidth}
{|>{\hsize=1.8\hsize}Y|>{\hsize=1.1\hsize}Y|>{\hsize=0.7\hsize}Y|>{\hsize=0.7\hsize}Y|>{\hsize=0.7\hsize}Y|}
    \multicolumn{5}{l}{\textbf{Задание выдал консультант:}} \bigstrut[t] \\
    \hline
    \scriptsize \textbf{Должность} 
        & \scriptsize \textbf{ФИО} 
        & \scriptsize \textbf{Учёная степень, звание} 
        & \scriptsize \textbf{Подпись} 
        & \scriptsize \textbf{Дата} \\
    \hline
    доцент отделения социально-гуманитарных наук & Меньшикова Е. В.\bigstrut & к. ф. н. & & \\ 
    \hline
\end{tabularx}

\vspace{2ex}

\begin{tabularx}{\textwidth}
{|>{\hsize=0.64\hsize}Y|>{\hsize=2.08\hsize}Y|>{\hsize=0.64\hsize}Y|>{\hsize=0.64\hsize}Y|}
    \multicolumn{4}{l}{\textbf{Задание принял к исполнению студент:}} \\
    \hline
    \scriptsize \textbf{Группа}
        & \scriptsize \textbf{ФИО}
        & \scriptsize \textbf{Подпись}
        & \scriptsize \textbf{Дата} \\
    \hline 
    0ВМ61\bigstrut & Смагулов Даулет Серикбаевич & & \\ 
    \hline
\end{tabularx}



\clearpage

\MakeUppercase{\textbf{Задание для раздела}} \\
\MakeUppercase{\textbf{<<Социальная ответственность>>}}

\bigskip

\begin{tabularx}{\textwidth}{|>{\hsize=0.5\hsize}Y|>{\hsize=1.5\hsize}Y|}
    \multicolumn{2}{l}{Студенту:} \\
    \hline
    \footnotesize \textbf{Группа} & 
    \footnotesize \textbf{ФИО} \\
    \hline 
    0ВМ61\bigstrut 
    & Смагулову Даулету Серикбаевичу \\ \hline 
\end{tabularx}

\vspace{2ex}

\begin{tabularx}{\textwidth}
{|>{\hsize=0.14\hsize}L|>{\hsize=0.16\hsize}L|>{\hsize=0.3\hsize}L|>{\hsize=0.4\hsize}L|}
    \hline
    \footnotesize \textbf{Школа} & ИЯТШ & 
    \footnotesize \textbf{Отделение школы (НОЦ)} & экспериментальной физики
    \bigstrut \\ \hline
    \footnotesize \textbf{Уровень образования} & магистратура &
    \footnotesize \textbf{Направление\,/\,специальность} & 01.04.02 <<Прикладная математика и информатика>>
    \bigstrut \\ \hline
\end{tabularx}

\vspace{2ex}

\textbf{Тема магистерской диссертации:}\\
\textbf{<<Копулярные модели для оценивания инвестиционного риска>>}

\begin{longtable}
{|p{\textwidth-2\tabcolsep}|}
\endfirsthead
\endhead
\endfoot
\endlastfoot
\hline
\textbf{Исходные данные к разделу <<Социальная ответственность>>}
\\ \hline
% \small
\begin{itshape}
\begin{tabular}{p{0.2cm}p{\hsize-0.7cm}}
1. & Целью данной работы является исследование копулярных методов применительно к оценке портфельного риска. \\
2. & Описание рабочего места (рабочей зоны) на предмет возникновения:
\end{tabular}
\begin{itemize}[leftmargin=1.2cm,topsep=-1ex,labelsep=0pt,labelwidth=0.5cm,after=\vspace{-\baselineskip}]
    \item вредных проявлений факторов производственной среды  (необходимо обеспечить оптимальные, в крайнем случае, допустимые значения микроклимата на рабочем месте, обеспечить комфортную освещённость рабочего места, уменьшить до допустимых пределов шум от персональной ЭВМ, вентиляции, обеспечить безопасные значения электромагнитных полей от персонального компьютера);
    \item опасных проявлений факторов производственной среды (в связи с присутствием электричества для питания энергоблока персонального компьютера и освещенности аудитории необходимо предусмотреть средства коллективной и индивидуальной защиты от электро-, пожаро- и взрывоопасности).
\end{itemize}
\end{itshape}
\\ \hline
\textbf{Перечень вопросов, подлежащих исследованию, проектированию и разработке:}
\\ \hline
\begin{itshape}
\begin{tabular}{p{0.2cm}p{\hsize-0.7cm}}
1. & Анализ выявленных вредных факторов проектируемой производственной среды в следующей последовательности:
\end{tabular}
\begin{itemize}[leftmargin=1.2cm,topsep=-1ex,labelsep=0pt,labelwidth=0.5cm,after=\vspace{-\baselineskip}]
    \item указывается воздействие фактора на организм человека;
    \item приводятся данные по оптимальным и допустимым значениям микроклимата на рабочем месте, перечисляются методы обеспечения этих значений; приводится расчет освещенности на рабочем месте;
    \item приводятся данные по реальным значениям шума на рабочем месте, разрабатываются мероприятия по защите персонала от шума, при этом приводятся значения предельно-допустимого уровня, средства коллективной защиты (СКЗ), средства индивидуальной защиты (СИЗ);
    \item приводятся данные по реальным значениям электромагнитных полей на рабочем месте от персонального компьютера, перечисляются СКЗ и СИЗ;
    \item приводятся допустимые нормы с необходимой размерностью (с ссылкой на соответствующий нормативно-технический документ);
    \item предлагаемые средства защиты (сначала коллективной защиты, затем – индивидуальные защитные средства).
\end{itemize}
\end{itshape}
\\ \hline
\begin{itshape}
\begin{tabular}{p{0.2cm}p{\hsize-0.7cm}}
2. & Анализ выявленных опасных факторов проектируемой произведённой среды в следующей последовательности
\end{tabular}
\begin{itemize}[leftmargin=1.2cm,topsep=-1ex,labelsep=0pt,labelwidth=0.5cm,after=\vspace{-\baselineskip}]
    \item приводятся данные по значениям напряжения используемого оборудования, классификация помещения по электробезопасности, допустимые безопасные для человека значения напряжения, тока и заземления; перечисляются СКЗ и СИЗ; приводится расчёт освещения рабочего места;
    \item приводится классификация пожароопасности помещений, указывается класс пожароопасности вашего помещения, перечисляются средства пожарообнаружения и принцип их работы, средства пожаротушения, принцип работы, назначение (какие пожары можно тушить, какие – нет), маркировка; 
    \item пожаровзрывобезопасность (причины, профилактические мероприятия).
\end{itemize}
\end{itshape}
\\ \hline
\begin{itshape}
\begin{tabular}{p{0.2cm}p{\hsize-0.7cm}}
3. & Охрана окружающей среды:
\end{tabular}
\begin{itemize}[leftmargin=1.2cm,topsep=-1ex,labelsep=0pt,labelwidth=0.5cm,after=\vspace{-\baselineskip}]
    \item анализ воздействия при работе на ПЭВМ на окружающую среду;
    \item наличие отходов (бумага, картриджи, компьютеры и т. д.);
    \item методы утилизации отходов.
\end{itemize}
\end{itshape}
\\ \hline
\begin{itshape}
\begin{tabular}{p{0.2cm}p{\hsize-0.7cm}}
4. & Защита в чрезвычайных ситуациях:
\end{tabular}
\begin{itemize}[leftmargin=1.2cm,topsep=-1ex,labelsep=0pt,labelwidth=0.5cm,after=\vspace{-\baselineskip}]
    \item выявление типичных аварийных ситуаций: сильных морозов, несанкционированного проникновения посторонних лиц;
    \item разработка превентивных мер по предупреждению ЧС, мер по повышению устойчивости объекта к данной ЧС, а также действий в результате возникшей ЧС и мер по ликвидации её последствий.
\end{itemize}
\end{itshape}
\\ \hline
\begin{itshape}
\begin{tabular}{p{0.2cm}p{\hsize-0.7cm}}
5. & Правовые и организационные вопросы обеспечения безопасности:
\end{tabular}
\begin{itemize}[leftmargin=1.2cm,topsep=-1ex,labelsep=0pt,labelwidth=0.5cm,after=\vspace{-\baselineskip}]
    \item специальные (характерные для проектируемой рабочей зоны) правовые нормы трудового законодательства: 
    ГОСТ 12.0.004-2015, ГОСТ 12.0.003-2015, ГОСТ 12.1.013-78, ГОСТ 12.1.038-82, СанПиН 2.2.2/2.4.1340-03, ГОСТ 12.1.003-83, СанПиН 2.2.4/2.1.8.10-32-2002, СанПиН 2.2.4.548-96, СНиП 23-05-95, ГОСТ 12.1.004-91, ГОСТ 12.1.010-76.
\end{itemize}
\end{itshape}
\\ \hline
\textbf{Перечень графического материала:}
\\ \hline
\begin{itshape}
\begin{tabular}{p{0.2cm}p{\hsize-0.7cm}}
1) & План размещения светильников; \\
2) & План эвакуации.
% 1) & План эвакуации; \\
% 2) & План размещения светильников на потолке рабочего помещения.
\end{tabular}
\end{itshape}
\\ \hline
\end{longtable}

% \begin{longtable}
% {|p{0.6\textwidth-2\tabcolsep}|p{0.4\textwidth-2\tabcolsep}|}
% \endfirsthead
% \endhead
% \endfoot
% \endlastfoot
% \hline
% \multicolumn{2}{|l|}{\textbf{Исходные данные к разделу <<Социальная ответственность>>}}\bigstrut \\
% \hline
% \footnotesize
% \begin{itshape}
% \begin{tabular}{p{0.1cm}p{\hsize-0.6cm}}
% 1. & Описание рабочего места (рабочей зоны) на предмет возникновения:
% \end{tabular}
% \begin{itemize}[leftmargin=1cm,topsep=-1ex,labelsep=0pt,labelwidth=0.45cm,after=\vspace{-\baselineskip}]
%     \item вредных проявлений факторов производственной среды  (метеоусловия, вредные вещества, освещение, шумы, вибрации, электромагнитные поля, ионизирующие излучения);
%     \item опасных проявлений факторов производственной среды (механической природы, термического характера, электрической, пожарной и взрывной природы);
%     \item негативного воздействия на окружающую природную среду (атмосферу, гидросферу, литосферу);
%     \item чрезвычайных ситуаций (техногенного, стихийного, экологического и социального характера).
% \end{itemize}
% \end{itshape}
% & 
% \footnotesize
% \\ \hline
% \footnotesize
% \begin{itshape}
% \begin{tabular}{p{0.1cm}p{\hsize-1cm}}
% 2. & Перечень законодательных и нормативных документов по теме.
% \end{tabular}
% \end{itshape}
% &
% \footnotesize
% \\ \hline
% \multicolumn{2}{|l|}{\textbf{Перечень вопросов, подлежащих исследованию, проектированию и разработке:}}\bigstrut
% \\ \hline
% \footnotesize
% \begin{itshape}
% \begin{tabular}{p{0.1cm}p{\hsize-1cm}}
% 1. & Анализ выявленных вредных факторов проектируемой производственной среды в следующей последовательности:
% \end{tabular}
% \begin{itemize}[leftmargin=1cm,topsep=-1ex,labelsep=0pt,labelwidth=0.45cm,after=\vspace{-\baselineskip}]
%     \item физико-химическая природа вредности, её связь с разрабатываемой  темой;
%     \item действие фактора на организм человека;
%     \item приведение допустимых норм с необходимой размерностью (со ссылкой на соответствующий нормативно-технический документ);
%     \item предлагаемые средства защиты 
%     (сначала коллективной защиты, затем – индивидуальные защитные средства).
% \end{itemize}
% \end{itshape}
% &
% \footnotesize
% \\ \hline
% \footnotesize
% \begin{itshape}
% \begin{tabular}{p{0.1cm}p{\hsize-1cm}}
% 2. & Анализ выявленных опасных факторов проектируемой произведённой среды в следующей последовательности:
% \end{tabular}
% \begin{itemize}[leftmargin=1cm,topsep=-1ex,labelsep=0pt,labelwidth=0.45cm,after=\vspace{-\baselineskip}]
%     \item механические опасности (источники, средства защиты;
%     \item термические опасности (источники, средства защиты);
%     \item электробезопасность (в т.ч. статическое электричество, молниезащита – источники, средства защиты);
%     \item пожаровзрывобезопасность (причины, профилактические мероприятия, первичные средства пожаротушения)
% \end{itemize}
% \end{itshape}
% & 
% \footnotesize
% \\ \hline
% \footnotesize
% \begin{itshape}
% \begin{tabular}{p{0.1cm}p{\hsize-1cm}}
% 3. & Охрана окружающей среды:
% \end{tabular}
% \begin{itemize}[leftmargin=1cm,topsep=-1ex,labelsep=0pt,labelwidth=0.45cm,after=\vspace{-\baselineskip}]
%     \item защита селитебной зоны
%     \item анализ воздействия объекта на атмосферу (выбросы);
%     \item анализ воздействия объекта на гидросферу (сбросы);
%     \item анализ воздействия объекта на литосферу (отходы);
%     \item разработать решения по обеспечению экологической безопасности со ссылками на НТД по охране окружающей среды.
% \end{itemize}
% \end{itshape}
% & 
% \footnotesize
% \\ \hline
% \footnotesize
% \begin{itshape}
% \begin{tabular}{p{0.1cm}p{\hsize-1cm}}
% 4. & Защита в чрезвычайных ситуациях:
% \end{tabular}
% \begin{itemize}[leftmargin=1cm,topsep=-1ex,labelsep=0pt,labelwidth=0.45cm,after=\vspace{-\baselineskip}]
%     \item перечень возможных ЧС на объекте;
%     \item выбор наиболее типичной ЧС;
%     \item разработка превентивных мер по предупреждению ЧС;
%     \item разработка мер по повышению устойчивости объекта к данной ЧС;
%     \item разработка действий в результате возникшей ЧС и мер по ликвидации её последствий.
% \end{itemize}
% \end{itshape}
% & 
% \footnotesize
% \\ \hline
% \footnotesize
% \begin{itshape}
% \begin{tabular}{p{0.1cm}p{\hsize-1cm}}
% 5. & Правовые и организационные вопросы обеспечения безопасности:
% \end{tabular}
% \begin{itemize}[leftmargin=1cm,topsep=-1ex,labelsep=0pt,labelwidth=0.45cm,after=\vspace{-\baselineskip}]
%     \item специальные (характерные для проектируемой рабочей зоны) правовые нормы трудового законодательства;
%     \item организационные мероприятия при компоновке рабочей зоны
% \end{itemize}
% \end{itshape}
% & 
% \footnotesize
% \\ \hline
% \multicolumn{2}{|l|}{\textbf{Перечень графического материала:}}\bigstrut
% \\ \hline
% \footnotesize
% \textit{При необходимости представить эскизные графические материалы к расчётному заданию (обязательно для специалистов и магистров)}
% &
% \\ \hline
% \end{longtable}

% \vspace{2ex}

\begin{tabularx}{\textwidth}
{|>{\hsize=1.5\hsize}L|>{\hsize=0.5\hsize}L|} \hline
\textbf{Дата выдачи задания для раздела по линейному графику}\bigstrut & 26.02.18 г.
\\ \hline
\end{tabularx}

\vspace{2ex}

\begin{tabularx}{\textwidth}
{|>{\hsize=1.8\hsize}Y|>{\hsize=1.1\hsize}Y|>{\hsize=0.7\hsize}Y|>{\hsize=0.7\hsize}Y|>{\hsize=0.7\hsize}Y|}
    \multicolumn{5}{l}{\textbf{Задание выдал консультант:}} \bigstrut[t] \\
    \hline
    \scriptsize \textbf{Должность} 
        & \scriptsize \textbf{ФИО} 
        & \scriptsize \textbf{Учёная степень, звание} 
        & \scriptsize \textbf{Подпись} 
        & \scriptsize \textbf{Дата} \\
    \hline
    профессор отделения общетехнических дисциплин & Федорчук Ю. М.\bigstrut & д. т. н. & & 26.02.18 г. \\ 
    \hline
\end{tabularx}

\vspace{2ex}

\begin{tabularx}{\textwidth}
{|>{\hsize=0.64\hsize}Y|>{\hsize=2.08\hsize}Y|>{\hsize=0.64\hsize}Y|>{\hsize=0.64\hsize}Y|}
    \multicolumn{4}{l}{\textbf{Задание принял к исполнению студент:}} \\
    \hline
    \scriptsize \textbf{Группа}
        & \scriptsize \textbf{ФИО}
        & \scriptsize \textbf{Подпись}
        & \scriptsize \textbf{Дата} \\
    \hline 
    0ВМ61\bigstrut & Смагулов Даулет Серикбаевич & & 26.02.18 г. \\ 
    \hline
\end{tabularx}

\endgroup

% \AtBeginEnvironment{tabular}{\setstretch{1.3}}
% \AtBeginEnvironment{tabularx}{\setstretch{1.3}}