% Моя преамбула

% Математические формулы
\usepackage{amsmath,amssymb,amsfonts,pxfonts}
\usepackage{unicode-math}
\usepackage{icomma}

% Переносы слов
\usepackage{polyglossia}
\setmainlanguage{russian}
\setotherlanguage{english}
\sloppy
\PolyglossiaSetup{russian}{indentfirst=true}
\renewcommand{\-}{\nobreak-\hspace{0pt}}

% Форматирование текста и математических формул
\setmainfont{XITS}
\setmathfont{XITS Math}
\usepackage[Symbol]{upgreek}
\usepackage[hyphenation,lastparline]{impnattypo}

% Оглавление и литература
\newcommand{\myRefName}{Список использованных источников}
\addto\captionsrussian{\renewcommand{\contentsname}{Оглавление}}
\usepackage[nottoc,notlot,notlof]{tocbibind}
\usepackage{csquotes}
% \usepackage[backend=biber,style=gost-numeric,giveninits=true,sorting=nyt]{biblatex}
% \usepackage[style=numeric,giveninits=true,sorting=nyt]{biblatex}
\usepackage[style=numeric,giveninits=true,sorting=nyt,language=autobib,autolang=other]{biblatex}
% \usepackage[%
%     backend=biber,% движок
%     bibencoding=utf8,% кодировка bib файла
%     sorting=nyt,%none,% настройка сортировки списка литературы
%     style=gost-numeric,% стиль цитирования и библиографии (по ГОСТ)
%     language=autobib,% получение языка из babel/polyglossia, default: autobib % если ставить autocite или auto, то цитаты в тексте с указанием страницы, получат указание страницы на языке оригинала
%     autolang=other,%other,% многоязычная библиография
%     clearlang=true,% внутренний сброс поля language, если он совпадает с языком из babel/polyglossia
%     defernumbers=true,% нумерация проставляется после двух компиляций, зато позволяет выцеплять библиографию по ключевым словам и нумеровать не из большего списка
%     sortcites=true,% сортировать номера затекстовых ссылок при цитировании (если в квадратных скобках несколько ссылок, то отображаться будут отсортированно, а не абы как)
%     doi=false,% Показывать или нет ссылки на DOI
%     isbn=false,% Показывать или нет ISBN, ISSN, ISRN
%     otherlangs=true,
% ]{biblatex}[2016/09/17]
\bibliography{references.bib}
% \bibliographystyle{abbrv}
\usepackage[titles]{tocloft}
\setlength\cftparskip{0pt}
\setlength{\cftbeforesecskip}{3pt}
\setlength{\cftbeforesubsecskip}{2pt}
\setlength{\cftbeforesubsubsecskip}{0pt}

% Абзац 15 мм
\parindent=15mm

% Жирный курсив
\newcommand{\boldit}[1]{\textbf{\textit{#1}}}

% Секции по центру
\usepackage{titlesec}
\titleformat{\section}{\centering\normalfont\bfseries}{\thesection}{0.2cm}{}
\titleformat{\subsection}[block]{\hspace{\parindent}\bfseries}{\thesubsection}{0.2cm}{}
\titleformat{\subsubsection}[block]{\hspace{\parindent}\bfseries}{\thesubsubsection}{0.2cm}{}
% Привязать заголовки к следующему абзацу
\usepackage{needspace}
% Секции без нумерации
\newcommand{\anonsection}[1]{\section*{#1}\addcontentsline{toc}{section}{#1}}
\newcommand{\anonsubsection}[1]{\subsection*{#1}\addcontentsline{toc}{subsection}{#1}}
% Нумерация картинок и таблиц
\usepackage{chngcntr}
\counterwithin{figure}{section}
\counterwithin{table}{section}
\counterwithin{equation}{section}
% Новая страница для секций
\newcommand{\sectionbreak}{\clearpage}

% Картинки
\usepackage[final]{graphicx}
\graphicspath{{images/}} % images folder
% Вставить картинку
\newcommand{\imgh}[5][h]
{
    \begin{figure}[#1]
    \centering
    \includegraphics[#3]{#2}
    \caption{#4}
    \label{ris:#5}
    \end{figure}
}
% example: \imgh{1.pdf}{width=0.93\textwidth}{Image}{img}
\usepackage[labelsep=endash,singlelinecheck=false]{caption}
\captionsetup*[table]{justification=raggedright}
\captionsetup*[figure]{justification=centering,name={Рисунок}}

% Поля страницы
\usepackage[left=3cm,right=1cm,top=2cm,bottom=2cm]{geometry}

% Полуторный интервал
\usepackage{setspace}
\newcommand\spread{1.3}
\spacing{\spread}
% Сохранить интервал в таблицах
\usepackage{etoolbox}

% Нумерация страниц
\usepackage{fancyhdr}
\pagestyle{fancy}
\fancyhead{} % No page header
\fancyfoot[L]{} % Empty
\fancyfoot[C]{} % Empty
\fancyfoot[R]{\thepage} % Remove headlines
% Remove header underlines
\renewcommand{\headrulewidth}{0pt}

% Таблицы
\usepackage{multirow, makecell}
\usepackage{tabularx}
% Горизонтальные линии
\usepackage{booktabs}
% Вертикальное выравнивание - по центру
\renewcommand\tabularxcolumn[1]{m{#1}}
\newcolumntype{Y}{>{\centering\arraybackslash}X}
\newcolumntype{R}{>{\raggedleft\arraybackslash}X}
\newcolumntype{L}{>{\raggedright\arraybackslash}X}
\newcolumntype{C}[1]{>{\HY\hspace{0pt}\centering\arraybackslash}m{#1}}
% hyphenation in a cell
\newcommand{\HY}{\hyphenpenalty=25\exhyphenpenalty=25}
% \newcolumntype{Z}{>{\HY\centering}X}
% re-enable hyphenation locally inside "Z" columns
% \renewcommand{\tabularxcolumn}[1]{>{\normalsize}b{#1}}
% Подгон больших таблиц под размер страницы
\usepackage{adjustbox}
\usepackage{bigstrut}
\usepackage{color, colortbl}
\usepackage{xcolor}
% Ячейка с диагональной линией
\usepackage{diagbox}
\setlength{\tabcolsep}{4pt}
% Межстрочный интервал в таблицах
% \AtBeginEnvironment{table}{\setstretch{1.3}}
% \AtBeginEnvironment{tabularx}{\setstretch{1.3}}
% \AtBeginEnvironment{tabular}{\setstretch{1.3}}
% Длинные таблицы
\usepackage{longtable}

% Аннотация
\renewcommand{\abstractname}{Аннотация}

% Пакет для алгоритмов
\usepackage{algorithm}
\usepackage{algpseudocode}
\renewcommand{\algorithmicrequire}{\textbf{Вход:}}
\renewcommand{\algorithmicensure}{\textbf{Выход:}}
\floatname{algorithm}{Алгоритм}
\AtBeginEnvironment{algorithmic}{\setstretch{1.3}}

% Списки
\usepackage{enumitem}
\setlist{
    topsep=4pt,
    itemsep=0ex,
    partopsep=1ex,
    parsep=0ex,
    labelwidth=.5\parindent,
    labelsep=0pt,
    leftmargin=\parindent+\labelsep+\labelwidth,
    align=left
}
\newbool{firstitem}
\newenvironment{where}
{\begin{itemize}[
    label=\ifbool{firstitem}{{где}\global\boolfalse{firstitem}}{},
    before=\booltrue{firstitem},
    leftmargin=0pt,
    labelwidth=0.85cm,
    itemindent=\labelwidth]}
{\end{itemize}}
\newenvironment{where*}
{\begin{itemize}[
    label=\ifbool{firstitem}{{where}\global\boolfalse{firstitem}}{},
    before=\booltrue{firstitem},
    leftmargin=0pt,
    labelwidth=1.5cm,
    itemindent=\labelwidth]}
{\end{itemize}}

% Многострочный комментарий
\usepackage{comment}

% Устранить висячие строки
\clubpenalty=10000
\widowpenalty=10000
\brokenpenalty=4991

% Символы, в том числе "галочка"
\usepackage{pifont}

% Диаграмма Гантта 
\usepackage{pgfgantt}
\newcommand{\supervisor}[4][]{\ganttbar[progress=0,progress label text=#1]{#2}{#3}{#4}}
\newcommand{\student}[4][]{\ganttbar[progress=100,progress label text=#1]{#2}{#3}{#4}}

% Перечисление
\usepackage{calc,array}
\newcounter{MyCounter}
\newcommand{\myItem}{\theMyCounter\stepcounter{MyCounter}}
% Нумерация кириллицей
\AddEnumerateCounter{\asbuk}{\russian@alph}{щ}
\AddEnumerateCounter{\Asbuk}{\russian@Alph}{Щ}
\makeatletter
\def\russian@Alph#1{\ifcase#1\or
  А\or Б\or В\or Г\or Д\or Е\or Ж\or
  И\or К\or Л\or М\or Н\or
  П\or Р\or С\or Т\or У\or Ф\or Х\or
  Ц\or Ш\or Щ\or Э\or Ю\or Я\else\xpg@ill@value{#1}{russian@Alph}\fi}
\def\russian@alph#1{\ifcase#1\or
  а\or б\or в\or г\or д\or е\or ж\or
  и\or к\or л\or м\or н\or
  п\or р\or с\or т\or у\or ф\or х\or
  ц\or ш\or щ\or э\or ю\or я\else\xpg@ill@value{#1}{russian@alph}\fi}
\makeatother
% Приложение
\newcounter{Appendix}
% \setcounter{Appendix}{1}
\renewcommand{\theAppendix}{\Alph{Appendix}}
\newcommand{\Appendix}{\refstepcounter{Appendix}\anonsection{Приложение \Alph{Appendix}}}

% Поворот страницы
\usepackage{lscape}

% Сортировка списка литературы
\DeclareSourcemap{
  \maps[datatype=bibtex]{
    \map{
      \step[fieldsource=language, match=russian, final]
      \step[fieldset=presort, fieldvalue={a}]
    }
    \map{
      \step[fieldsource=language, notmatch=russian, final]
      \step[fieldset=presort, fieldvalue={z}]
    }
  }
}
\def\cyrdash{\hbox to.8em{--\hss--}}
\DefineBibliographyExtras{russian}{%
  \protected\def\bibrangedash{\cyrdash}%
    % \cyrdash\penalty\value{abbrvpenalty}}% almost unbreakable dash
  \protected\def\bibdaterangesep{\bibrangedash}%тире для дат
}
\DefineBibliographyExtras{english}{%
  \protected\def\bibrangedash{\cyrdash}%
    % \cyrdash\penalty\value{abbrvpenalty}}% almost unbreakable dash
  \protected\def\bibdaterangesep{\bibrangedash}%тире для дат
}

% Счётчики
\usepackage[figure,table]{totalcount} 
\usepackage{totcount,assoccnt}
\usepackage{totpages}

\newtotcounter{appendixNum}
\DeclareAssociatedCounters{Appendix}{appendixNum}

\AtBeginDocument{
%% регистрируем счётчики в системе totcounter
    \regtotcounter{totalcount@figure}
    \regtotcounter{totalcount@table}
    \regtotcounter{TotPages}
    \regtotcounter{appendixNum}
}

\newtotcounter{citenum}
\makeatletter
\defbibenvironment{counter} %Env of bibliography
  {\setcounter{citenum}{0}%
  \renewcommand{\blx@driver}[1]{}%
  } %what is doing at the beginining of bibliography. In your case it's : a. Reset counter b. Say to print nothing when a entry is tested.
  {} %Здесь то, что будет выводиться командой \printbibliography. \thecitenum сюда писать не надо
  {\stepcounter{citenum}} %What is printing / executed at each entry.
\makeatother
\defbibheading{counter}{}

% Склонять существительное по числительному
\makeatletter
\def\formbytotal#1#2#3#4#5{%
    \newcount\@c
    \@c\totvalue{#1}\relax
    \newcount\@last
    \newcount\@pnul
    \@last\@c\relax
    \divide\@last 10
    \@pnul\@last\relax
    \divide\@pnul 10
    \multiply\@pnul-10
    \advance\@pnul\@last
    \multiply\@last-10
    \advance\@last\@c
    \total{#1}~#2%
    \ifnum\@pnul=1#5\else%
    \ifcase\@last#5\or#3\or#4\or#4\or#4\else#5\fi
    \fi
}
\makeatother