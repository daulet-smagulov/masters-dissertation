\newpage

\anonsection{Заключение}

В данной работе я использовал копулярные модели для исследования многомерной зависимости между финансовыми временными рядами и сравнения риск метрик для управления портфелем.
В качестве исследуемых переменных я использовал четыре временных ряда из дневных цен закрытия фьючерсов на индекс РТС и акции <<НорНикеля>>, <<Сбербанка>> и <<Газпрома>>.
В работе использовались наблюдения за два года: с 16 декабря 2015 по 16 декабря 2017.

Я использовал двухэтапный параметрический подход оценки максимального правдоподобия.
Для каждого из активов был произведён фитинг трёх распределений: гиперболического, устойчивого и Мейкснера.
Тестирование качества фитинга показало удовлетворительные результаты оценки качества фитинга для всех распределений.
На основе результатов теста Крамера\,--\,фон~Мизеса для каждого из активов было выбрано распределение Мейкснера в качестве маргинального.

Далее для имеющегося портфеля активов был произведён фитинг трёх распределения копул: Гауссовой, $t$\,--\,Стьюдента и R-vine.
Выбор семейств вайн копулы выполнялся на основе тестов Вуонга и Кларка.
Оценка параметров моделей была произведена методом <<Инверсии $\tau$ Кендалла>>.
Тестирование качества оценки параметров копул показало, что модель $t$ и R-vine копул оказались наименее и наиболее адеватными соответственно.

Был предложен алгоритм для расчёта риск метрик с использованием копулярных моделей, основанный на Монте-Карло симуляциях.
Этот алгоритм был также усовершенстован за счёт првоедения бутстрап-процедуры, благодаря которой были получены более точные оценки риск метрик и доверительный интервал.
В результате было выяснено, что исследуемая модель перестаёт быть консервативной на уровне больше 99\%.

Как показал график кривых VaR и CVaR на уровне 95\%, а также тест Купича, все модели являются консервативными, причём более точно оценивает риск вайн копула, в то время как $t$ копула его переоценивает.
Данные результаты были ожидаемы исходя из результатов тестов адекватности моделей копула.