\setcounter{page}{2}
\begingroup

\small
\centering
\singlespacing
\renewcommand\tabularxcolumn[1]{p{#1}}

\MakeUppercase{\textbf{Задание для раздела}} \\
\MakeUppercase{\textbf{<<Финансовый менеджмент, ресурсоэффективность}} \\
\MakeUppercase{\textbf{и ресурсосбережение>>}}

\bigskip

\begin{tabularx}{\textwidth}{|>{\hsize=0.5\hsize}Y|>{\hsize=1.5\hsize}Y|}
    \multicolumn{2}{l}{Студенту:} \\
    \hline
    \footnotesize \textbf{Группа} & 
    \footnotesize \textbf{ФИО} \\
    \hline 
    0ВМ61\bigstrut 
    & Смагулову Даулету Серикбаевичу \\ \hline 
\end{tabularx}

\vspace{2ex}

\begin{tabularx}{\textwidth}
{|>{\hsize=0.14\hsize}L|>{\hsize=0.16\hsize}L|>{\hsize=0.3\hsize}L|>{\hsize=0.4\hsize}L|}
    \hline
    \footnotesize \textbf{Школа} & ИЯТШ & 
    \footnotesize \textbf{Отделение школы (НОЦ)} & экспериментальной физики
    \bigstrut \\ \hline
    \footnotesize \textbf{Уровень образования} & магистратура &
    \footnotesize \textbf{Направление\,/\,специальность} & 01.04.02 <<Прикладная математика и информатика>>
    \bigstrut \\ \hline
\end{tabularx}

% \vspace{1ex}

\begin{longtable}
{|p{0.5\textwidth-2\tabcolsep}|p{0.5\textwidth-2\tabcolsep}|}
\endfirsthead
\endhead
\endfoot
\endlastfoot
\hline
\multicolumn{2}{|l|}{\textbf{Исходные данные к разделу <<Финансовый менеджмент, ресурсоэффективность}} \\
\multicolumn{2}{|l|}{\textbf{и ресурсосбережение>>}}
\\ \hline
\begin{itshape}
\begin{tabular}{p{0.1cm}p{\hsize-0.6cm}}
1. & Стоимость ресурсов научного исследования (НИ): материально\-технических, энергетических, финансовых, информационных и человеческих
\end{tabular}
\end{itshape}
\vspace{\fill}
&
\begin{itshape}
    \begin{tabularx}{\hsize-2\tabcolsep}{>{\hsize=1.1\hsize}L >{\hsize=0.9\hsize}L}
        \multicolumn{2}{l}{Заработная плата} \\
        руководителя & 33 664 руб\,/\,мес; \\
        инженера & 9893 руб\,/\,мес; \\ 
        Интернет & 350 руб\,/\,мес; \bigstrut \\ 
        Электроэнергия & 5.8 руб\,/кВт\,$\cdot$\,ч ;
    \end{tabularx}
\end{itshape}
\\ \hline
% \begin{itshape}
% \begin{tabular}{p{0.1cm}p{\hsize-0.6cm}}
% 2. & Нормы и нормативы расходования ресурсов
% \end{tabular}
% \end{itshape}
% &
% \\ \hline
\begin{itshape}
\begin{tabular}{p{0.1cm}p{\hsize-0.6cm}}
2. & Используемая система налогообложения, ставки налогов, отчислений, дисконтирования и кредитования
\end{tabular}
\end{itshape}
&
\begin{itshape}
    \begin{tabularx}{\hsize}{>{\hsize=1.5\hsize}L >{\hsize=0.5\hsize}L}
        Районный коэффициент\bigstrut & 1.3 \\
        Коэффициент отчислений во внебюджетные фонды & 0.271
    \end{tabularx}
\end{itshape}
\\ \hline
\multicolumn{2}{|l|}{\textbf{Перечень вопросов, подлежащих исследованию, проектированию и разработке:}}
\\ \hline
\begin{itshape}
\begin{tabular}{p{0.1cm}p{\hsize-0.6cm}}
1. & Оценка коммерческого и инновационного потенциала НТИ
\end{tabular}
\end{itshape}
&
\textit{См. главу~\ref{F:comm}}
\\ \hline
% \begin{itshape}
% \begin{tabular}{p{0.1cm}p{\hsize-0.6cm}}
% 2. & Разработка устава научно-технического проекта
% \end{tabular}
% \end{itshape}
% &
% \\ \hline
\begin{itshape}
\begin{tabular}{p{0.1cm}p{\hsize-0.6cm}}
2. & Планирование процесса управления НТИ: структура и график проведения, бюджет, риски и организация закупок
\end{tabular}
\end{itshape}
&
\textit{См. главу~\ref{F:plan}}
% \\ \hline
% \begin{itshape}
% \begin{tabular}{p{0.1cm}p{\hsize-0.6cm}}
% 4. & Определение ресурсной, финансовой, экономической эффективности
% \end{tabular}
% \end{itshape}
% &
\\ \hline
\multicolumn{2}{|l|}{\textbf{Перечень графического материала}}
\\ \hline
\multicolumn{2}{|l|}{\parbox{\hsize-2\tabcolsep}{%
\begin{itshape}
\begin{enumerate}[labelsep=4pt,topsep=0pt,leftmargin=4pt]
    \item Сегментирование рынка (табл.~\ref{tab:F:segments})
    \item Сравнение конкурентных технических решений (табл.~\ref{tab:F:competitors})
    \item Матрица SWOT (табл.~\ref{tab:F:swot})
    \item Оценка готовности проекта к коммерциализации (табл.~\ref{tab:F:comm})
    \item Рабочая группа проекта (табл.~\ref{tab:F:group})
    \item Календарный план проекта (табл.~\ref{tab:F:plan})
    \item Диаграмма Ганта проведения НТИ (табл.~\ref{tab:F:gantt})
    \item Расчёт бюджета исследования (табл.~\ref{tab:F:bTime}\,--\,\ref{tab:F:budget})
    \item Реестр рисков (табл.~\ref{tab:F:risk})
\end{enumerate}
\end{itshape}
}}
\\ \hline

\end{longtable}

% \vspace{1ex}

\begin{tabularx}{\textwidth}
{|>{\hsize=1.5\hsize}L|>{\hsize=0.5\hsize}L|} \hline
\textbf{Дата выдачи задания для раздела по линейному графику}\bigstrut &
\\ \hline
\end{tabularx}

\vspace{2ex}

\begin{tabularx}{\textwidth}
{|>{\hsize=1.8\hsize}Y|>{\hsize=1.1\hsize}Y|>{\hsize=0.7\hsize}Y|>{\hsize=0.7\hsize}Y|>{\hsize=0.7\hsize}Y|}
    \multicolumn{5}{l}{\textbf{Задание выдал консультант:}} \bigstrut[t] \\
    \hline
    \scriptsize \textbf{Должность} 
        & \scriptsize \textbf{ФИО} 
        & \scriptsize \textbf{Учёная степень, звание} 
        & \scriptsize \textbf{Подпись} 
        & \scriptsize \textbf{Дата} \\
    \hline
    доцент отделения социально-гуманитарных наук & Меньшикова Е. В.\bigstrut & к. ф. н. & & \\ 
    \hline
\end{tabularx}

\vspace{2ex}

\begin{tabularx}{\textwidth}
{|>{\hsize=0.64\hsize}Y|>{\hsize=2.08\hsize}Y|>{\hsize=0.64\hsize}Y|>{\hsize=0.64\hsize}Y|}
    \multicolumn{4}{l}{\textbf{Задание принял к исполнению студент:}} \\
    \hline
    \scriptsize \textbf{Группа}
        & \scriptsize \textbf{ФИО}
        & \scriptsize \textbf{Подпись}
        & \scriptsize \textbf{Дата} \\
    \hline 
    0ВМ61\bigstrut & Смагулов Даулет Серикбаевич & & \\ 
    \hline
\end{tabularx}



\clearpage

\MakeUppercase{\textbf{Задание для раздела}} \\
\MakeUppercase{\textbf{<<Социальная ответственность>>}}

\bigskip

\begin{tabularx}{\textwidth}{|>{\hsize=0.5\hsize}Y|>{\hsize=1.5\hsize}Y|}
    \multicolumn{2}{l}{Студенту:} \\
    \hline
    \footnotesize \textbf{Группа} & 
    \footnotesize \textbf{ФИО} \\
    \hline 
    0ВМ61\bigstrut 
    & Смагулову Даулету Серикбаевичу \\ \hline 
\end{tabularx}

\vspace{2ex}

\begin{tabularx}{\textwidth}
{|>{\hsize=0.14\hsize}L|>{\hsize=0.16\hsize}L|>{\hsize=0.3\hsize}L|>{\hsize=0.4\hsize}L|}
    \hline
    \footnotesize \textbf{Школа} & ИЯТШ & 
    \footnotesize \textbf{Отделение школы (НОЦ)} & экспериментальной физики
    \bigstrut \\ \hline
    \footnotesize \textbf{Уровень образования} & магистратура &
    \footnotesize \textbf{Направление\,/\,специальность} & 01.04.02 <<Прикладная математика и информатика>>
    \bigstrut \\ \hline
\end{tabularx}

\vspace{2ex}

\textbf{Тема магистерской диссертации:}\\
\textbf{<<Копулярные модели для оценивания инвестиционного риска>>}

\begin{longtable}
{|p{\textwidth-2\tabcolsep}|}
\endfirsthead
\endhead
\endfoot
\endlastfoot
\hline
\textbf{Исходные данные к разделу <<Социальная ответственность>>}
\\ \hline
% \small
\begin{itshape}
\begin{tabular}{p{0.2cm}p{\hsize-0.7cm}}
1. & Целью данной работы является исследование копулярных методов применительно к оценке портфельного риска. \\
2. & Описание рабочего места (рабочей зоны) на предмет возникновения:
\end{tabular}
\begin{itemize}[leftmargin=1.2cm,topsep=-1ex,labelsep=0pt,labelwidth=0.5cm,after=\vspace{-\baselineskip}]
    \item вредных проявлений факторов производственной среды  (необходимо обеспечить оптимальные, в крайнем случае, допустимые значения микроклимата на рабочем месте, обеспечить комфортную освещённость рабочего места, уменьшить до допустимых пределов шум от персональной ЭВМ, вентиляции, обеспечить безопасные значения электромагнитных полей от персонального компьютера);
    \item опасных проявлений факторов производственной среды (в связи с присутствием электричества для питания энергоблока персонального компьютера и освещенности аудитории необходимо предусмотреть средства коллективной и индивидуальной защиты от электро-, пожаро- и взрывоопасности).
\end{itemize}
\end{itshape}
\\ \hline
\textbf{Перечень вопросов, подлежащих исследованию, проектированию и разработке:}
\\ \hline
\begin{itshape}
\begin{tabular}{p{0.2cm}p{\hsize-0.7cm}}
1. & Анализ выявленных вредных факторов проектируемой производственной среды в следующей последовательности:
\end{tabular}
\begin{itemize}[leftmargin=1.2cm,topsep=-1ex,labelsep=0pt,labelwidth=0.5cm,after=\vspace{-\baselineskip}]
    \item указывается воздействие фактора на организм человека;
    \item приводятся данные по оптимальным и допустимым значениям микроклимата на рабочем месте, перечисляются методы обеспечения этих значений; приводится расчет освещенности на рабочем месте;
    \item приводятся данные по реальным значениям шума на рабочем месте, разрабатываются мероприятия по защите персонала от шума, при этом приводятся значения предельно-допустимого уровня, средства коллективной защиты (СКЗ), средства индивидуальной защиты (СИЗ);
    \item приводятся данные по реальным значениям электромагнитных полей на рабочем месте от персонального компьютера, перечисляются СКЗ и СИЗ;
    \item приводятся допустимые нормы с необходимой размерностью (с ссылкой на соответствующий нормативно-технический документ);
    \item предлагаемые средства защиты (сначала коллективной защиты, затем – индивидуальные защитные средства).
\end{itemize}
\end{itshape}
\\ \hline
\begin{itshape}
\begin{tabular}{p{0.2cm}p{\hsize-0.7cm}}
2. & Анализ выявленных опасных факторов проектируемой произведённой среды в следующей последовательности
\end{tabular}
\begin{itemize}[leftmargin=1.2cm,topsep=-1ex,labelsep=0pt,labelwidth=0.5cm,after=\vspace{-\baselineskip}]
    \item приводятся данные по значениям напряжения используемого оборудования, классификация помещения по электробезопасности, допустимые безопасные для человека значения напряжения, тока и заземления; перечисляются СКЗ и СИЗ; приводится расчёт освещения рабочего места;
    \item приводится классификация пожароопасности помещений, указывается класс пожароопасности вашего помещения, перечисляются средства пожарообнаружения и принцип их работы, средства пожаротушения, принцип работы, назначение (какие пожары можно тушить, какие – нет), маркировка; 
    \item пожаровзрывобезопасность (причины, профилактические мероприятия).
\end{itemize}
\end{itshape}
\\ \hline
\begin{itshape}
\begin{tabular}{p{0.2cm}p{\hsize-0.7cm}}
3. & Охрана окружающей среды:
\end{tabular}
\begin{itemize}[leftmargin=1.2cm,topsep=-1ex,labelsep=0pt,labelwidth=0.5cm,after=\vspace{-\baselineskip}]
    \item анализ воздействия при работе на ПЭВМ на окружающую среду;
    \item наличие отходов (бумага, картриджи, компьютеры и т. д.);
    \item методы утилизации отходов.
\end{itemize}
\end{itshape}
\\ \hline
\begin{itshape}
\begin{tabular}{p{0.2cm}p{\hsize-0.7cm}}
4. & Защита в чрезвычайных ситуациях:
\end{tabular}
\begin{itemize}[leftmargin=1.2cm,topsep=-1ex,labelsep=0pt,labelwidth=0.5cm,after=\vspace{-\baselineskip}]
    \item выявление типичных аварийных ситуаций: сильных морозов, несанкционированного проникновения посторонних лиц;
    \item разработка превентивных мер по предупреждению ЧС, мер по повышению устойчивости объекта к данной ЧС, а также действий в результате возникшей ЧС и мер по ликвидации её последствий.
\end{itemize}
\end{itshape}
\\ \hline
\begin{itshape}
\begin{tabular}{p{0.2cm}p{\hsize-0.7cm}}
5. & Правовые и организационные вопросы обеспечения безопасности:
\end{tabular}
\begin{itemize}[leftmargin=1.2cm,topsep=-1ex,labelsep=0pt,labelwidth=0.5cm,after=\vspace{-\baselineskip}]
    \item специальные (характерные для проектируемой рабочей зоны) правовые нормы трудового законодательства: 
    ГОСТ 12.0.004-2015, ГОСТ 12.0.003-2015, ГОСТ 12.1.013-78, ГОСТ 12.1.038-82, СанПиН 2.2.2/2.4.1340-03, ГОСТ 12.1.003-83, СанПиН 2.2.4/2.1.8.10-32-2002, СанПиН 2.2.4.548-96, СНиП 23-05-95, ГОСТ 12.1.004-91, ГОСТ 12.1.010-76.
\end{itemize}
\end{itshape}
\\ \hline
\textbf{Перечень графического материала:}
\\ \hline
\begin{itshape}
\begin{tabular}{p{0.2cm}p{\hsize-0.7cm}}
1) & План размещения светильников; \\
2) & План эвакуации.
% 1) & План эвакуации; \\
% 2) & План размещения светильников на потолке рабочего помещения.
\end{tabular}
\end{itshape}
\\ \hline
\end{longtable}

\begin{tabularx}{\textwidth}
{|>{\hsize=1.5\hsize}L|>{\hsize=0.5\hsize}L|} \hline
\textbf{Дата выдачи задания для раздела по линейному графику}\bigstrut & 26.02.18 г.
\\ \hline
\end{tabularx}

\vspace{2ex}

\begin{tabularx}{\textwidth}
{|>{\hsize=1.8\hsize}Y|>{\hsize=1.1\hsize}Y|>{\hsize=0.7\hsize}Y|>{\hsize=0.7\hsize}Y|>{\hsize=0.7\hsize}Y|}
    \multicolumn{5}{l}{\textbf{Задание выдал консультант:}} \bigstrut[t] \\
    \hline
    \scriptsize \textbf{Должность} 
        & \scriptsize \textbf{ФИО} 
        & \scriptsize \textbf{Учёная степень, звание} 
        & \scriptsize \textbf{Подпись} 
        & \scriptsize \textbf{Дата} \\
    \hline
    профессор отделения общетехнических дисциплин & Федорчук Ю. М.\bigstrut & д. т. н. & & 26.02.18 г. \\ 
    \hline
\end{tabularx}

\vspace{2ex}

\begin{tabularx}{\textwidth}
{|>{\hsize=0.64\hsize}Y|>{\hsize=2.08\hsize}Y|>{\hsize=0.64\hsize}Y|>{\hsize=0.64\hsize}Y|}
    \multicolumn{4}{l}{\textbf{Задание принял к исполнению студент:}} \\
    \hline
    \scriptsize \textbf{Группа}
        & \scriptsize \textbf{ФИО}
        & \scriptsize \textbf{Подпись}
        & \scriptsize \textbf{Дата} \\
    \hline 
    0ВМ61\bigstrut & Смагулов Даулет Серикбаевич & & 26.02.18 г. \\ 
    \hline
\end{tabularx}

\clearpage

\MakeUppercase{\textbf{Заланированные результаты обучения по ООП}}

\begin{longtable}
{|>{\raggedright\arraybackslash}p{0.15\textwidth-2\tabcolsep}|>{\raggedright\arraybackslash}p{0.85\textwidth-2\tabcolsep}|}
\endfirsthead
\endhead
\endfoot
\endlastfoot
\hline
\parbox{\hsize}{\strut\centering\textbf{Код результата}\strut} & \parbox{\hsize}{\strut\centering\textbf{Результаты обучения\\(выпускник должен быть готов)}\strut}
% \\
% \centering\textbf{результата} & \textbf{(выпускник должен быть готов)}
\\ \hline
\multicolumn{2}{|c|}{Профессиональные компетенции\setcounter{MyCounter}{1}}
\\ \hline
ПК-\myItem & Самостоятельная работа
\\ \hline
ПК-\myItem & Использовать современные прикладные программные средства и осваивать современные технологии программирования
\\ \hline
ПК-\myItem & Использовать стандартные пакеты прикладных программ для решения практических задач на ЭВМ, отлаживать, тестировать прикладное программное обеспечение
\\ \hline
ПК-\myItem & Настраивать, тестировать и осуществлять проверку вычислительной техники и программных средств
\\ \hline
ПК-\myItem & Демонстрировать знание современных языков программирования, операционных систем, офисных приложений, Интернета, способов и механизмов управления данными; принципов организации, состава и схемы работы операционных систем
\\ \hline
ПК-\myItem & Решать проблемы, брать на себя ответственность
\\ \hline
ПК-\myItem & Проводить организационно-управленческие расчеты, осуществлять организацию и техническое оснащение рабочих мест
\\ \hline
ПК-\myItem & Организовывать работу малых групп исполнителей
\\ \hline
ПК-\myItem & Определять экономическую целесообразность принимаемых технических и организационных решений
\\ \hline
ПК-\myItem & Владеть основными методами защиты производственного персонала и населения от возможных последствий аварий, катастроф, стихийных бедствий
\\ \hline
ПК-\myItem & Знать основные положения законы и методы естественных наук; выявлять естественнонаучную сущность проблем, возникающих в ходе профессиональной деятельности, использовать для их решения соответствующий естественнонаучный аппарат
\\ \hline
ПК-\myItem & Применять математический аппарат для решения поставленных задач, использовать соответствующую процессу математическую модель и проверять ее адекватность
\\ \hline
ПК-\myItem & Применять знания и навыки управления информацией
\\ \hline
ПК-\myItem & Самостоятельно изучать новые разделы фундаментальных наук
\\ \hline
\multicolumn{2}{|c|}{Универсальные компетенции\setcounter{MyCounter}{1}}
\\ \hline
УК-\myItem & Владеть культурой мышления, иметь способности к обобщению, анализу, восприятию информации, постановке цели и выбору путей ее достижения
\\ \hline
УК-\myItem & Логически верно, аргументировано и ясно строить устную и письменную речь
\\ \hline
УК-\myItem & Уважительно и бережно относится к историческому наследию и культурным традициям, толерантно воспринимать социальные и культурные различия; понимать движущие силы и закономерности исторического процесса, место человека в историческом процессе, политической организации общества
\\ \hline
УК-\myItem & Понимать и анализировать мировоззренческие, социально и личностно значимые философские проблемы
\\ \hline
УК-\myItem & Владеть одним из иностранных языков на уровне бытового общения, а также переводить профессиональные тексты с иностранного языка
\\ \hline
УК-\myItem & К кооперации с коллегами, работе в коллективе
\\ \hline
УК-\myItem & Находить организационно-управленческие решения в нестандартных ситуациях и готов нести за них ответственность
\\ \hline
УК-\myItem & Использовать нормативно-правовые документы в своей деятельности
\\ \hline
УК-\myItem & Стремиться к саморазвитию, повышению своей квалификации и мастерства
\\ \hline
УК-\myItem & Осознавать социальную значимость своей будущей профессии, обладать высокой мотивацией к выполнению профессиональной деятельности
\\ \hline
УК-\myItem & Использовать основные положения и методы социальных, гуманитарных и экономических наук при решении социальных и профессиональных задач
\\ \hline
УК-\myItem & Анализировать социально значимые проблемы и процессы
\\ \hline
УК-\myItem & Использовать основные законы естественнонаучных дисциплин в профессиональной деятельности, применять методы математического анализа и моделирования, теоретического и экспериментального исследования
\\ \hline
УК-\myItem & Понимать сущность и значение информации в развитии современного информационного общества, осознавать опасности и угрозы, возникающие в этом процессе, соблюдать основные требования информационной безопасности, в том числе защиты государственной тайны
\\ \hline
УК-\myItem & Оформлять, представлять и докладывать результаты выполненной работы
\\ \hline
УК-\myItem & Создавать и редактировать тексты профессионального назначения
\\ \hline
УК-\myItem & Использовать для решения коммуникативных задач современные технические средства и информационные технологии
\\ \hline
УК-\myItem & Владеть средствами самостоятельного, методически правильного использования методов физического воспитания и укрепления здоровья, быть способным к достижению должного уровня физической подготовленности для обеспечения полноценной социальной и профессиональной деятельности
\\ \hline
\end{longtable}

\endgroup