\section{Финансовый менеджмент, \\ресурсоэффективность и ресурсосбережение}
\label{section:financial}

% \anonsubsection{Введение}

% Целью данного раздела является проектирование и создание конкурентоспособных разработок и технологий, отвечающих предъявляемым требованиям в области ресурсоэффективности и ресурсосбережения.

% % Достижение цели обеспечивается решением задач:

% % \begin{itemize}
% %     \item разработка общей экономической идеи проекта, формирование концепции проекта;                 
% %     \item организация работ по научно-исследовательскому проекту;
% %     \item определение возможных альтернатив проведения научных исследований; 
% %     \item планирование научно-исследовательских работ; 
% %     \item оценки коммерческого потенциала и перспективности проведения научных исследований с позиции ресурсоэффективности и ресурсосбережения; 
% %     \item определение ресурсной (ресурсосберегающей), финансовой, бюджетной, социальной и экономической эффективности исследования.
% % \end{itemize}

% В данном разделе представлены итоги следующих этапов управления научной работой:
% \begin{enumerate}
%     \item Инициация проекта.
%     \item Планирование проекта.
%     \item Исполнение проекта.
% \end{enumerate}

\subsection{Предпроектный анализ}

\subsubsection{Потенциальные потребители результатов исследования}

Исследование, проведённое в данной работе, имеет непосредственное отношение к риск-менеджменту и инвестициям. 
Потенциальными потребителями её результатов могут быть как инвесторы, владеющие портфелем, так и банки и банковские организации. 
Также данными исследованиями могут воспользоваться государственные учреждения, интернет-ресурсы и другие связанные с финансовыми рисками организации.

Подобные организации пользуются преимущественно методом исторического моделирования для оценивания портфельного риска.
Также весьма распространён метод Монте-Карло, используемый для симуляции поведения ценных бумаг и дальнейшего расчёта VaR, CVaR, коэффициента Шарпа и других риск-метрик.
Результаты данной работы будут наиболее привлекательны преимущественно для мелких компаний, частных инвесторов и акционеров.

% \begin{table}
% \centering
% \caption{Карта сегментирования рынка услуг по моделям оценивания портфельного риска}
% \label{tab:F:segments}
% \setlength{\tabcolsep}{6pt}
% \begin{adjustbox}{max width=\textwidth}
% \begin{tabularx}{\textwidth}{ c|c *{3}{|Y} }
%     \cline{3-5}
%     \multicolumn{2}{c|}{} & \multicolumn{3}{c}{Виды риск-метрик} \\ \cline{3-5}
%     \multicolumn{2}{c|}{} & VaR & CVaR & 
%     \parbox[c][1.6cm]{3.5cm}{\centering Коэффициент Шарпа}
%     \\ \hline
%     \multirow{3}{*}{\rotatebox{90}{\parbox{5cm}{\centering Консервативность}}}
%     & \parbox[c][2cm]{3.5cm}{\centering Консервативная} & & & \\ \cline{2-5}
%     & \parbox[c][2cm]{3.5cm}{\centering Точная} & & & \\ \cline{2-5}
%     & \parbox[c][2cm]{3.5cm}{\centering Агрессивная} & & & \\ \hline
% \end{tabularx}
% \end{adjustbox}
% \end{table}
% \nopagebreak

% \begin{table}
% \centering
% \begin{tabular}{*{6}{c}}
%     Исторический & \cellcolor{gray!50} & 
%     Монте-Карло & \cellcolor{gray!100} & 
%     Копула & \cellcolor{gray!150}
% \end{tabular}
% \end{table}

\begin{table}[hb]
\centering
\caption{Карта сегментирования рынка услуг по моделям оценивания портфельного риска}
\label{tab:F:segments}
\begin{tabularx}{\textwidth}{cc*{3}{|Y}}
    \toprule
    \multicolumn{2}{c}{} & \multicolumn{3}{c}{Виды риск-метрик} \\ \cmidrule{3-5}
    \multicolumn{2}{c}{} & VaR & CVaR & 
    \parbox[c][1.6cm]{3.5cm}{\centering Коэффициент Шарпа}
    \\ \hline
    \multirow{3}{*}{\rotatebox{90}{\parbox{5cm}{\centering Размер компании}}}
    & \parbox[c][2cm]{3.5cm}{\centering Крупные} 
    & \cellcolor{gray!33} 
    & \cellcolor{gray!33} 
    & \cellcolor{gray!66} \\ \cline{3-5}
    & \parbox[c][2cm]{3.5cm}{\centering Средние} 
    & \cellcolor{gray!33}
    & \cellcolor{gray!66}
    & \\ \cline{3-5}
    & \parbox[c][2cm]{3.5cm}{\centering Мелкие}
    & \cellcolor{gray!99} 
    & \cellcolor{gray!99} 
    & \cellcolor{gray!99} \\ \Xhline{1pt}
\end{tabularx} \\
\vspace{1ex}
\begin{tabularx}{0.8\textwidth}{*{3}{>{\hsize=0.3\hsize}X >{\hsize=1.7\hsize}X}}
    % \cline{1-1} \cline{3-3}
    \cellcolor{gray!33} & Исторический
    & \cellcolor{gray!66} & Монте-Карло
    & \cellcolor{gray!99} & Копула
    % \cline{1-1} \cline{3-3}
\end{tabularx}
\end{table}

\subsubsection{Анализ конкурентных технических решений с позиции ресурсоэффективности и ресурсосбережения}

Для оценки инвестиционного риска в широком смысле существует множество моделей, причём оцениваемые риск-метрики также могут быть самые разные.
Исходя из результатов сегментирования, к таким моделям можно отнести историческое моделирование и метод Монте-Карло.
Помимо них, существуют множество других подходов, использующих, как и данная методология, модели копула.
Добавим для сравнения работу A.~Lourme и F.~Maurer~\cite{Lourme2016}, поскольку идея авторов взята за основу для текущего подхода.

Наиболее значимые для данной работы критерии сравнения и оценки ресурсоэффективности и ресурсосбережения приведены в таблице~\ref{tab:F:competitors}.
Взятые для сравнения методы: \textit{h}~--- исторчиеский, \textit{MC}~--- Монте-Карло, \textit{L}~--- работа A.~Lourme и F.~Maurer~\cite{Lourme2016}, \textit{cur}~--- данная работа.
Итоговое значение конкурентоспособности модели $C$ получено по формуле
\begin{equation} C = \sum_i C_i = \sum_i w_i \cdot B_i, \end{equation}
\begin{where}
  %  \item $C$~--- конкурентоспособность модели, 
    \item $C_i$~--- конкурентоспособность модели по $i$-му критерию, 
    \item $w_i$~--- вес $i$-го критерия, 
    \item $B_i$~--- балл модели по $i$-му критерию.
\end{where}

\begin{table}[tb]
\caption{Оценочная карта для сравнения конкурентных технических решений}
\label{tab:F:competitors}
\centering
\tabcolsep=5pt
\begin{tabularx}{\textwidth}
    {>{\HY\hspace{0pt}\hsize=3.52\hsize}L*{9}{>{\hsize=0.72\hsize}Y}} 
    \toprule
    \multicolumn{1}{>{\hsize=3.52\hsize}Y}{\multirow{2}{*}{Критерий оценки}} 
        & \multicolumn{1}{>{\hsize=0.72\hsize}Y}{\multirow{2}{*}{Вес}} 
        & \multicolumn{4}{m{0.288\textwidth}}{\centering Баллы} 
        & \multicolumn{4}{m{0.288\textwidth}}{\HY\hspace{0pt}\centering Конкурентоспособность} 
        \\ \cmidrule{3-10}
    & & $B_h$ & $B_{MC}$ & $B_L$ & $B_{cur}$ & $C_h$ & $C_{MC}$ & $C_L$ & $C_{cur}$ \\ \midrule[1pt]
    \multicolumn{10}{c}{Технические критерии оценки ресурсоэффективности} \\ \midrule
    Консервативность оценки         & 0.25 & 3 & 4 & 4 & 5 & 0.75 & 1.00    & 1.00    & 1.25 \\ \midrule[0pt]
    Адекватность оценки             & 0.20  & 3 & 4 & 4 & 5 & 0.60  & 0.80  & 0.8  & 1.00    \\ \midrule[0pt]
    Простота методологии            & 0.15 & 5 & 4 & 4 & 3 & 0.75 & 0.60  & 0.60  & 0.45 \\ \midrule[0pt]
    Независимость от объёма данных  & 0.15 & 1 & 5 & 5 & 4 & 0.15 & 0.75 & 0.75 & 0.60  \\ \midrule[0pt]
    \midrule
    \multicolumn{10}{c}{Экономические критерии оценки эффективности} \\ \midrule
    Распространённость на рынке     & 0.10  & 5 & 4 & 2 & 1 & 0.50  & 0.40  & 0.20  & 0.10  \\ \midrule[0pt]
    Затраты                         & 0.15 & 5 & 3 & 3 & 5 & 0.75 & 0.45 & 0.45 & 0.75 \\ \midrule
    Итого                           & 1.00    &   &   &   &   & 3.50  & 4.00    & 3.80  & 4.15 \\ \bottomrule
\end{tabularx}
\end{table}

По результатам расчётов видно, что модель исторического моделирования является наиболее простой и потому более распространённой, однако качество результатов хуже по сравнению с методом Монте-Карло или работой A.~Lourme и F.~Maurer~\cite{Lourme2016}. 
В свою очередь оставшиеся две модели удовлетворяют желаемому качеству результатов в меньшей степени по сравнению с нашим методом.
Отсюда можно сделать вывод, что для выбранных критериев исследуемая модель является наиболее удовлетворительной.

% \subsubsection{FAST-анализ}

% FAST-анализ (методика системного анализа функций)~--- способ анализа и классификации функций. 
% Суть метода основывается на том, что затраты, связанные с созданием и использованием любого объекта, выполняющего заданные функции, состоят из необходимых для его изготовления и эксплуатации и дополнительных, функционально неоправданных, излишних затрат, которые возникают из-за введения ненужных функций, не имеющих прямого отношения к назначению объекта, или связаны с несовершенством конструкции, технологических процессов, применяемых материалов, методов организации труда и т.д.

% Рассмотрим стадии FAST-анализа для исследуемой модели.

% \textbf{Стадия 1.} Выбор объекта.
% \nopagebreak

% Объектом исследования является оценка портфельного риска с помощью копул.

% \textbf{Стадия 2.} Описание функций, выполняемых объектом.
% \nopagebreak

% Все функции высокого уровня, используемые в модели, обозначены в таблице~\ref{tab:F:functions}.

% \begin{table}[htb]
% \caption{Классификация функций, выполняемых объектом исследования}
% \label{tab:F:functions}
% \centering
% \begin{tabularx}{\textwidth}{L *{3}{c}} 
%     \toprule
%     \multicolumn{1}{c}{\multirow{2}{*}{Наименование функции}}
%         & \multicolumn{3}{c}{Ранг функции} 
%     \\ \cmidrule{2-4}
%     & Главная & Основная & Вспомогательная \\ \midrule
%     F1. Расчёт VaR и CVaR                       & \checkmark &            &            \\
%     F2. Определение корреляции активов портфеля &            & \checkmark &            \\
%     F3. Фитинг маргинальных распределений       &            &            & \checkmark \\
%     F4. Фитинг параметров копул                 &            &            & \checkmark \\
%     F5. Оптимизация портфеля                    &            &            & \checkmark \\
%     \bottomrule
% \end{tabularx}
% \end{table}

% \textbf{Стадия 3.} Определение значимости выполняемых объектом функций.
% \nopagebreak

% На первом этапе строим матрицу смежности функций (см. табл.~\ref{tab:F:adjMat}). 
% Для этого сравниваем функции по их значимости из предыдущего пункта.
% На втором этапе преобразуем полученную матрицу в матрицу количественных соотношений функций (табл.~\ref{tab:F:relationMat}). 
% Для этого в предыдущей матрицы знаки <<{<}>>, <<{=}>> и <<{>}>> меняем на числа 0.5, 1 и 1.5 соответственно. 
% Полученные значения суммируем и нормируем, тем самым определяя значимость каждой функции.

% \begin{table}[htb]
%     \caption{Матрица смежности}
%     \label{tab:F:adjMat}
%     \centering
%     \begin{tabularx}{0.5\textwidth}{Y|*{5}{Y}}
%         \toprule
%           & F1 & F2 & F3 & F4 & F5 \\ 
%         \midrule
%         F1 &  = &  > &  > &  > &  > \\
%         F2 &  < &  = &  > &  > &  > \\
%         F3 &  < &  < &  = &  = &  > \\
%         F4 &  < &  < &  = &  = &  > \\
%         F5 &  < &  < &  < &  < &  = \\ 
%         \bottomrule
%     \end{tabularx}
% \end{table}

% \begin{table}[htb]
%     \caption{Матрица количественных соотношений функций}
%     \label{tab:F:relationMat}
%     \centering
%     \begin{tabularx}{0.9\textwidth}
%     {Y|*{5}{>{\hsize=0.8\hsize}Y}|>{\hsize=1.2\hsize}Y|>{\hsize=1.8\hsize}Y}
%         \toprule
%           & F1 & F2 & F3 & F4 & F5 & Сумма & Значимость \\ 
%         \midrule
%         F1 &   1 & 1.5 & 1.5 & 1.5 & 1.5 & 7   & 0.304  \\
%         F2 & 0.5 &   1 & 1.5 & 1.5 & 1.5 & 6   & 0.261 \\
%         F3 & 0.5 & 0.5 &   1 &   1 & 0.5 & 3.5 & 0.152 \\
%         F4 & 0.5 & 0.5 &   1 &   1 & 0.5 & 3.5 & 0.152 \\
%         F5 & 0.5 & 0.5 & 0.5 & 0.5 &   1 & 3   & 0.131 \\
%         \midrule
%         Итого & & & & & & 23 & 1 \\
%         \bottomrule
%     \end{tabularx}
% \end{table}

\subsubsection{SWOT-анализ}

SWOT~--- Strengths (сильные стороны), Weaknesses (слабые стороны), Opportunities (возможности) и Threads (угрозы)~--- это комплексный анализ научно-исследовательского проекта, применяемый для исследования его внешней и внутренней среды.
Для проведения данного анализа необходимо выделить каждый из его компонентов применительно к исследуемой модели. 
Далее каждый компонент внешней среды комбинируется с каждым компонентом внутренней среды. 
Итоги отображены в игровой матрице SWOT-анализа (табл.~\ref{tab:F:swot}).

\begin{table}[tbh!]
\caption{SWOT-анализ}
\label{tab:F:swot}
\centering
\small
\renewcommand\tabularxcolumn[1]{p{#1}}
\begin{tabularx}{\textwidth}
{L|L|L} \toprule
    & \textbf{Сильные стороны:}
    \begin{enumerate}[wide=0pt,labelsep=4pt,after=\vspace{-\baselineskip}]
        \item Раздельное исследование маргиналов и зависимости активов.
        \item Слабая зависимость от объёма портфеля.
        \item Высокая адекватность и консервативность.
    \end{enumerate} & \textbf{Слабые стороны:}
    \begin{enumerate}[wide=0pt,labelsep=4pt,after=\vspace{-\baselineskip}]
        \item Сложность алгоритма.
        \item Длительность вычислений.
        \item Малая распространённость на рынке.
    \end{enumerate} \\
    \midrule
    \textbf{Возможности:}
    \begin{enumerate}[wide=0pt,labelsep=4pt,after=\vspace{-\baselineskip}]
        \item Высокий спрос на рисковые модели со стороны потребителей.
        \item Анализ портфелей со сложной взаимосвязью между активами.
        \item Анализ портфелей с большим числом активов.
    \end{enumerate} 
    & Высокое качество получаемых результатов вкупе с высоким спросом на рисковые модели применительно к портфелям с большим количеством активов позволят быстро занять свою нишу на рынке.
    & Благодаря высокому спросу можно упростить алгоритм, используя его отдельно для каждой конкретной задачи.
    \\ \midrule
    \textbf{Угрозы:}
    \begin{enumerate}[wide=0pt,labelsep=4pt,after=\vspace{-\baselineskip}]
        \item Наличие множества альтернативных подходов.
        \item Развитие рынка, усложнение исходных данных.
    \end{enumerate}
    & Универсальность модели и высокое качество получаемых результатов позволит эффективнее бороться с конкурентами.
    & Малая распространённость на рынке подобных моделей позволит эффективнее бороться с конкурентами.
    \\ \bottomrule
\end{tabularx}
\end{table}

\subsubsection{Оценка готовности проекта к коммерциализации}
\label{F:comm}

Для данной работы проведён анализ степени проработанности проекта с позиции коммерциализации.
Данный анализ приведён в таблице~\ref{tab:F:comm} с оценками степени готовности научного проекта в коммерческом отношении.
Каждый показатель анализа был оценён по пятибалльной шкале. 
Оценки степени проработанности научного проекта трактуются следующим образом:

\begin{enumerate}[label=\arabic* --]
    \item не проработано;
    \item проработано слабо;
    \item выполнено, но качество под сомнением;
    \item выполнено качественно;
    \item имеется положительное заключение независимого эксперта.
\end{enumerate}

Оценка уровня имеющихся знаний у разработчика определяется в соответствии со следующей системой баллов:

\begin{enumerate}[label=\arabic* --]
    \item не знаком или знаком мало;
    \item знаком с теорией;
    \item знаком с теорией и практическими примерами применения;
    \item знаком с теорией и самостоятельно выполняет;
    \item знаком с теорию, выполняет, может консультировать.
\end{enumerate}

Итоговая оценка Б определяется как сумма всех оценок $\text{Б}_i$ по каждому $i$-му показателю:
\begin{equation}
    \text{Б} = \sum_i \text{Б}_i.
\end{equation}

\begin{longtable}
{m{0.5\textwidth-2\tabcolsep} C{0.25\textwidth-2\tabcolsep} C{0.25\textwidth-2\tabcolsep}}
\caption{Оценка готовности научного проекта к коммерциализации}
\label{tab:F:comm}
\\
\toprule
\centering Наименование & Степень проработанности научного проекта & Уровень знаний у~разработчика \\ \midrule[1pt]
\centering 1 & 2 & 3 \\ 
\midrule
\endfirsthead
\\
\midrule
\centering 1 & 2 & 3 \\ 
\midrule
\endhead

\midrule
\endfoot
\endlastfoot

Определён имеющийся научно\-технический задел & 5 & 5 \\ \midrule[0pt]
Определены перспективные направления коммерциализации научно-технического задела & 5 & 5 \\ \midrule[0pt]
Определены отрасли и технологии (товары, услуги) для предложения на рынке & 5 & 5 \\ \midrule[0pt]
Определена товарная форма научно\-технического задела для представления на рынок & 3 & 2 \\ \midrule[0pt]
Определены авторы и осуществлена охрана их прав & 5 & 4 \\ \midrule[0pt]
Проведена оценка стоимости интеллектуальной собственности & 1 & 1 \\ \midrule[0pt]
Проведены маркетинговые исследования рынков сбыта & 3 & 2 \\ \midrule[0pt]
Разработан бизнес\-план коммерциализации научной разработки & 1 & 1 \\ \midrule[0pt]
Определены пути продвижения научной разработки на рынок & 1 & 1 \\ \midrule[0pt]
Разработана стратегия (форма) реализации научной разработки & 4 & 4 \\ \midrule[0pt]
Проработаны вопросы международного сотрудничества и выхода на зарубежный рынок & 1 & 1 \\ \midrule[0pt]
Проработаны вопросы использования услуг инфраструктуры поддержки, получения льгот & 1 & 1 \\ \midrule[0pt]
Проработаны вопросы финансирования коммерциализации научной разработки & 1 & 1 \\ \midrule[0pt]
Имеется команда для коммерциализации научной разработки & 1 & 1 \\ \midrule[0pt]
Проработан механизм реализации научного проекта & 1 & 1 \\
\midrule
Итого & 38 & 35 \\
\bottomrule
\end{longtable}

Полученные суммарные значения соответствуют средней перспективности проекта на коммерциализацию.
Поэтому целесообразно в качестве метода коммерциализации выбрать передачу интеллектуальной собственности третьему лицу на коммерческих условиях.

\subsection{Планирование управления научно-техническим проектом}
\label{F:plan}

\subsubsection{Цели и результат проекта}

Целями проекта с экономической точки зрения являются изучение существующих методов и их усовершенствование, а также последующее удовлетворение ожиданий заинтересованных сторон. 
Более подробная информация о заинтересованных сторонах отражена в табл.~\ref{tab:F:interest}
% , о целях, ожидаемых результатах и требованиях~--- в табл.~\ref{tab:F:aim}.

\begin{table}[htb]
\caption{Заинтересованные стороны проекта}
\label{tab:F:interest}
\centering
\tabcolsep=10pt
\begin{tabularx}{\textwidth}
{lL} \toprule
    Заинтересованные стороны & Ожидания заинтересованных сторон \\ \midrule
    Инвестиционные фонды 
    & \multirow{3}{*}{\parbox{\hsize}{\raggedright Высокое качество результатов, большие объёмы данных, быстродействие}}
    \\ %\cmidrule{1-1}
    Брокеры & \\ %\cmidrule{1-1}
    Частные инвесторы & \\ \bottomrule
\end{tabularx}
\end{table}

\textbf{Цель проекта:} оценка VaR и CVaR портфеля с использованием копул.

\textbf{Ожидаемые результаты проекта:}
\begin{itemize}
    \item оценить параметры маргинальных распределений активов и параметры копул;
    \item построить многомерное распределение для исследуемого портфеля;
    \item получить теоретические P\&L портфеля;
    \item рассчитать VaR и CVaR и сравнить со значениями, полученными для разных моделей.
\end{itemize}

\textbf{Критерии приёмки результата проекта:}
\begin{itemize}
    \item Характеристики модели оцениваются с помощью тестов. Критерием значимости было выбрано значение $\text{\emph{p}-value} = 0.05$.
    \item Время для расчётов должно быть меньше, чем у моделей, конкурирующих по качеству характеристик.
\end{itemize}

\textbf{Требования к результатам проекта}
\begin{itemize}
    \item консервативность и адекватность;
    \item возможность обработки больших портфелей;
    \item быстродействие.
\end{itemize}

\subsubsection{Организационная структура проекта}
Список участников и их функции отображены в табл.~\ref{tab:F:group}.

\begin{table}[tb]
\caption{Рабочая группа проекта}
\label{tab:F:group}
\centering
\renewcommand\tabularxcolumn[1]{p{#1}}
\begin{tabularx}
    {\textwidth}
    {L >{\HY\hspace{0pt}\hsize=0.6\hsize}L >{\HY\hspace{0pt}\hsize=1.8\hsize}L >{\HY\hspace{0pt}\hsize=0.6\hsize}L}
    \toprule
    ФИО, основное место работы, должность & Роль в проекте & Функции & Трудозатраты,~ч \\
    \midrule
    Семёнов Михаил Евгеньевич, ТПУ, доцент & Руководитель 
    & Составление и утверждение научного задания, календарное планирование  работ по теме, оценка эффективности полученных результатов & 594 \\ 
    \midrule[0pt]
    Смагулов Даулет Серикбаевич, ТПУ, инженер & Исполнитель 
    & Выполнение поставленной задачи, составление и оформление пояснительной записки к ВКР & 2016 \\
    \bottomrule
\end{tabularx}
\end{table}

\subsubsection{План проекта}

Календарный план проекта отображён в табл.~\ref{tab:F:plan}, а также в виде временной диаграммы Ганта в табл.~\ref{tab:F:gantt}.
В данной диаграмме этапы работы представлены в виде протяжённых временных отрезков. 

Итого общая занятость в проекте руководителя составляет 99 дней, студента~--- 336 дней.

\begin{table}[htb]
\caption{Календарный план проекта}
\label{tab:F:plan}
\centering
\renewcommand\tabularxcolumn[1]{p{#1}}
\setcounter{MyCounter}{1}
\begin{tabularx}
    {\textwidth}
    % {cXcccl}
    {>{\hsize=0.2\hsize}Y >{\raggedright\hsize=1.9\hsize}X >{\hsize=0.8\hsize}Y >{\hsize=0.9\hsize}X >{\hsize=0.9\hsize}X >{\hsize=1.3\hsize}L}
    \toprule
    № & Этап & \HY\hspace{0pt}\raggedright Длительность, раб. дни 
        & Дата \mbox{начала} & Дата окончания & Состав участников \\
    \midrule
    \myItem & Составление и утверждение научного задания 
        & 4 & 01.03.17 & 05.03.17 & Руководитель \\ \midrule[0pt]
    \myItem & Календарное планирование по теме
        & 8 & 06.03.17 & 15.03.17 & Руководитель \\ \midrule[0pt] 
    \myItem & Подбор и изучение материалов по теме 
        & 4 & 16.03.17 & 20.03.17 & Руководитель, исполнитель \\ \midrule[0pt] 
    \myItem & Выбор исходных данных
        & 5 & 21.03.17 & 26.03.17 & Руководитель \\ \midrule[0pt] 
    \myItem & Выбор направления исследований
        & 12 & 27.03.17 & 10.04.17 & Руководитель, исполнитель\\ \midrule[0pt] 
    \myItem & Разработка методологии 
        & 66 & 11.04.17 & 30.06.18 & Руководитель, исполнитель \\ \midrule[0pt]
    \myItem & Написание кода
        & 62 & 01.07.17 & 13.09.18 & Исполнитель \\ \midrule[0pt] 
    \myItem & Тестирование
        & 125 & 14.09.17 & 13.02.18 & Исполнитель \\ \midrule[0pt] 
    \myItem & Составление пояснительной записки к ВКР
        & 67 & 14.02.18 & 06.05.18 & Исполнитель \\ \midrule
    \multicolumn{1}{c}{} & \multicolumn{1}{l}{Итого} & \multicolumn{1}{c}{435} & \multicolumn{3}{c}{} \\ \bottomrule
\end{tabularx}
\end{table}

\begin{table}[htbp]
\caption{Диаграмма Ганта проведения проекта}
\label{tab:F:gantt}
\centering
\noindent
\begin{tabularx}{0.6\textwidth}{*{2}{|>{\hsize=0.5\hsize}X|>{\hsize=1.5\hsize}X}} 
    \cline{1-1} \cline{3-3}
    \cellcolor{gray!33} & Руководитель & & Исполнитель \\
    \cline{1-1} \cline{3-3}
\end{tabularx}
\medskip

\rotatebox{90}{
\begin{ganttchart}[
    vgrid={
        *{30}{draw=none},dotted,*{29}{draw=none},dotted,*{30}{draw=none},dotted,*{29}{draw=none},dotted,*{30}{draw=none},dotted,*{30}{draw=none},dotted,*{29}{draw=none},dotted,*{30}{draw=none},dotted,*{29}{draw=none},dotted,*{30}{draw=none},dotted,*{30}{draw=none},dotted,*{27}{draw=none},dotted,*{30}{draw=none},dotted,*{29}{draw=none},dotted,*{30}{draw=none},dotted,*{29}{draw=none},dotted},
    hgrid={*{3}{dotted},draw=none,*{2}{dotted},draw=none,dotted,draw=none,*{3}{dotted}},
    x unit=0.33mm,
    y unit chart=1.2cm,
    bar label node/.append style={text width=6cm},
    time slot format=isodate,
    ]{2017-03-01}{2018-06-30}
    \setcounter{MyCounter}{1}
    \linespread{1}
    
    \gantttitlecalendar{year, month} \\
    
    \supervisor[4 дня]{\myItem. Составление и утверждение научного задания}{2017-03-01}{2017-03-05} \\
    \supervisor[8 дней]{\myItem. Календарное планирование по теме}{2017-03-06}{2017-03-15} \\
    \supervisor[4 дня]{\myItem. Подбор и изучение материалов по теме}{2017-03-16}{2017-03-20} \\
    \student[4 дня]{}{2017-03-16}{2017-03-20} \\
    \supervisor[5 дней]{\myItem. Выбор исходных данных}{2017-03-21}{2017-03-26} \\
    \supervisor[12 дней]{\myItem. Выбор направления исследований}{2017-03-27}{2017-04-10} \\
    \student[12 дней]{}{2017-03-27}{2017-04-10} \\
    \supervisor[$66$ дней]{\myItem. Разработка методологии}{2017-04-11}{2017-06-30} \\
    \student[$66$ дней]{}{2017-04-11}{2017-06-30}  \\
    \student[62 дня]{\myItem. Написание кода}{2017-07-01}{2017-09-13} \\
    \student[125 дней]{\myItem. Тестирование}{2017-09-14}{2018-02-13} \\
    \student[67 дней]{\myItem. Составление пояснительной записки к ВКР}{2018-02-14}{2018-05-06}
\end{ganttchart}
}
\end{table}

\subsubsection{Бюджет научного исследования}

Бюджет данного исследования включает в себя основную и дополнительную заработную плату участников проекта, отчисления на социальные нужды и накладные расходы:
\begin{equation}\label{eq:F:budget}
    \text{С} = \text{С}_\text{зп} + \text{С}_\text{внеб} + \text{С}_\text{накл},
\end{equation}
\begin{where}
    \item $\text{С}_\text{зп} = \text{З}_\text{осн} + \text{З}_\text{доп}$~--- отчисления на заработную плату;
    \item $\text{З}_\text{осн}$ и $\text{З}_\text{доп}$~--- основная и дополнительная заработные платы соответственно; 
    \item $\text{С}_\text{внеб}$~--- отчисления на социальные нужды (во внебюджетные фонды);
    \item $\text{С}_\text{накл}$~--- накладные расходы.
\end{where}

\begin{table}[bt]
\centering
\caption{Баланс рабочего времени}
\label{tab:F:bTime}
\begin{tabularx}{\textwidth}
{>{\hsize=1.8\hsize}L >{\hsize=0.6\hsize}Y >{\hsize=0.6\hsize}Y}
    \toprule
    Показатели рабочего времени & Руководитель & Инженер \\
    \midrule
    Календарное число дней & 365 & 365 \\
    \midrule[0pt]
    Количество нерабочих дней & & \\
    -- выходные дни & 52 & 52 \\
    -- праздничные дни & 14 & 14 \\
    \midrule[0pt]
    Потери рабочего времени & & \\
    -- отпуск & 48 & 48 \\
    -- невыходы по болезни & -- & -- \\
    \midrule
    Действительный годовой фонд рабочего времени & 251 & 251 \\
    \bottomrule
\end{tabularx}
\end{table}

Основная заработная плата рассчитывается по следующей формуле:
\begin{equation}
    \text{З}_\text{осн} = \text{З}_\text{дн} \cdot T_\text{р},
\end{equation}
\begin{where}
    \item $T_\text{р}$ -- продолжительность выполненной работы в днях,
    \item $\text{З}_\text{дн}$ -- среднедневная заработная плата работника, рассчитываемая по следующей формуле:
\end{where}
\begin{equation}
    \text{З}_\text{дн} = \frac{\text{З}_\text{м} \cdot \text{М}}{F_\text{д}},
\end{equation}
\begin{where}
    \item $\text{З}_\text{м}$ -- месячный должностной оклад работника, руб.;
    \item М -- количество месяцев работы без отпуска в течение года, при 6\-дневной рабочей неделе $\text{М} = 10.4$ мес.;
    \item $F_\text{д}$\,-- действительный годовой фонд рабочего времени научно-технического персонала в рабочих днях. 
    Последнее значение определяется из табл.~\ref{tab:F:bTime}. 
\end{where}

Месячный должностной оклад работника определяется как
\begin{equation}
    \text{З}_\text{м} = \text{З}_\text{б} \cdot k_\text{р},
\end{equation}
\begin{where}
    \item $\text{З}_\text{б}$ -- базовый оклад работника, руб., 
    \item $k_\text{р}$ -- районный коэффициент, равный для Томска 1.3. 
    Расчёт основной заработной платы приведён в табл.~\ref{tab:F:salary}.
\end{where}

\begin{table}
\centering
\caption{Расчёт основной заработной платы}
\label{tab:F:salary}
\begin{tabularx}{\textwidth}
{>{\hsize=1.4\hsize}L >{\hsize=0.9\hsize}R >{\hsize=0.6\hsize}Y R R >{\hsize=0.8\hsize}R >{\hsize=1.2\hsize}R}
    \toprule
    Исполнители & $\text{З}_\text{б}$, руб. & $k_\text{р}$ & $\text{З}_\text{м}$, руб. & $\text{З}_\text{дн}$, руб. & $T_\text{р}$, руб. & $\text{З}_\text{осн}$, руб. \\
    \midrule
    Руководитель & 33\,664 & 1.3 & 43\,763.20 & 1\,813.30 &  99 & 179\,516.30 \\
    Инженер      &  9\,893 & 1.3 & 12\,860.90 &    532.88 & 336 & 179\,048.30 \\
    \bottomrule
\end{tabularx}
\end{table}

\begin{table}[htb]
\centering
\caption{Заработная плата на этапах исследования}
\label{tab:F:stepSalary}
\setcounter{MyCounter}{1}
\renewcommand\tabularxcolumn[1]{p{#1}}
\begin{tabularx}
{\textwidth}
{>{\hsize=0.2\hsize}R >{\raggedright\hsize=1.9\hsize}X >{\hsize=1.1\hsize}Y >{\hsize=0.9\hsize}R >{\hsize=0.9\hsize}R >{\hsize=\hsize}R}
    \toprule
    № & Этап & \HY\hspace{0pt}\raggedright Исполнители по категориям
        & \HY\hspace{0pt}Трудоёмкость, чел.\cdot дн. & Удельная з/п, руб. & Общая з/п по тарифу, руб. \\
    \midrule
    \myItem & Составление и утверждение научного задания 
        & Руководитель &   4 & 1\,813.30 &   7\,253.18 \\ \midrule[0pt]
    \myItem & Календарное планирование по теме
        & Руководитель &   8 & 1\,813.30 &  14\,506.37 \\ \midrule[0pt]
    \myItem & Подбор и изучение
        & Руководитель &   4 & 1\,813.30 &   7\,253.18 \\ 
            & материалов по теме
        & Исполнитель  &   4 &    532.88 &   2\,131.53 \\ \midrule[0pt]
    \myItem & Выбор исходных данных
        & Руководитель &   5 & 1\,813.30 &   9\,066.48 \\ \midrule[0pt]
    \myItem & Выбор направления
        & Руководитель &  12 & 1\,813.30 &  21\,759.55 \\
            & исследований
        & Исполнитель  &  12 &    532.88 &   6\,394.58 \\ \midrule[0pt]
    \myItem & Разработка
        & Руководитель &  66 & 1\,813.30 & 119\,677.54 \\
            & методологии 
        & Исполнитель  &  66 &    532.88 &  35\,170.21 \\ \midrule[0pt]
    \myItem & Написание кода
        & Исполнитель  &  62 &    532.88 &  33\,038.68 \\ \midrule[0pt]
    \myItem & Тестирование
        & Исполнитель  & 125 &    532.88 &  66\,610.24 \\ \midrule[0pt]
    \myItem & Составление пояснительной записки к ВКР
        & Исполнитель & 67 & 532.88 & 35\,703.09 \\ \midrule
    \multicolumn{1}{c}{} & \multicolumn{1}{l}{Итого} & \multicolumn{3}{l}{} & \multicolumn{1}{c}{358\,564.60} \\
    \bottomrule
\end{tabularx}
\end{table}

% Величина расходов по заработной плате определяется исходя из трудоёмкости выполняемых работ и действующей системы оплаты труда.

Значения $\text{З}_\text{дн}$ используем в табл.~\ref{tab:F:stepSalary} для расчёта заработной платы участников исследования за выполнение каждого этапа.

Дополнительную заработную плату примем равной $12\%$ от основной:
\begin{equation}
    \text{З}_\text{доп} = 0.12 \cdot \text{З}_\text{осн}.
\end{equation}

Так как работа производилась только с использованием персонального компьютера, все накладные расходы составляет плата за электроэнергию и интернет.
В расчётах будем учитывать, что мощность компьютера руководителя равна $P_\text{рук} = 0.1$ кВт, мощность компьютера исполнителя~--- $P_\text{исп} = 0.05$ кВт.
Также учитываем одинаковую плату за интернет $S_\text{и} = 350$ руб/мес.
Тогда при 6-часовом рабочем дне накладные расходы составляют
\begin{equation}\label{eq:F:energy}
    \text{С}_\text{накл} = 6 \cdot \left( T_\text{рук} \cdot P_\text{рук} + T_\text{исп} \cdot P_\text{исп} \right) \cdot S_\text{эл} + \frac{T_\text{р}}{30} \cdot S_\text{и},
\end{equation}
\begin{where}
    \item $S_\text{эл} = 5.8$ руб\,/\,кВт\,\cdot\,ч~--- удельная плата за электроэнергию.
\end{where}

Также необходимо учесть отчисления на социальные нужды:
\begin{equation}\label{eq:F:soc}
    \text{С}_\text{внеб} = k_\text{внеб} \cdot \text{С}_\text{зп},
\end{equation}
\begin{where}
    \item $k_\text{внеб} = 0.271$ -- коэффициент отчислений на уплату во внебюджетные фонды (пенсионный фонд, фонд обязательного медицинского страхования и~пр.).
\end{where}

\begin{table}
\caption{Итоговые значения расчёта бюджета исследования}
\label{tab:F:budget}
\centering
\begin{tabularx}{\textwidth}
{>{\hsize=0.6\hsize}L >{\hsize=1.8\hsize}L >{\hsize=0.6\hsize}R}
    \toprule
    Обозначение & Описание & Значение, руб. \\
    \midrule
    $\text{З}_\text{осн}$ & Основная заработная плата & 358\,564,60 \\
    $\text{З}_\text{доп}$ & Дополнительная заработная плата & 43\,027,75 \\
    $\text{С}_\text{накл}$ & Накладные расходы & 6\,004,16 \\
    $\text{С}_\text{внеб}$ & Отчисления на социальные нужды & 108\,831,50 \\
    \midrule
    $\text{С}$ & Бюджет исследования & 516\,428,03 \\
    \bottomrule
\end{tabularx}
\end{table}

Теперь, используя формулы (\ref{eq:F:budget}), (\ref{eq:F:soc}) и (\ref{eq:F:energy}), а также результаты, приведённые в табл.~\ref{tab:F:salary} и \ref{tab:F:stepSalary}, рассчитываем бюджет данного исследования. 
Итоговые значения отображены в таблице~\ref{tab:F:budget}.
Таким образом, бюджет исследования составляет 516\,428,03 руб.

\subsubsection{Реестр рисков проекта}

Во время проекта существует риск возникновения неопределённых событий, которые могут повлечь за собой нежелательные эффекты. 
Для таких событий составлен реестр рисков, содержащий в себе общую информацию о них и отображённый в таблице~\ref{tab:F:risk}.
Вероятность наступления и влияние определённого риска оцениваются по пятибалльной шкале. 
Уровень риска может быть высокий, средний или низкий в зависимости от вероятности наступления и степени влияния риска.

\begin{table}[tbh]
\caption{Реестр рисков}
\label{tab:F:risk}
\centering
\small
\renewcommand\tabularxcolumn[1]{p{#1}}
\begin{tabularx}{\textwidth}
{>{\HY\hspace{0pt}\hsize=0.9\hsize}L
 >{\HY\hspace{0pt}\hsize=1.2\hsize}L
 >{\HY\hspace{0pt}\hsize=1.1\hsize}Y
 >{\HY\hspace{0pt}\hsize=0.7\hsize}Y
 >{\HY\hspace{0pt}\hsize=0.8\hsize}Y
 >{\HY\hspace{0pt}\hsize=1.2\hsize}X
 >{\HY\hspace{0pt}\hsize=1.1\hsize}L}
    \toprule
    Риск & Потенциальное воздействие & Вероятность наступления & Влияние & Уровень & Способы смягчения & Условия наступления \\
    \midrule
    Управление проектом & Некорректный сбор информации & 3 & 5 & Высокий & Распределение обязанностей & Несогласованность действий \\
    \midrule[0pt]
    Технический & Некорректные результаты & 3 & 5 & Высокий & Чёткое планирование & Несогласованность действий \\
    \midrule[0pt]
    Внешний & Несоответствие плану & 2 & 2 & Низкий & Резервное время & Отсутствие данных \\
    \bottomrule
\end{tabularx}
\end{table}