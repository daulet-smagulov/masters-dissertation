% Моя преамбула

% Математические формулы
\usepackage{amsmath,amssymb,amsfonts,pxfonts}
\usepackage{unicode-math}
\usepackage{icomma}

% Переносы слов
\usepackage{polyglossia}
\setmainlanguage{russian}
\setotherlanguage{english}
\sloppy
\PolyglossiaSetup{russian}{indentfirst=true}
\renewcommand{\-}{\nobreak-\hspace{0pt}}

% Форматирование текста и математических формул
\setmainfont{XITS}
\setmathfont{XITS Math}
\usepackage[Symbol]{upgreek}

% Оглавление и литература
\newcommand{\myRefName}{Список использованных источников}
\addto\captionsrussian{\renewcommand{\contentsname}{Оглавление}}
% \addto\captionsrussian{\renewcommand{\refname}{\myRefName}}
\usepackage[nottoc,notlot,notlof]{tocbibind}
\usepackage{csquotes}
\usepackage[style=numeric,giveninits=true,sorting=nyt]{biblatex}
\bibliography{references.bib}
% \bibliographystyle{abbrv}
% \usepackage[
%     backend=biber, 
%     sorting=nyt,
%     bibstyle=gost-authoryear,
%     citestyle=gost-authoryear
% ]{biblatex}

% Абзац 15 мм
\parindent=15mm

% Жирный курсив
\newcommand{\boldit}[1]{\textbf{\textit{#1}}}

% Секции по центру
\usepackage{titlesec}
\titleformat{\section}{\centering\normalfont\bfseries}{\thesection}{0.2cm}{}
%\titleformat*{\section}{\centering\normalfont\large\bfseries}
% \titleformat*{\subsection}{\normalfont\normalsize\bfseries}
\titleformat{\subsection}[block]{\hspace{\parindent}\bfseries}{\thesubsection}{0.2cm}{}
\titleformat{\subsubsection}[block]{\hspace{\parindent}\bfseries}{\thesubsubsection}{0.2cm}{}
% Привязать заголовки к следующему абзацу
\usepackage{needspace}
% Секции без нумерации
\newcommand{\anonsection}[1]{\section*{#1}\addcontentsline{toc}{section}{#1}}
\newcommand{\anonsubsection}[1]{\subsection*{#1}\addcontentsline{toc}{subsection}{#1}}
% Нумерация картинок и таблиц
\usepackage{chngcntr}
\counterwithin{figure}{section}
\counterwithin{table}{section}
\counterwithin{equation}{section}
% Новая страница для секций
\newcommand{\sectionbreak}{\clearpage}

% Картинки
\usepackage[labelsep=endash]{caption}
\usepackage[final]{graphicx}
\graphicspath{{images/}} % images folder
% Вставить картинку
\newcommand{\imgh}[5][h]
{
    \begin{figure}[#1]
    \centering
    \includegraphics[#3]{#2}
    \caption{#4}
    \label{ris:#5}
    \end{figure}
}
% example: \imgh{1.pdf}{width=0.93\textwidth}{Image}{img}
\RequirePackage{caption}
% \DeclareCaptionLabelSeparator{dot}{. }
\captionsetup[figure]{justification=centering,labelsep=endash,name={Рисунок}}
\captionsetup[table]{justification=raggedright,singlelinecheck=false,labelsep=endash}

% Поля страницы
\usepackage[left=3cm,right=1cm,top=2cm,bottom=2cm]{geometry}

% Полуторный интервал
% \linespread{1.3}
% \renewcommand\baselinestretch{1.33}
\usepackage{setspace}
\newcommand\spread{1.3}
\spacing{\spread}
% Сохранить интервал в таблицах
\usepackage{etoolbox}

% Нумерация страниц
\usepackage{fancyhdr}
\pagestyle{fancy}
\fancyhead{} % No page header
\fancyfoot[L]{} % Empty
\fancyfoot[C]{} % Empty
\fancyfoot[R]{\thepage} % Remove headlines
% Remove header underlines
\renewcommand{\headrulewidth}{0pt}

% Таблицы
\usepackage{multirow, makecell}
\usepackage{tabularx}
% Горизонтальные линии
\usepackage{booktabs}
% Вертикальное выравнивание - по центру
\renewcommand\tabularxcolumn[1]{m{#1}}
\newcolumntype{Y}{>{\centering\arraybackslash}X}
\newcolumntype{R}{>{\raggedleft\arraybackslash}X}
\newcolumntype{L}{>{\raggedright\arraybackslash}X}
\newcolumntype{C}[1]{>{\HY\hspace{0pt}\centering\arraybackslash}m{#1}}
% hyphenation in a cell
\newcommand{\HY}{\hyphenpenalty=25\exhyphenpenalty=25}
% \newcolumntype{Z}{>{\HY\centering}X}
% re-enable hyphenation locally inside "Z" columns
% \renewcommand{\tabularxcolumn}[1]{>{\normalsize}b{#1}}
% Подгон больших таблиц под размер страницы
\usepackage{adjustbox}
\usepackage{bigstrut}
\usepackage{color, colortbl}
\usepackage{xcolor}
% Ячейка с диагональной линией
\usepackage{diagbox}
\setlength{\tabcolsep}{4pt}
% Межстрочный интервал в таблицах
% \AtBeginEnvironment{table}{\setstretch{1.3}}
% \AtBeginEnvironment{tabularx}{\setstretch{1.3}}
% \AtBeginEnvironment{tabular}{\setstretch{1.3}}
% Длинные таблицы
\usepackage{longtable}

% Аннотация
\renewcommand{\abstractname}{Аннотация}

% Пакет для алгоритмов
\usepackage{algorithm}
\usepackage{algpseudocode}
\renewcommand{\algorithmicrequire}{\textbf{Вход:}}
\renewcommand{\algorithmicensure}{\textbf{Выход:}}
\floatname{algorithm}{Алгоритм}
\AtBeginEnvironment{algorithmic}{\setstretch{1.3}}

% Списки
\usepackage{enumitem}
\setlist{
    topsep=4pt,
    itemsep=0ex,
    partopsep=1ex,
    parsep=0ex,
    labelwidth=.5\parindent,
    labelsep=0pt,
    leftmargin=\parindent+\labelsep+\labelwidth,
    align=left
}
\newbool{firstitem}
\newenvironment{where}
{\begin{itemize}[
    label=\ifbool{firstitem}{{где}\global\boolfalse{firstitem}}{},
    before=\booltrue{firstitem},
    leftmargin=0pt,
    labelwidth=0.85cm,
    itemindent=\labelwidth]}
{\end{itemize}}
\newenvironment{where*}
{\begin{itemize}[
    label=\ifbool{firstitem}{{where}\global\boolfalse{firstitem}}{},
    before=\booltrue{firstitem},
    leftmargin=0pt,
    labelwidth=1.5cm,
    itemindent=\labelwidth]}
{\end{itemize}}
% \setlist[where]{label=\ifbool{firstitem}{\global\boolfalse{firstitem}}{},before=\booltrue{firstitem}}

% Многострочный комментарий
\usepackage{comment}

% Устранить висячие строки
\clubpenalty=10000
\widowpenalty=10000

% Символы, в том числе "галочка"
\usepackage{pifont}

% Диаграмма Гантта 
\usepackage{pgfgantt}
\newcommand{\supervisor}[4][]{\ganttbar[progress=0,progress label text=#1]{#2}{#3}{#4}}
\newcommand{\student}[4][]{\ganttbar[progress=100,progress label text=#1]{#2}{#3}{#4}}

% Перечисление
\usepackage{calc,array}
\newcounter{MyCounter}
\newcommand{\myItem}{\theMyCounter\stepcounter{MyCounter}}
\newcounter{Appendix}
% \setcounter{Appendix}{1}
\renewcommand{\theAppendix}{\Asbuk{Appendix}}
\newcommand{\Appendix}{\refstepcounter{Appendix}\anonsection{Приложение \Asbuk{Appendix}}}

% Поворот страницы
\usepackage{lscape}