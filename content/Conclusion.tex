\anonsection{Заключение}

В данной работе мы использовали математический аппарат копул-функций для исследования многомерной зависимости между финансовыми временными рядами и сравнения риск-метрик VaR и CVaR для управления портфелем.
В качестве исходных данных мы использовали четыре временных ряда из дневных цен закрытия фьючерсов на индекс РТС и акции ПАО <<НорНикеля>>, ПАО <<Сбербанка>> и ПАО <<Газпрома>>.
В работе использовались наблюдения за два года: с 16 декабря 2015 по 16 декабря 2017.

Для оценки параметров копулярных моделей был выбран двухэтапный параметрический подход оценки максимального правдоподобия.
Для каждого временного ряда была выполненена оценка параметров для распределений-кандидатов: гиперболического, устойчивого Парето и Мейкснера.
Стастическое тестирование качества полученных оценок распределений показало удовлетворительные результаты оценки качества, полученных значений параметоров для всех распределений.
На основе результатов теста Крамера\,--\,фон~Мизеса для всех временных рядов было выбрано четырехпараметрическое распределение Мейкснера в качестве частного (маргинального).

Была поставлена и решена задача поиска оптимального портфеля  для указанных активов, затем осуществлена оценка параметровдля трех копулярных моделей: Гауссовой, $t$\,--\,Стьюдента и R-vine.
Выбор из семейств вайн копулы выполнялся на основе тестов Вуонга и Кларка.
Оценка параметров мкопулярных моделей была произведена методом <<Инверсии $\tau$ Кендалла>>.
Статистическое тестирование качества оценки параметров копул показало, что модель $t$-копулы и R-vine копулы оказались наименее и наиболее адеватными соответственно.

Был предложен алгоритм для расчёта риск-метрик с использованием копулярных моделей, основанный на Монте-Карло моделировании.
Этот алгоритм был также усовершенстован за счёт проведения бутстрап-процедуры, благодаря которой были получены несмещенные оценки риск-метрик, а также доверительный интервал для них.
В результате было выяснено, что исследуемые модели перестают быть консервативными на уровне больше 99\%.

Как показали результаты моделирования для кривых VaR и CVaR на уровне 95\% -- все используемые копулярные модели являются консервативными, причём более точно оценивает риск вайн копула, в то время как $t$-копула его переоценивает.
%Данные результаты были ожидаемы исходя из результатов тестов адекватности моделей копула.