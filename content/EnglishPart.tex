\section*{Literature review}

There are many different approaches that are actively used in applications to represent multivariate dependencies, for instance, principal component analysis, Bayesian networks, fuzzy techniques, factor analysis, and joint distribution function~\cite{Huynh2014, Kole2007}. 

In 1959, A.~Sklar~\cite{Sklar1959} first proved the theorem that a collection of marginal distributions can be coupled together via a \textit{copula} to form a multivariate distribution.
% The copula contains all the information about the dependence structure of the involved variables.
In the paper~\cite{Penikas2010}, the author introduced copula models concepts and its application to the different financial issues including the task of risk measurement.

Closed-form expressions to calculate the sensitivity of the risk measure, \textit{CVaR}, were proposed in the paper by Stoyanov et al.~\cite{Stoyanov2013}.

Studies of price risks in framework of portfolio management~\cite{Ane2003, Kole2007, Lourme2016, Xu2008} in many respects are similar to each other and differ only in the used data and insignificant variations in the copula models estimation. 
Among the studies, we single out the paper by An\'e et al.~\cite{Ane2003} that was one of the first where authors selected the dependence structure of international stock index returns through the Clayton copula. 
Lourme~et~al.~\cite{Lourme2016} address the issue of testing the full Gaussian and Student's~$t$ copulas in a risk management framework.
They proposed the $d$-dimensional compact Gaussian and Student's~$t$ confidence area inside of which a random vector with uniform margins on $(0, 1)$ falls with probability~$\alpha$. 
The results evidence that the Student's~$t$ copula \textit{VaR} model is an attractive alternative to the Gaussian one.
A portfolio of stocks, bonds and real estate was considered by Kole~et~al.~\cite{Kole2007} to determine the importance of selecting the right copula for risk management. 
The Gaussian, the Student’s~$t$ and the Gumbel copulas have been used to model the dependence of the daily returns on indices that approximate these three asset classes were tested. 
Then according to Value-at-Risk computations it was established that the Gaussian copula is too optimistic on the diversification benefits of the assets, while the Gumbel copula is too pessimistic.

Estimation of the unknown copula and marginal parameters is an important problem.
Nowadays many algorithms for constructing and fitting copulas have been designed.
For copula model estimation, there exist three methods: the full parametric method~\cite{Patton2006}, the semi-parametric method~\cite{Chen2006, Lourme2016}, and the non-parametric method~\cite{Fermanian2003, Kim2007}.
The full parametric method is implemented via two-stage maximum likelihood estimation (MLE) proposed in the monographs~\cite{Joe1997, Joe2014}.
In the dissertation research~\cite{Xu2008}, the two-stage  MLE method was applied, while the author uses all possible combinations of different marginal distributions (Gaussian, the Student's $t$, and skewed $t$ distribution) and different archimedean copulas in the estimation and testing process. 
The decision of choosing the marginal distribution is taken after the second step of MLE method.
For this purpose, a modification of the superior predictive ability of the Hansen test was proposed~\cite{Hansen2005}; it allows one to identify a copula that has superior forecasting ability.

Multivariate copulas based on the one distribution (for instance, normal or Student's~$t$) or on one the generator function lack the flexibility of accurately modelling the dependence among larger numbers of variables~\cite{Brechmann2013}. 
These lacks predetermined the direction of further research, as a result of which the regular vine copulas' (R-vine) concept was proposed by Joe~\cite{Joe1996} and developed in more detail in papers~\cite{Brechmann2013,Cooke2015}.
There are a lot of methods to work with vine copulas~\cite{Cooke2015,Czado2010,Dissmann2013,Nikoloulopoulos2012}. Nikoloulopoulos~et~al.~\cite{Nikoloulopoulos2012} applied the vine copulas with asymmetric tail dependence for financial return data.
A novel algorithms for evaluating a \textit{regular vine copula} parameters and simulating
from specified R-vines were proposed by Di{\ss}mann~et~al.~\cite{Dissmann2013}. 
% The selection of the R-vine tree structure based on a maximum spanning tree algorithm (MST), where edge weights are chosen appropriately to reflect large dependencies.
The use of vine copula is proposed for measuring systemic risk in the paper~\cite{Pourkhanali2016}. 
The authors developed a metric that captures crucial features of the dependence relationship: tail-dependence and correlation asymmetry.


\section*{Methodology}

\subsection*{Initial data}

First, initial time series should be converted to logarithmic returns.
In this way, we can obtain the data set we can further use in marginal distribution parameters estimation.
Eq.~(\ref{eng:log-returns}) transforms a price series $p$ into a log-returns $r$ series for each asset:

\begin{equation}\label{eng:log-returns}
r_{t,i}=\log \frac{p_{t,i}}{p_{t-1,i}},
\end{equation}
\begin{where*}
    \item $i \in \overline{1, d}$, $d$ is the number of assets;
    \item $t\in \overline{1, T}$ is a time point, in our case $T=253$;
\end{where*}

Since financial time series have a nonlinear dependence which is not covered by usual Peasrson's correlation coefficient, we use rank correlation coefficients: Kendall's $\tau$ and Spearman's $\rho$. 
Following to \cite{Dissmann2013}, in further calculations, we used Kendall's~$\tau$.

Let $X$ and $Y$ be two random variables defined in some probability space.
Then Spearman's rank correlation $\rho$ coefficient is defined as follows~\cite{Mye2003}
\begin{equation}\label{eng:spearman}
\rho = r(\text{rg}_X, \text{rg}_Y) = \frac{\text{cov}(\text{rg}_{X},\text{rg}_{Y})}{\sigma_{rg_X} \sigma_{rg_Y}},
\end{equation}
\begin{where*}
    \item $r(x,y)$~--- Pearson's correlation;
    \item $\text{rg}_X$ and $\text{rg}_Y$~--- random variables' rank;
    \item $\text{cov} (\text{rg}_{X}, \text{rg}_{Y})$~--- covariance of random variables' rank;
    \item $\sigma$~--- standard deviation.
\end{where*}

For two independent pairs $(X_a, X_b)$ and $(Y_a, Y_b)$ of the same random variables one can calculate Kendall's $\tau$~\cite{Kendall1970}:
\begin{eqnarray}\label{eng:kendall}
\tau &=& P\left[(X_a-X_b)(Y_a-Y_b)>0\right]-P\left[(X_a-X_b)(Y_a-Y_b)<0\right] = \\ 
&=& r\left(\text{sgn}(X_a-X_b),\text{sgn}(Y_a-Y_b)\right),\nonumber
\end{eqnarray}
\begin{where*}
    \item $r(x,y)$~--- Pearson's correlation;
    \item $sgn(\cdot)$~--- real number's sign.
\end{where*}

In contrast with Pearson's correlation, both of these rank coefficients are less sensitive to strong outliers which are the features of financial time series.
We will base our choice on article~\cite{Dissmann2013} and use Kendall's~$\tau$ in further calculations.

In the research the portfolio with the following assets is used:
\begin{enumerate}[label=\arabic*)]
    \item RTS index futures;
    \item Sberbank PJSC stocks futures;
    \item Gazprom PJSC stocks futures;
    \item MMC Norilsk Nickel PJSC stocks futures.
\end{enumerate}

Our sample has daily close values and covers the period of two years since December 16, 2015 to December 16, 2017 (504 observations). 
Denote them as RTS, SBRF, GAZP and GMKR respectively.
All that data regarding the futures prices were collected from the Finam Holdings service.

The obtained time series main characteristics are shown on Table~\ref{eng:tab:eng:assets}.

\begin{table}[htb]
    \centering
    \caption{Main characteristics of log-returns}
    \label{eng:tab:eng:assets}
    \begin{adjustbox}{max width=\textwidth}
    \begin{tabular}{l|cc|rrrr|rrrr}
        \toprule
        \multirow{2}{*}{Assets} & \multicolumn{2}{c|}{Moments} & \multicolumn{4}{c|}{Spearman's $\rho$} & \multicolumn{4}{c}{Kendall's $\tau$} \\
        & $\mu$ & $\sigma$ & \multicolumn{1}{c}{RTS} & \multicolumn{1}{c}{SBRF} & \multicolumn{1}{c}{GAZR} & \multicolumn{1}{c|}{GMKR} & \multicolumn{1}{c}{RTS} & \multicolumn{1}{c}{SBRF} & \multicolumn{1}{c}{GAZR} & \multicolumn{1}{c}{GMKR} \\ 
        \midrule
RTS  &  0.00075 & 0.016 &     1 & 0.702 & 0.621 & 0.323 &     1 & 0.515 & 0.444 & 0.218 \\
SBRF &  0.00154 & 0.016 & 0.702 &     1 & 0.519 & 0.308 & 0.515 &     1 & 0.364 & 0.208 \\
GAZR & -0.00001 & 0.013 & 0.621 & 0.519 &     1 & 0.375 & 0.444 & 0.364 &     1 & 0.256 \\
GMKR &  0.00033 & 0.015 & 0.323 & 0.308 & 0.375 &     1 & 0.218 & 0.208 & 0.256 &     1 \\    
        \bottomrule
    \end{tabular}
    \end{adjustbox}
\end{table}


\subsection*{Marginal distributions parameters estimation}

Many ways to describe financial data using Gaussian (normal) distribution exist today~\cite{Json1949}. 
On the other hand, a lot of empirical studies have shown that Gaussian distribution has a lot of problems with description of financial data, for instance see~\cite{Limp2011, Rachev2005, Wilmott2007}. 
Various non-normal distributions have been proposed for modelling extreme events, we choose the Hyperbolic~\cite{Barndoff1983}, Stable~\cite{Nolan2009, Rachev2005, Stoyanov2013} and Meixner~\cite{Schoutens2002} distributions as the three possible forms of marginal distributions.
These types of distribution are able to handle financial data features, such as heavy tails and skewness~\cite{Stoyanov2013}. 

A hyperbolic distribution $H(\pi, \zeta, \delta, \mu)$ is four parameter distribution \cite{Barndoff1983} determined by $\pi$ as a steepness parameter, $\zeta$ as a symmetry one, $\mu$ as a location one, and $\delta$ as a scale one.
The distribution is symmetrical about
$\mu$ if $\zeta=0$.
Eq.~(\ref{eng:hpdf}) describes the probability density function (PDF) of the univariate hyperbolic distribution:
\begin{eqnarray}\label{eng:hpdf} 
f_H(x|\pi,\zeta,\delta,\mu)=\frac{1}{2 \sqrt{1+\pi^2} 
K_1(\zeta) }e^{-\zeta \left[ \sqrt{1+\pi^2}
\sqrt{1+\big(\frac{x-\mu}{\delta})^2}-
\pi\frac{x-\mu}{\delta}\right]},
\end{eqnarray}
\begin{where*}
    \item $K_1$ is the 1st order modified Bessel function of the third kind~\cite{Bessel1824};
    \item $\pi \in \mathbb{R}$, $\zeta > 0$, $\delta > 0$, $\mu \in \mathbb{R}$.
\end{where*}

A stable distribution $S(\alpha, \beta, \gamma, \mu)$ is also described by four parameters~\cite{Rachev2005, Nolan2009, Stoyanov2013}.
The parameter $\alpha$ determines the tail weight and the kurtosis, $\beta$ determines the skewness, $\gamma$ is a scale parameter, and $\mu$ is a location parameter.
Since the probability density function and cumulative distribution function (c.d.f.) of the stable distribution does not exist in a closed-form, we use its characteristic function:
\begin{eqnarray}\label{eng:StChF} 
\varphi_S(x|\alpha,\beta,\gamma,\delta) &=& \exp{\left[ix\delta-|\gamma x|^\alpha \left(1 - i\beta\text{sgn}(x)\Upphi(x)\right)\right]}, \\
\Upphi(x) &=& \left\{ \begin{aligned}
    & \left(|\gamma x|^{1-\alpha} - 1\right)\tan{\frac{\pi\alpha}{2}}, & & \alpha \ne 1, \\
    & -\frac{2}{\pi}\log{|\gamma x|}, & & \alpha = 1,
\end{aligned} \right. \nonumber
\end{eqnarray}
\begin{where*}
    \item $\alpha \in (0;\ 2]$, $\beta \in [-1;\ 1]$, $\gamma > 0$, $\delta \in \mathbb{R}$;
    \item $i$ is the imaginary unit.
\end{where*}

A Meixner distribution $M(\alpha, \beta, \delta, \mu)$ has four parameters: $\mu$ is the location parameter, $\alpha$ is the scale parameter, $\beta$ is the skewness parameter, and $\delta$ is the shape parameter \cite{Schoutens2002}.
Eq.~(\ref{eng:mpdf}) describes the density of the Meixner distribution:%
\begin{equation} \label{eng:mpdf}
    f_M(x|\alpha,\beta,\delta,\mu)=\frac{\left(2\cos{\frac{\beta}{2}}\right)^{2\delta}}{2\alpha\pi\Upgamma(2\delta)}\exp{\frac{\beta(x-\mu)}{\alpha}}\left|\Upgamma\left(\delta + i\frac{x-\mu}{\alpha}\right)\right|^2,
\end{equation}
\begin{where*}
    \item $\Upgamma(z)$ is the gamma function for complex arguments;
    \item $\alpha > 0$, $|\beta| < \pi$, $\delta > 0$, $\mu \in \mathbb{R}$.
\end{where*}

Parameters for hyperbolic distribution have been estimated by the Nelder-Mead method, for the stable and the Meixner distribution –– by the Cramér\,--\,von~Mises distance. 
The results obtained for the log-returns are shown in Table~\ref{eng:tab:eng:marginals}.

\begin{table}
\centering
\caption{Marginal distribution estimation results}
\label{eng:tab:eng:marginals}
\begin{tabularx}{\textwidth}
{>{\hsize=2.4\hsize}X >{\hsize=0.4\hsize}Y *{4}{>{\hsize=0.8\hsize}R}}
\toprule \multicolumn{2}{c}{Parameters} & \multicolumn{1}{c}{RTS} &
\multicolumn{1}{c}{SBRF} & \multicolumn{1}{c}{GAZR} &
\multicolumn{1}{c}{GMKR} \\ \midrule[1pt]
\multirow{4}{*}{\parbox{\hsize}{Hyperbolic distribution}}
    &    $\pi$ &    0.00336 &    0.06100 &    0.06751 &    0.03301 \\
    &  $\zeta$ &    0.68417 &    0.80977 &    0.73310 &    3.31449 \\
    & $\delta$ &    0.00694	&    0.00823 &    0.00609 &    0.02232 \\
    &    $\mu$ &    0.00067 & $-$0.00003 & $-$0.00139 & $-$0.00076 \\ \midrule
\multirow{4}{*}{\parbox{\hsize}{Stable distribution}}
    & $\alpha$ &    1.53561 &    1.56414 &    1.86326 &    1.92994 \\
    &  $\beta$ &    0.21114 &    0.22262 &    0.85066 &    0.66465 \\
    & $\gamma$ &    0.00884 &    0.00926 &    0.00770 &    0.00999 \\
    & $\delta$ &    0.00020 &    0.00063 & $-$0.00099 & $-$0.00016 \\ \midrule
\multirow{4}{*}{\parbox{\hsize}{Meixner distribution}}
    & $\alpha$ &    0.03306 &    0.03064 &    0.02642 &    0.00428 \\
    &  $\beta$ &    0.30800 &    0.45599 &    0.22236 &    0.87412 \\
    & $\delta$ &    0.44168 &    0.51881 &    0.47397 &   18.31193 \\
    &    $\mu$ & $-$0.00099 & $-$0.00173 & $-$0.00143 & $-$0.03615 \\ \bottomrule
\end{tabularx}
\end{table}

To assess the quality of estimated parameters we use Kolmogorov\,--\,Smirnov, Anderson\,--\,Darling and Cramér\,--\,von~Mises goodness-of-fit tests.
These tests compare empirical observations with data simulated using marginals with obtained parameters.

\begin{table}
\centering
\caption{Statistical tests $p$-values}
\label{eng:tab:eng:margintest}
\begin{tabularx}{\textwidth}
{>{\hsize=2.2\hsize}X X *{4}{>{\hsize=0.7\hsize}Y}} \toprule
\multicolumn{1}{c}{Test} & \multicolumn{1}{c}{Distribution} & RTS & SBRF & GAZR & GMKR \bigstrut \\ \midrule[1pt]
\multirow{3}{*}{Kolmogorov\,--\,Smirnov}
    & Hyperbolic & 0.91 & 0.88 & 1.00 & 0.79 \\
    & Stable     & 0.89 & 0.94 & 0.87 & 0.94 \\
    & Meixner    & 0.99 & 0.95 & 1.00 & 0.96 \\ \midrule
\multirow{3}{*}{Anderson\,--\,Darling}
    & Hyperbolic & 0.88 & 0.94 & 1.00 & 0.93 \\
    & Stable     & 0.73 & 0.87 & 0.47 & 0.97 \\
    & Meixner    & 0.87 & 0.92 & 1.00 & 0.90 \\ \midrule
\multirow{3}{*}{Cram\'er\,--\,von Mises}
    & Hyperbolic & 0.94 & 0.89 & 1.00 & 0.90 \\
    & Stable     & 0.97 & 0.92 & 0.94 & 0.98 \\
    & Meixner    & 0.99\cellcolor{gray!33} & 0.95\cellcolor{gray!33} & 1.00\cellcolor{gray!33} & 0.98\cellcolor{gray!33} \\ \bottomrule
\end{tabularx}
\end{table}


\subsection*{Copula model parameters estimations}

% We use multivariate and R-vine copula models.
% Let us assume the following designations.

% Let $d$-dimensional copula \emph{C} be \textit{elliptical} if its distribution function is defined as follows
% \begin{equation} \label{eng:EllipCop}
%     C(u_1, u_2, \ldots,u_d|\theta) = F_d(F^{-1}(u_1|\theta),F^{-1}(u_2|\theta), \ldots,F^{-1}(u_d|\theta)|\theta),
% \end{equation}
% \begin{where*}
%     \item $F(x)$ is cumulative function of marginal distribution;
%     \item $F^{-1}(p)$ is inverse marginal distribution function;
%     \item $F_d(x_1, x_2, \dots, x_d)$ is $d$-dimensional joint cumulative function;
%     \item $\theta$ is parameter vector.
% \end{where*}

At this stage, we suggest the constructing of copula using two types of copula models: multivariate copula and regular vine (R-vine) copula. 
% For the sake of brevity, the known formulae corresponding to the copulas are not reported here, but are available in \cite{Nelsen1999, Joe2014,Czado2010,Cooke2015}.

First, we should generate points of the empirical copula also known as \textit{pseudo-observations}. 
Considering  Eq.~(\ref{eng:log-returns}) $\boldsymbol{r}_i = (r_{1,i}, r_{2,i},  \ldots, r_{T,i})^\intercal$ for all historical observations (log-returns) $i \in \overline{1,d}$, pseudo-observations are then defined as: %via the Eq.~(\ref{eng:pobs}):
\begin{equation} \label{eng:pobs}
    u_{t,i} = \frac{\text{rg}(r_{t,i})}{T + 1},\ \forall \ t \in \overline{1,T},\ i \in \overline{1,d},
\end{equation}
where
     $\text{rg}(r_{t,i})$ denotes the rank of $r_{t,i}$ (from lowest to highest) of the observed values $r_{\tau,i}, \tau \in \overline{1,T}$ \cite{Copula}
%
Each element $u_{t,i}$ is
between $0$ and $1$. 
% Pairs plots of the joint distribution of observed data and the pseudo-observations are shown on Fig.~\ref{eng:eng;pairs}.

In this study we use elliptical copulas of two families: Gaussian (normal) and Student's~$t$ copulas.

To estimate the copula parameters we use
pseudo-observations calculated by Eq.~(\ref{eng:pobs}). 
The decision of choosing the copula parameters is taken by "Inversion of Kendall’s tau" method~\cite{Koj2010}. 
Then we execute a parametric bootstrap-based goodness-of-fit (GoF) test of elliptical copulas to check their quality~\cite{Gen2009}. Estimated parameters are shown in Eq.~(\ref{eng:gausscopfit}) and~(\ref{eng:tcopfit}). Test results are shown in Table~\ref{eng:CopPars}. 
As one can see the parameters of elliptical copulas and results of GoF test are very close to each other.
\begin{equation} \label{eng:gausscopfit}
    \Sigma_{Gauss} = \left(
    \begin{array}{cccc}
        1 & & & \\
        0.723 & 1 & & \\
        0.642 & 0.540 & 1 & \\
        0.335 & 0.320 & 0.391 & 1
    \end{array} \right),
\end{equation}

\begin{equation} \label{eng:tcopfit}
    \Sigma_t = \left(
    \begin{array}{cccc}
        1 & & & \\
        0.723 & 1 & & \\
        0.642 & 0.540 & 1 & \\
        0.335 & 0.320 & 0.391 & 1
    \end{array} \right), \ \nu = 4.
\end{equation}

The negative side of using multivariate copula model is that we can not (a) check the quality, and (b) construct the cumulative distribution function of Student's~$t$ copula with non-integer degrees of freedom.

Alternative way to construct copula models is using R-vine copulas. 
As we know from \cite{Bedfort2002}, a $d$-dimensional vine is a copula constructed of $d(d - 1)/2$ usual bivariate copulas.  
The main advantage of the vine model is that all of its component copulas are represented by a pair-copula (two-dimensional function).
This copula is easier to be interpreted and visualized, and we have a lot of methods to work with it today~\cite{Cooke2015, Czado2010, Dissmann2013}. 

Following~\cite{Dissmann2013}, we use absolute empirical Kendall's $\tau$ as a measure of dependence, since it makes it independent of the assumed distribution. 
We use the same method~\cite{Koj2010} to estimate the parameters as we did it with multivariate copulas. Also, we can use non-integer degrees of freedom for copulas with two parameters. 
In addition, we choose different families for each pair~\cite{Bel2010, Clarke2007, Vuong1989}. 
Using abbreviations for copula types: $SG$~--- Survival Gumbel, $SC$~--- Survival Clayton, $SBB1$~--- Survival Clayton-Gumbel, $BB7$~--- Joe-Clayton, $Ind$~--- independence copula, the estimated R-vine copula is given by \cite{Czado2010}:
\begin{gather} \label{eng:vinefit}
    M = \left(
        \begin{array}{cccc}
        2 &   &   &   \\
        4 & 1 &   &   \\
        3 & 4 & 3 &   \\
        1 & 3 & 4 & 4 \\
        \end{array} \right), \\
    F = \left(
        \begin{array}{lll}%{*4{C{5em}}}
        Fr\ &\  &    \\
        Gu\ &\ Cl\ &   \\
        St\ &\ SBB1\ &\ SG\
        \end{array} \right), \nonumber\\    
    P_1 = \left(
        \begin{array}{ccc}
        0.544 & & \\
        1.098 & 0.158 & \\
        0.722 & 0.124 & 1.327
        \end{array} \right), \nonumber\\
    P_2 = \left(
        \begin{array}{ccc}
        0 &  & \\
        0 & 0 & \\
        7.566 & 1.682 & 0
        \end{array} \right). \nonumber
\end{gather}
\begin{where*}
    \item $M$ is a matrix which defines the tree structure;
    \item $F$ is a family matrix;
    \item $P_1$, $P_2$ are matrices with first and second parameters respectively.
\end{where*}

Matrix $F$ contains the following families of distributions: $St$~--- Students's, $Gu$~--- Gumbel, $Cl$~--- Clayton, $Fr$~--- Frank, $SBB1$~--- Survival BB1, $SG$~--- Survival Gumbel.

\begin{table}
\centering
\caption{The parameters estimation test results for multivariate Gaussian, Student's~$t$, and R-vine copulas}
\label{eng:CopPars}
\begin{tabularx}{\textwidth}
{>{\hsize=1.6\hsize}L >{\hsize=0.7\hsize}R>{\hsize=0.7\hsize}R}
\toprule
Copula    & Statistics  & $p$-value \\ \midrule
Gaussian  & $S_n=0,034$ & 0,19 \\
Student's & $S_n=0,391$ & 0,05 \\
R-vine    & $W=15,15$   & 0,95    \\ \bottomrule
\end{tabularx}
\end{table}

To check the estimated parameters, we use a goodness-of-fit test based on the Cram\'er-von Mises statistic, $S_n$  \cite{Koj2010} and the White’s information matrix equality, $W$ \cite{White1982}. 
The result of the test implementation is shown in Table~\ref{eng:CopPars}. As one can see the $p$-values of elliptical copulas are less than corresponding $p$-value of the R-vine copula.


\section*{Portfolio Application}\label{eng:PortApp}

In this section, we present some simulation results to compare the performances of VaR and CVaR on an equally weighted portfolio composed of $d=4$ assets. 
Then we applied the above results to compute the optimal weights of each asset, which is one of the major concerns in the field of portfolio risk management.

\subsection*{Mean-Conditional-Value-at-Risk Portfolio Optimization}

The advantage of the CVaR portfolio optimization is that we can formulate the mean-CVaR portfolio optimization as a linear programming problem~\cite{Rock2000}.
If we can find a portfolio with a low CVaR, then it will also have a low VaR~\cite{wnc04}.
We assume a "full investment" portfolio with only long positions, furthermore, to avoid a corner portfolio case, let the minimal weight be limited by $0.05$. 
The mean-CVaR portfolio weights we obtained are $0.05$, $0.114$, $0.384$, and $0.452$ for 
RTS, SBRF,  GAZP, and GMKR respectively.

Now let us compare VaR and CVaR values of equally weighted and mean-CVaR optimal portfolio obtaining for historical scenario by empirical methods~\cite{Rock2000}.
Let us consider $$\alpha = \{99.9\%,\, 99.5\%,\, 99\%,\, 95\%,\, 90\%\}$$ as a confidence level for VaR and CVaR computations. 
It can be seen from the simulation results (Table~\ref{eng:comparison}) that values of risk measures for optimal portfolio are distinctively better, as expected. 
Further, we will consider the optimal portfolio only.

\begin{table}[hbt]
\centering
\caption{Risk measures and associated bias for different portfolios and level,~$\alpha$}
\label{eng:comparison}
\begin{tabularx}{0.8\textwidth}
{>{\setlength{\hsize}{\hsize}}Y 
*{3}{>{\setlength{\hsize}{\hsize}}R
@{\ /\ }>{\setlength{\hsize}{\hsize}}X}}
\toprule
    \multirow{2}{*}{$\upalpha$, \%} & \multicolumn{6}{c}{$\emph{VaR}_\alpha$ / $\emph{CVaR}_\alpha$, $\times 10^{-2}$} \\ \cmidrule{2-7} 
    & \multicolumn{2}{c}{Optimal} & \multicolumn{2}{c}{Equiweighted} & \multicolumn{2}{c}{Bias, $\times 10^{-2}$} \\ \midrule
    90.0 & $1.31$ & $2.01$ & $1.32$ & $2.14$ & $0.01$ & $0.13$ \\ 
    95.0 & $1.69$ & $2.48$ & $1.82$ & $2.74$ & $0.13$ & $0.25$ \\ 
    99.0 & $2.59$ & $3.96$ & $2.79$ & $4.36$ & $0.21$ & $0.41$ \\ 
    99.5 & $3.63$ & $5.22$ & $4.05$ & $5.50$ & $0.43$ & $0.28$ \\ 
    99.9 & $5.62$ & $5.86$ & $6.05$ & $6.29$ & $0.44$ & $0.43$ \\ 
    \bottomrule
\end{tabularx}
\end{table}


\subsection*{Efficient Algorithm of Risk Measure Computation using Copula Models}

Now we propose the following algorithm based on Monte-Carlo simulation of pseudo-observations to compute the risk measures.

Algorithm~\ref{eng:PnL-quantiles} represents the method we used to compute VaR and CVaR. Method is based on Monte-Carlo simulation of pseudo-observations using proposed copula models with estimated parameters. 
In order to generate random samples (line~\ref{eng:Alg:simulation}) we used Gaussian and Student's $t$ copula parameters (Table~\ref{eng:CopPars}) and the R-vine array structure, Eq.~(\ref{eng:vinefit}). 
Then we transform each univariate pseudo-observation series to quantiles: $[0,1] \to \mathbb{R}$  (lines~\ref{eng:Alg:transform:start}--\ref{eng:Alg:transform:end}). 
In order to implement the transformation we take a quantile of each log-returns series using simulated pseudo-observations throw all dimensions of the copula as probabilities (line~\ref{eng:Alg:transform}). 
Using optimal mean-CVaR portfolio weights we compute portfolio's Profit \& Loss series (line~\ref{eng:Alg:PnL}) and risk measures (line~\ref{eng:Alg:risk-measures}). 

The obtained results for VaR and CVaR are shown in Table~\ref{eng:VaR-results} and \ref{eng:ES-results} respectively. As we can see, 
vine copula model is the most conservative one. 
It means that we will not loose more than we predict by the model, in other words, the model does not underestimate the risk. 
Thus, we can say that the R-vine copula model has superior forecasting ability than the Gaussian and the Student's~$t$ one.

\begin{algorithm}
\caption{Computation of Risk Measures by a Copula}
\label{eng:PnL-quantiles}
\begin{algorithmic}[1]
	\Require Log-returns $\{r_{i,t}\}$, weights $w_i$ of optimal portfolio, $i \in \overline{1,d}$, $d$-dimensional copula c.d.f. with parameters and the array structure for R-vine, level $\alpha$ for $VaR_\alpha$ and $CVaR_\alpha$ calculation.
	\State Generate a sample of pseudo-observations $\{\hat{u}_{i,s}\} \in [0, 1]^d, \ i \in \overline{1,d}, \ s \in \overline{1, S}$ according to the given copula.\label{eng:Alg:simulation}
	\State Transform simulated pseudo-observations to univariate quantiles:
	\label{eng:Alg:transform:start}
	\For {$i \in \overline{1,d}$}
        \For {$s \in \overline{1,S}$}
            \State Set $\hat{r}_{i,s} \gets F^{-1}_i (\hat{u}_{i,s})$. 
            \label{eng:Alg:transform}
	    \EndFor
	\EndFor \label{eng:Alg:transform:end}
	\State Compute the portfolio Profit \& Loss series:
	\For {$s \in \overline{1,S}$} 
	    \State $P\&L_k \gets \sum_{i=1}^d \hat{r}_{i,s} \cdot w_i$. \label{eng:Alg:PnL}
	\EndFor
	\State Calculate $\emph{VaR}_\alpha$, $\emph{CVaR}_\alpha$ of Profit \& Loss series \label{eng:Alg:risk-measures} % by Eq.~\ref{eng:VaR}~and~\ref{eng:ES2} respectively.
	\Ensure $\emph{VaR}_\alpha$ and $\emph{CVaR}_\alpha$ of simulated Profit \& Loss series.
\end{algorithmic}
\end{algorithm}

\begin{table}
\centering
\caption{VaR obtained empirically and estimated by Gaussian\,/\,Student's~$t$\,/\,R-vine copulas}
\label{eng:VaR-results}
\begin{adjustbox}{max width=\textwidth}
\begin{tabularx}{\textwidth}{
>{\setlength{\hsize}{\hsize}}Y|
>{\setlength{\hsize}{\hsize}}Y
*{2}{|>{\setlength{\hsize}{0.4\hsize}}R
*{2}{@{\,/\,}>{\setlength{\hsize}{0.4\hsize}}R}}}
\toprule
\multicolumn{1}{c|}{Level, \%} & \multicolumn{1}{c|}{$\widehat{\emph{VaR}}$, $\times 10^{-2}$} & \multicolumn{3}{c|}{$\emph{VaR}^{est}$, $\times 10^{-2}$} & \multicolumn{3}{c}{$\Delta$, $\times 10^{-3}$} \bigstrut[t] \\ \midrule
90.0   & $1.31$ & $1.57$ & $1.71$ & $1.51$ & $2.55$ & $3.94$ & $1.94$ \\ 
95.0   & $1.69$ & $2.15$ & $2.37$ & $1.94$ & $4.56$ & $6.81$ & $2.48$ \\ 
99.0   & $2.59$ & $3.28$ & $3.64$ & $3.28$ & $6.95$ & $10.54$ & $6.97$ \\ 
99.5 & $3.63$ & $3.70$ & $4.02$ & $3.94$ & $0.69$ & $3.95$ & $3.13$ \\ 
99.9 & $5.62$ & $5.63$ & $5.83$ & $6.01$ & $0.12$ & $2.18$ & $3.90$ \\ \bottomrule
\end{tabularx}
\end{adjustbox}
\end{table}

\begin{table}
\centering
\caption{CVaR obtained empirically and estimated by Gaussian\,/\,Student's~$t$\,/\,R-vine copula}
\label{eng:ES-results}
\begin{adjustbox}{max width=\textwidth}
\begin{tabularx}{\textwidth}{
>{\setlength{\hsize}{\hsize}}Y|
>{\setlength{\hsize}{\hsize}}Y
*{2}{|>{\setlength{\hsize}{0.4\hsize}}R
*{2}{@{\,/\,}>{\setlength{\hsize}{0.4\hsize}}R}}}
\toprule 
\multicolumn{1}{c|}{Level, \%} & \multicolumn{1}{c|}{$\widehat{\emph{CVaR}}$, $\times 10^{-2}$} & \multicolumn{3}{c|}{$\emph{CVaR}^{est}$, $\times 10^{-2}$} & \multicolumn{3}{c}{$\Delta$, $\times 10^{-3}$} \bigstrut[t] \\ \midrule
90.0   & $2.01$ & $2.38$ & $2.58$ & $2.32$ & $3.68$ & $5.66$ & $3.12$ \\ 
95.0   & $2.48$ & $2.95$ & $3.17$ & $2.92$ & $4.65$ & $6.89$ & $4.35$ \\ 
99.0   & $3.96$ & $4.25$ & $4.35$ & $4.50$ & $2.98$ & $3.94$ & $5.41$ \\ 
99.5 & $5.22$ & $5.05$ & $4.95$ & $5.43$ & $-1.76$ & $-2.72$ & $2.06$ \\ 
99.9 & $5.86$ & $5.81$ & $6.00$ & $6.01$ & $-0.51$ & $1.38$ & $1.51$ \\ \bottomrule
\end{tabularx}
\end{adjustbox}
\end{table}


\subsection*{Stability Study and Risk Measure Curve}

Now let us make a stability research of proposed method. For this, we make bootstrap procedure replicating Algorithm~\ref{eng:PnL-quantiles}.
The obtained results are shown in Table~\ref{eng:VaR-boot}, \ref{eng:ES-boot}.

We report the bias, the standard deviation (SD) and the mean square error (MSE) based  on  $N=200$  replications.  The bias value is better at lower levels: 99\%, 95\%, 90\% for VaR, and 95\%, 90\% for CVaR.
The SD and the MSE metrics show the greater instability of vine copula related to Gaussian (the most stable one) and Student's~$t$ model.

\begin{table}
\centering
\caption{VaR estimation by Gaussian\,/\,Student's~$t$\,/\,R-vine~copula obtained by bootstrap procedure}
\label{eng:VaR-boot}
\begin{adjustbox}{max width=\textwidth}
\begin{tabular}{c*{4}{|r@{\,/\,}r@{\,/\,}r}} \toprule
\multicolumn{1}{c|}{Level, \%} & \multicolumn{3}{c|}{$\overline{\emph{VaR}}_\alpha$, $\times 10^{-2}$} & \multicolumn{3}{c|}{$\Delta, \times 10^{-3}$} & \multicolumn{3}{c|}{SD, $\times 10^{-2}$} & \multicolumn{3}{c}{RMSE, $\times 10^{-3}$} \\ \midrule
90.0   & $1.61$ & $1.62$ & $1.61$ & $2.99$ &  $3.03$ &  $2.96$ & $0.80$ & $0.79$ & $0.85$ & $3.10$ & $3.13$ & $3.08$ \\
95.0   & $2.19$ & $2.20$ & $2.18$ & $5.00$ &  $5.06$ &  $4.93$ & $1.18$ & $1.37$ & $1.48$ & $5.14$ & $5.24$ & $5.14$ \\
99.0   & $3.46$ & $3.50$ & $3.46$ & $8.71$ &  $9.11$ &  $8.78$ & $1.79$ & $2.32$ & $2.47$ & $8.89$ & $9.40$ & $9.11$ \\
99.5 & $4.04$ & $4.21$ & $4.09$ & $4.15$ &  $5.82$ &  $4.64$ & $4.22$ & $6.10$ & $5.69$ & $5.91$ & $8.42$ & $7.33$ \\
99.9 & $5.62$ & $5.53$ & $5.49$ & $0.05$ & $-0.86$ & $-1.21$ & $3.81$ & $5.18$ & $5.16$ & $3.81$ & $5.24$ & $5.29$ \\ \bottomrule
\end{tabular}
\end{adjustbox}
\end{table}

\begin{table}
\centering
\caption{CVaR estimation by Gaussian\,/\,Student's~$t$\,/\,Vine~copula obtained by bootstrap procedure}
\label{eng:ES-boot}
\begin{adjustbox}{max width=\textwidth}
\begin{tabular}{c*{4}{|r@{\,/\,}r@{\,/\,}r}} \toprule
\multicolumn{1}{c|}{Level, \%} & \multicolumn{3}{c|}{$\overline{\emph{CVaR}}_\alpha$, $\times 10^{-2}$} & \multicolumn{3}{c|}{$\Delta, \times 10^{-3}$} & \multicolumn{3}{c|}{SD, $\times 10^{-2}$} & \multicolumn{3}{c}{RMSE, $\times 10^{-3}$} \\ \midrule
90.0   & $2.46$ & $2.47$ & $2.45$ &  $4.53$ &  $4.62$ &  $4.43$ & $1.08$ & $1.32$ & $1.35$ & $4.66$ & $4.80$ & $4.63$ \\ 
95.0   & $3.06$ & $3.08$ & $3.05$ &  $5.78$ &  $5.92$ &  $5.62$ & $1.49$ & $1.90$ & $1.91$ & $5.97$ & $6.21$ & $5.93$ \\ 
99.0   & $4.36$ & $4.40$ & $4.34$ &  $4.08$ &  $4.46$ &  $3.86$ & $3.05$ & $4.31$ & $4.21$ & $5.09$ & $6.19$ & $5.71$ \\ 
99.5 & $5.00$ & $4.99$ & $4.94$ & $-2.26$ & $-2.31$ & $-2.86$ & $4.05$ & $5.40$ & $5.34$ & $4.63$ & $5.87$ & $6.05$ \\ 
99.9 & $5.76$ & $5.68$ & $5.68$ & $-1.03$ & $-1.79$ & $-1.88$ & $2.45$ & $3.86$ & $3.52$ & $2.65$ & $4.25$ & $3.98$ \\ \bottomrule
\end{tabular}
\end{adjustbox}
\end{table}

% Fig.~\ref{eng:VaR-curve} and \ref{eng:ES-curve} show the dynamics of Profit \& Loss series and the movement of 95\%-level VaR and CVaR respectively. VaR and CVaR curves simulated by copula models are lower than a historical one, thus, the proposed models are more conservative. % Daulet, 22.08
% .
% According to the results of a retrospective forecast 
% % (Fig.~\ref{eng:VaR-curve} and \ref{eng:ES-curve})
% , it can be seen that proposed copula models approximate Profit \& Loss series satisfactory. 
% We note that the Profit \& Loss curve movements through a copula line based on CVaR models twice only (Fig.~\ref{eng:ES-curve}) while the Profit \& Loss curve breaks VaR lines~--- five times of 504 observations (Fig.~\ref{eng:VaR-curve}). Both cases correspond less that 2\%, therefore, curves satisfied the given 95\%-level. From a technical point of view, the support CVaR curve would be more preferable.

The $p$-values of the Kupiec's VaR test~\cite{Kupiec95} for all proposed copula models are greater than the critical level $\alpha = 0.05$. 
All three models have superior prediction ability than usual empirical method.  
Table~\ref{tab:eng:Kupiec} shows the results of Kupiec's VaR test \cite{Kupiec95}.

\begin{table}[hbt!]
    \centering
    \caption{Kupiec's test $LR$-statistics and $p$-value for VaR at 95\% confidence level computation}
    \label{tab:eng:Kupiec}
    \setlength{\tabcolsep}{10pt}
    \begin{tabular}{lcc} \toprule
        Method & LR-statistics & $p$-value \\ \midrule
        empirical & 0.00 & 0.98 \\
        Gaussian copula & 3.02 & 0.08 \\
        $t$-copula & 10.29 & 0.00 \\
        R-vine copula & 2.17 & 0.14 \\ \bottomrule
    \end{tabular}
\end{table}



