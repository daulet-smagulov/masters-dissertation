\newpage

\anonsection{Введение}

В данной работе рассчитываются общие меры риска, используемые в риск-менеджменте: это \textit{Value-at-Risk} (VaR) и \textit{Conditional-Value-at-Risk} (CVaR), которые можно определить для разных уровней значимости. 
В статье О. Крицкий и М. Ульянова \cite{Kritski2007} показано, что при наличии корреляции в динамике активов портфеля оценка одномерных величин VaR и CVaR оказывается неадекватной по отношению к риску портфеля. 
Поэтому для оценки риска портфеля необходимо использовать $d$-мерные величины, определяемые с помощью многомерной зависимости.

Существует много всевозможных подходов, которые активно применяются для представления многомерной зависимости, например, метод главных компонент, байесовские сети, нечёткая логика, факторный анализ и использование совместной многомерной функции распределения \cite{Huynh2014, Kole2007}. 
Зависимость между случайными величинами $X_1, X_2, \ldots, X_d$ полностью определяется совместной функцией распределения $F(X_1, X_2, \ldots, X_d)$. 
Идея разделения этой функции на две части, одна из которых определяет структуру зависимости, а другая -- маргинальное поведение, приводит к понятию \boldit{копула}.

В 1959 году Абе Шкляр \cite{Sklar1959} впервые вывел и доказал теорему о том, что набор маргинальных распределения может быть объединён и сформировать одно многомерное распределение посредством копулы. 
Копула содержит всю информацию о структуре зависимости вовлеченных переменных. 
В статье \cite{Penikas2010} автор представил концепты копулярных моделей и их применение в различных финансовых вопросах, включая риск-менеджмент.

Сегодня существует множество способов описания финансовых данных с использованием нормального (Гауссового) распределения. 
Хорошо известно, что \boldit{Гауссова копула}, т.~е. копула, построенная нормальными маргинальными распределениями, используется в качестве одного из способов описания портфеля в теории Марковица. 
С другой стороны, многие эмпирические исследования показали, что распределение Гаусса имеет множество недостатков в описании зависимости финансовых данных \cite{Limp2011, Rachev2005, Wilmott2007}. 
Более того, стандарт EBA \cite{EBA2015} не рекомендует использовать Гауссовы копулы для моделирования финансового риска. 
Например, в большинстве случаев более подходящей оказывается копула \boldit{$t$\,--\,Стьюдента} с несколькими целыми степенями свободы (обычно три или четыре).
Аналитическое решение для нахождения меры чувствительности риска -- CVaR -- было представлено в статье \cite{Stoyanov2013}.

Главной целью моего исследования в данной работе было сравнить значения оценки финансового риска в рамках управления портфелем путём определения параметров моделей копула в в соответствии с динамикой и корреляцией цен составляющих его активов. 
Иными словами, задачей исследования было протестировать модели копула путём сравнения риск метрик.

Исследования \cite{Ane2003, Kole2007, Lourme2016, Xu2008} рыночного риска в рамках управления портфелем во многом сходны друг с другом и отличаются зачастую только используемыми данными и незначительными деталями в оценке моделей копулы. 
Среди комплексных исследований я хотел бы выделить статью Тьерри Ане и Сесиль Харуби \cite{Ane2003} – одну из первых, в которой авторы использовали копулу Клейтона в качестве инструмента для определения структуры зависимости между акциями международного фондового индекса.
Александр Лурме и др.\,\cite{Lourme2016} фокусируются на тестировании и сравнении Гауссовой и $t$-копул в рамках риск-менеджмента. 
Они предлагают $d$-мерную компактную Гауссову и $t$\,--\,Стьюдента  доверительную область, внутри которой вектор из выборки случайной величины, равномерно распределённой на отрезке $[0, 1]$, попадает с вероятностью~$\alpha$.
Результаты показывают, что $t$\,--\,Стьюдента VaR-модель является более консервативной и адекватной по сравнению c Гауссовой.
Портфель акций, облигаций и сделок на недвижимость был рассмотрен в \cite{Kole2007} для определения важность выбора правильной копулы для риск-менеджмента. 
Были опробованы копулы Гауссовы, Стьюдента и Гумбеля для моделирования зависимости дневных доходностей индексов, которые аппроксимируют данные три класса активов. 
Затем по расчетам VaR было установлено, что гауссова копула слишком оптимистична в отношении преимуществ диверсификации активов, тогда как копула Гумбеля слишком пессимистична.

При моделировании копулы важной проблемой является оценка параметров. 
В настоящее время разработано множество алгоритмов построения копул. 
Для оценки модели копула существуют три метода: полный параметрический \cite{Patton2006}, полу-параметрический \cite {Chen2006, Lourme2016} и непараметрический метод \cite{Fermanian2003, Kim2007}. 
Полный параметрический метод реализуется посредством двухэтапной оценки максимального правдоподобия (Maximum Likelihood Estimation -- MLE), предложенный Гарри Джо \cite{Joe1997, Joe2014}.
Копула фитится с использованием двухступенчатого параметрического подхода MLE, также называемого методом оценки частных распределений (Inference Functions for Margins -- IFM). 
Этот метод фитит копулу в два этапа: (1) оценивает параметры маргиналов и (2) и затем параметры копулы.

Для двумерного случая основными семейства копул -- эллиптические (Гауссова, $t$\,--\,Стьюдента), архимедовы (Клейтон, Франка, Джо) и экстремальные (Гумбеля, Коши).
В диссертационном исследовании \cite{Xu2008} был использован двухэтапный MLE-метод, в ходе которого автор использует все возможные комбинации различных маргинальных распределения (нормальное, Стьюдента с асимметрией и без) и различные архимедовы копулы для оценки и тестирования. 
Решение о выборе маргинального распределение определяется после второго этапа MLE. 
Для этой цели была предложен модифицированный для прогнозирования тест Хансена \cite{Hansen2005}; данный тест позволяет определить наиболее адекватную копулу.

Многомерные копулы, основанные на одном распределении (например, Гауссова или $t$\,--\,Стьюдента) или созданные из так называемых функций-генераторов, не обладают гибкостью точного моделирования зависимости между большим числом переменных \cite{Brechmann2013}. Эти недостатки предопределили направление дальнейших исследований, в результате которых Гарри Джо \cite{Joe1996} была предложена концепция \boldit{регулярных вайн-копул} (vine), в дальнейшем более подробно исследованная в \cite{Brechmann2013, Cooke2015}.
R-vine представляют собой гибкую графическую модель для описания многомерных копул, построенных с использованием каскада двумерных копул (двумерных функций). 
Такую копулу легче интерпретировать и визуализировать, и сегодня существует масса способов для работы с ней \cite{Cooke2015, Czado2010, Dissmann2013}.
Например, в исследовании \cite{Dissmann2013} описаны новые алгоритмы оценки параметров вайн-копулы и моделирования с использованием специальных R-vine копул.
Выбор R-vine древовидной структуры определяется на основе алгоритма максимального покрывающего дерева (Maximum spanning tree -- MST), в котором весовые коэффициенты выбраны таким образом, чтобы отразить большие зависимости.

В данной работе я осуществляю расчёт модели копула для портфеля, состоящего из четырёх активов. Для этого я использую четыре временных ряда из ежедневных цен фьючерсов. Я использую полный параметрический MLE-метод оценки копулы, описанный Г. Джо \cite{Joe1997, Joe2014}.

Моя работа содержит три нововведения. 
Во-первых, я показываю, что Гауссовы, Стьюдента и вайн-копулы могут быть использованы для представления многомерной зависимости в коротких временных рядах (только 253 наблюдения), в то время как применение копул изучалось только по отношению к длительным наблюдениям \cite{Dissmann2013, Kole2007, Lourme2016}.
Во-вторых, я использую маргинальные распределения, отличные от нормального: гиперболическое~\cite{Barndoff1983}, устойчивое~\cite{Rachev2005} Мейкснера~\cite{Schoutens2002}. 
Стоянов и др.~\cite{Stoyanov2013} рассматривают подобную задачу, используя только симметричные варианты распределений Стьюдента и устойчивого, а также обобщенное нормальное распределение. 
В-третьих, при моделировании R-vine копулы я использую 
нецелые степени свободы для двух-параметрических копул, которые существенно увеличивают гибкость и расширяют возможности копул в рамках риск-менеджмента.

Чтобы сравнить показатели различных мер риска (VaR, CVaR), мы предлагаем использовать Монте-Карло симуляции для моделирования оптимального CVaR-портфеля.
Преимущество оптимизации портфеля CVaR заключается в том, что оптимизация CVaR-портфеля производится путём решения задачи линейного программирования \cite{Rock2000}.
Мы заимствуем и сравниваем исторические P\&L с кривыми VaR и CVaR, оцениваемых с помощью моделей копул, на разных уровнях значимости.

В разделе~\ref{section:methodology} описана методология модели. 
В разделе~\ref{section:calibration} описаны исходне данные и приведены результаты калибровки модели: оценённые параметры и результаты оптимизации портфеля. 
Раздел~\ref{section:risk-measures} посвящён расчёту риск метрик.