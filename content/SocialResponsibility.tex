\section{Социальная ответственность}
\label{section:socialResp}

Охрана труда~--- это система законодательных, социально\-экономических, организационных, технологических, гигиенических и лечебно\-профилактических мероприятий и средств, обеспечивающих безопасность, сохранение здоровья и работоспособности человека в процессе труда~\cite{Osah2016}.
Правила по охране труда и техники безопасности вводятся в целях предупреждения несчастных случаев, обеспечения безопасных условий труда работающих и являются обязательными для исполнения рабочими, руководящими, инженерно\-техническими работниками.

Данная работа была проведена за компьютером, или персональной электронной вычислительной машиной (ПЭВМ).
Поэтому рамках текущего раздела будут рассмотрены соответствующие вопросы, связанные со следующими компонентами охраны труда:

\begin{itemize}
    \item выявление и изучение вредных и опасных производственных факторов при работе с ПЭВМ;
    \item оценка охраны труда;
    \item определение способов снижения действия вредных факторов до безопасных пределов или по возможности до полного их исключения;
    \item техника производственной безопасности;
    \item охрана окружающей среды.
\end{itemize}


\subsection{Описание рабочего места}

Объектом исследования данного раздела является рабочее место (РМ) и помещение, в котором оно находится. 
Параметры помещения, в котором осуществлялась данная работа: длина $a = 8$ м, ширина $b = 4$ м, высота $h = 3$ м. 
Площадь помещения составляет $S = a \cdot b = 32$ $\text{м}^2$, объём равен $V = a \cdot b \cdot h = 96$~$\text{м}^3$.

В помещении присутствует окно, через которое осуществляется вентиляция помещения, его ширина и высота составляют 1,5 и 2 м соответственно.
В помещении используется комбинированное освещение~--- искусственное (люминесцентные лампы типа ЛБ) и естественное (свет из окна). 
В зимнее время помещение отапливается. 
Электроснабжение сети переменного напряжения 220В. 
Согласно ГОСТ 12.1.013-78, данное помещение относится к типу без повышенной опасности в отношении поражения человека электрическим током~\cite{Gost2001}.

Рабочая поверхность имеет высоту 0,75 м.
Работа производилась на двух разных ПЭВМ: персональный компьютер (ПК) и ноутбук.
ПК обладает следующими характеристиками: процессор Core i3, оперативная память 4 Гб, операционная система Microsoft Windows 7 Professional, частота процессора – 2,00 ГГц, монитор с диагональю 24$"$, разрешением $1920 \times 1080$ и частотой 120 Гц.
Ноутбук имеет следующие характеристики: процессор Core i5, оперативная память 8 Гб, операционная система Mac OS X 10.11 El Capitan, экран с диагональю 13,3$"$ и разрешением $2560 \times 1440$.


\subsection{Требования к ПЭВМ и организация работы}

\subsubsection{Организационные мероприятия}

Весь персонал обязан знать и строго соблюдать правила техники безопасности. 
Обучение персонала технике безопасности и производственной санитарии состоит из вводного инструктажа и инструктажа на рабочем месте ответственным лицом.

Проверка знаний правил техники безопасности проводится квалификационной комиссией после обучения на рабочем месте.
Проверяемому присваивается соответствующая его знаниям и опыту работы квалификационная группа по технике безопасности и выдается специальное удостоверение.

Лица, обслуживающие электроустановки, не должны иметь увечий и болезней, мешающих производственной работе.
Состояние здоровья устанавливается медицинским освидетельствованием.

\subsubsection{Технические мероприятия}

Рациональная планировка рабочего места предусматривает четкий порядок и постоянство размещения предметов, средств труда и документации. 
На рис.~\ref{ris:SR:zones} изображены следующие зоны рабочего пространства:

\begin{itemize}
    \item зона максимальной досягаемости рук;
    \item зона досягаемости пальцев при вытянутой руке;
    \item зона лёгкой досягаемости ладони;
    \item оптимальное пространство для грубой ручной работы;
    \item оптимальное пространство для тонкой ручной работы.
\end{itemize}

\imgh[ht]{SR-zones.png}{width=\textwidth}{Зоны досягаемости рук в горизонтальной плоскости}{SR:zones}

То, что требуется для выполнения работ, должно располагаться в легко досягаемой зоне рабочего пространства.
Оптимальное размещение предметов труда и документации в зонах досягаемости рук: дисплей размещается в зоне \textit{а} (в центре), клавиатура~--- в зоне \textit{г}/\textit{д}, системный блок~--- в зоне \textit{б} (слева), принтер~--- в зоне \textit{а} (справа), литература и документация, необходимая при работе,~--- в зоне легкой досягаемости ладони \textit{в} (слева), а литература, не используемая постоянно,~--- в выдвижных ящиках стола. 

При проектировании письменного стола должны быть учтены следующие требования, описанные в СанПиН 2.2.2/2.4.1340-03~\cite{SanPin2003}:

\begin{enumerate}
[leftmargin=0pt,itemindent=\parindent+\labelwidth+\labelsep]
    \item Рекомендованная высота рабочей поверхности стола в пределах 680\,--\,800 мм. 
    Высота рабочей поверхности, на которую устанавливается клавиатура, должна быть 650 мм. 
    Рабочий стол должен быть шириной не менее 700 мм и длиной не менее 1400 мм. 
    Должно иметься пространство для ног высотой не менее 600 мм, шириной~--- не менее 500 мм, глубиной на уровне колен~--- не менее 450 мм и на уровне вытянутых ног~--- не менее 650 мм.
    \item Рабочее кресло должно быть подъёмно\-поворотным и регулируемым по высоте и углам наклона сиденья и спинки, а также расстоянию спинки до переднего края сиденья. 
    Рекомендуемая высота сиденья над уровнем пола 420\,--\,550 мм. 
    Конструкция рабочего кресла должна обеспечивать: ширину и глубину поверхности сиденья не менее 400 мм; поверхность сиденья с заглублённым передним краем.
    \item Монитор должен быть расположен на уровне глаз оператора на расстоянии 500\,--\,600 мм. 
    Согласно нормам угол наблюдения в горизонтальной плоскости должен быть не более $45^\circ$ к нормали экрана.
    Лучше, если угол обзора будет составлять $30^\circ$. 
    Кроме того, должна быть возможность выбирать уровень контрастности и яркости изображения на экране.
    \item Должна предусматриваться возможность регулирования экрана:
    \begin{itemize}[leftmargin=\leftmargin+\labelwidth+\labelsep]
        \item по высоте $+3$ см;
        \item по наклону от 10 до 20 градусов относительно вертикали;
        \item в левом и правом направлениях.
    \end{itemize}
    \item Клавиатуру следует располагать на поверхности стола на расстоянии 100\,--\,300 мм от края. 
    Нормальным положением клавиатуры является её размещение на уровне локтя оператора с углом наклона к горизонтальной плоскости в $15^\circ$. 
    Более удобно работать с клавишами, имеющими вогнутую поверхность, четырёхугольную форму с закруглёнными углами.
    Конструкция клавиши должна обеспечивать оператору ощущение щелчка. 
    Цвет клавиш должен контрастировать с цветом панели.
    \item При однообразной умственной работе, требующей значительного нервного напряжения и большого сосредоточения, рекомендуется выбирать неяркие, малоконтрастные цветочные оттенки, которые не рассеивают внимание (малонасыщенные оттенки холодного зеленого или голубого цветов). 
    При работе, требующей интенсивной умственной или физической напряженности, рекомендуются оттенки тёплых тонов, которые возбуждают активность человека. 
\end{enumerate}


\subsection{Анализ опасных и вредных производственных факторов}

Опасным производственным фактором, согласно~\cite{Dahf2016}, называется такой производственный фактор, воздействие которого в определённых условиях приводит к травме или резкому ухудшению здоровья.
Вредным производственным фактором называется такой производственный фактор, воздействие которого на работающего в определенных условиях приводит к заболеванию или снижению трудоспособности.

На инженера, работа которого связана с моделированием на ПЭВМ, воздействуют следующие факторы:

\begin{itemize}
    \item физические:
    \begin{itemize}[leftmargin=\labelwidth+\labelsep]
        \item температура и влажность воздуха;
        \item шум;
        \item статическое электричество;
        \item электромагнитное поле низкой частоты и ионизирующее излучение;
        \item освещённость РМ.
    \end{itemize}
    \item психофизиологические:
    \begin{itemize}[leftmargin=\labelwidth+\labelsep]
        \item физические перегрузки (статические, динамические);
        \item нервно-психологические перегрузки (умственное напряжение, монотонность труда, эмоциональные перегрузки).
    \end{itemize}
\end{itemize}

Учитываем, что работа производилась на ПЭВМ с жидкокристаллическим экраном, которые не имеют ионизирующего излучения. 
Поэтому оценка параметров по данному фактору производиться не будет.


\subsubsection{Электромагнитное поле}

Известно, что ПЭВМ являются источником электромагнитного поля (ЭМП) радиочастотного диапазона. 
При его длительном постоянном воздействии на человека наблюдаются нарушения сердечнососудистой, дыхательной и нервной систем, появляется утомляемость, ухудшение самочувствия, гипотония, также характерна головная боль, изменение проводимости сердечной мышцы. 
Переход ЭМП в тепловую энергию вызывает повышение температуры тела, локальный избирательный нагрев тканей, органов, клеток.

При работе с компьютером предельно допустимые уровни (ПДУ) электромагнитных полей (ЭМП) и максимальное время нахождения под его излучением нормируются СанПиН 2.2.2/2.4.1340-03~\cite{SanPin2003}:
\begin{itemize}
    \item до 10 $\text{мкВт}/\text{см}^2$~--- не более 8 часов в сутки;
    \item 10\,--\,100 $\text{мкВт}/\text{см}^2$~--- не более 2 часов в сутки;
    \item более 100 $\text{мкВт}/\text{см}^2$~--- не более 20 мин в сутки.
\end{itemize}

Напряженность ЭМП на расстоянии 50 см вокруг ПЭВМ по электрической составляющей должна быть:
\begin{itemize}
    \item в диапазоне частот 5 Гц\,--\,2 кГц~--- 25 В\,/\,м;
    \item в диапазоне частот 2\,--\,400кГц~--- 2,5 В\,/\,м.
\end{itemize}

Плотность магнитного потока должна быть не более:
\begin{itemize}
    \item в диапазоне частот 5 Гц\,--\,2 кГц~--- 250 нТл;
    \item в диапазоне частот 2\,--\,400кГц~--- 25 нТл.
\end{itemize}

Среди средств защиты от ЭМП выделяют следующие~\cite{Bzd2007}:
\begin{enumerate}
[leftmargin=0pt,itemindent=\parindent+\labelwidth+\labelsep]
    \item Организационные мероприятия~--- выбор рациональных режимов работы оборудования, ограничение места и времени нахождения персонала в зоне воздействия ЭМП, т.\,е. защита расстоянием и временем;
    \item Инженерно-технические мероприятия, включающие рациональное размещение оборудования, использование средств, ограничивающих поступление электромагнитной энергии (поглотители мощности, экранирование и т.п.);
    \item Лечебно-профилактические мероприятия в целях предупреждения, ранней диагностики и лечения здоровья персонала;
    \item Средства индивидуальной защиты, к которым относятся защитные очки, щитки, шлемы, защитная одежда, выполненная из металлизированной ткани (кольчуга). При этом следует отметить, что использование СИЗ возможно при кратковременных работах и является мерой аварийного характера.
\end{enumerate}


\subsubsection{Электростатическое поле}

Опасность возникновения статического электричества проявляется в возможности образования электрической искры и вредном воздействии его на человеческий организм, причём не только в случае непосредственного контакта с зарядом, но и за счёт действий электрического поля, которое возникает при заряде. 
Cтатическое электричество также возникает на экране монитора при включённом питании компьютера. 

Согласно СанПиН 2.2.2/2.4.1340-03~\cite{SanPin2003}, при работе с ПЭВМ оптимальные значения параметров электростатического поля следующие: напряжённость~--- 15~кВ\,/\,м; электростатический потенциал экрана видеомонитора~--- 500~В.

Основные способы защиты от статического электричества следующие: заземление оборудования, увлажнение окружающего воздуха. Также целесообразно применение полов из непроводяещего материала.


\subsubsection{Производственный шум}

Предельно допустимый уровень (ПДУ) шума - это уровень фактора, который при ежедневной (кроме выходных дней) работе, но не более 40 часов в неделю в течение всего рабочего стажа, не должен вызывать заболеваний или отклонений в состоянии здоровья, обнаруживаемых современными методами исследований в процессе работы или в отдаленные сроки жизни настоящего и последующих поколений. 
Соблюдение ПДУ шума не исключает нарушения здоровья у сверхчувствительных лиц.

Допустимый уровень шума регулируется ГОСТ 12.1.003-83~\cite{Gost2002} и СанПиН 2.2.4/2.1.8.10-32-2002~\cite{SanPin2002}. 
Уровень шума на рабочем месте математиков\-программистов и операторов видеоматериалов не должен превышать 50 дБА, а в залах обработки информации на вычислительных машинах~--- 65 дБА.
При значениях шума выше допустимого уровня необходимо предусмотреть средства коллективной и индивидуальной защиты:
\begin{itemize}
    \item в качестве средств коллективной защиты (СКЗ) выступают следующие мероприятия:
    \begin{itemize}[leftmargin=\labelwidth+\labelsep]
        \item устранение причин шума или существенное его ослабление в источнике образования;
        \item изоляция источников шума от окружающей среды средствами звукопоглощения;
        \item применение средств, снижающих шум и вибрацию на пути их распространения;
    \end{itemize}
    \item средствами индивидуальной защиты (СИЗ) являются:
    \begin{itemize}[leftmargin=\labelwidth+\labelsep]
        \item спецодежда, спецобувь;
        \item защитные средства органов слуха: наушники, беруши, антифоны.
    \end{itemize}
\end{itemize}

В целях защиты от шумов вентиляторы ЭВМ заключают в защитный кожух и устанавливают их внутри корпуса. 
Для снижения уровня шума стены и потолок помещений, где установлены компьютеры, могут быть облицованы звукопоглощающими материалами с максимальными коэффициентами звукопоглощения в области частот 63\,--\,8000 Гц.
Кроме того, для снижения шума на рабочем месте можно предпринять следующие действия:
\begin{itemize}
    \item установить пластиковые окна, для улучшения шумоизоляции;
    \item использовать звукопоглощающие материалы, такие как пенополистирол, поролоновые маты, пробковые полотна и плиты;
    \item устройство подвесного потолка, который служит звукопоглощающим экраном.
\end{itemize}


\subsubsection{Микроклимат}

Микроклимат производственных помещений~--- это климат внутренней среды помещений, который определяется действующими на организм сочетаниями температуры, влажности и скорости движения воздуха, а также температуры окружающих поверхностей~\cite{Bzd2007}.

Нормы оптимальных и допустимых условий микроклимата регулирует СанПиН 2.2.4.548–96~\cite{SanPin1997}.
Эти нормы устанавливаются в зависимости от времени года, характера трудового процесса и производственного помещения.
Все категории работ разграничиваются на основе интенсивности энергозатрат организма в ккал\,/\,ч (Вт). 
Работа математика-программиста, выполняемая сидя и сопровождающаяся незначительным физическим напряжением, относится к категории Ιа~--- работа с интенсивностью энергозатрат до 120 ккал\,/\,ч (до 139 Вт). 
Для данной категории допустимые нормы микроклимата помещения представлены в таблице~\ref{tab:SR:mcclimate}.

\begin{table}
\centering
\caption{Оптимальные и допустимые параметры микроклимата для категории Ia}
\label{tab:SR:mcclimate}
\begin{tabularx}{\textwidth}
{>{\hsize=0.9\hsize}Y >{\hsize=0.9\hsize}Y >{\hsize=1.1\hsize}Y >{\hsize=1.1\hsize}Y} 
    \toprule
    Период года &  Температура, $^\circ$C & Относительная влажность, \% & Скорость движения воздуха, м/с 
    \\ \midrule[1pt]
    \multicolumn{4}{c}{Оптимальные значения}
    \\ \midrule
    Холодный и переходный & 21\,--\,23 & 50\,--\,55 & 0,1 \\ \midrule[0pt] 
    Тёплый                & 22\,--\,24 & 50\,--\,55 & 0,5 \\ \midrule
    \multicolumn{4}{c}{Допустимые значения}
    \\ \midrule 
    Холодный и переходный & 20\,--\,25 & 15\,--\,75 & 0,1 \\ \midrule[0pt] 
    Тёплый                & 15\,--\,28 & 20\,--\,80 & 0,5 \\ \bottomrule 
\end{tabularx}
\end{table}

Анализируя данные таблицы~\ref{tab:SR:mcclimate} и состояние рабочей комнаты, микроклимат которой поддерживается на оптимальном уровне системой водяного центрального отопления и естественной вентиляцией, можно сделать вывод, что параметры микроклимата производственного помещения соответствуют нормам.

В производственных помещениях, где допустимые нормативные величины микроклимата поддерживать не представляется возможным, необходимо проводить мероприятия по защите работников от возможного перегревания и охлаждения. 
% Это достигается различными средствами: 
% применением систем местного кондиционирования воздуха; 
% использованием СИЗ от повышенной или пониженной температуры; 
% регламентацией периодов работы в неблагоприятном микроклимате и отдыха в помещении с микроклиматом, нормализующим тепловое состояние; 
% сокращением рабочей смены и др.

% Профилактика перегревания работников в нагревающем микроклимате включает следующие мероприятия: 
% нормирование верхней границы внешней термической нагрузки на допустимом уровне применительно к 8-часовой рабочей смене; 
% регламентация продолжительности воздействия нагревающей среды (непрерывно и за рабочую смену) для поддержания среднесменного теплового состояния на оптимальном или допустимом уровне; 
% использование специальных СКЗ и СИЗ, уменьшающих поступление тепла извне к поверхности тела человека и обеспечивающих допустимое тепловое состояние работников.


\subsubsection{Освещённость рабочей зоны}

Освещение~--- важнейший фактор создания нормальных условий труда для работника. 
В случае недостатка освещённости рабочего места у человека не только уменьшается острота зрения, но и вызывается утомление организма в целом, что приводит к снижению производительности труда и увеличению опасности заболеваний.

Согласно санитарно-гигиеническим требованиям, рабочее место с ПЭВМ должно освещаться комбинированным освещением. 
Естественное освещение поступает в помещение через одно окно в светлое время суток. 
Искусственное освещение обеспечивается за счет люминесцентных ламп типа ЛБ только в темное время суток либо при недостаточном естественном освещении, т.~к. отличается относительной сложностью восприятия его зрительным органом человека.

С целью обеспечения требуемых норм необходимо произвести расчёт искусственной освещенности.
Расчёт общего равномерного искусственного освещения горизонтальной рабочей поверхности выполняется методом коэффициента светового потока, учитывающим световой поток, отражённый от потолка и стен. Длина помещения $a = 8$ м, ширина $b = 4$ м, высота потолка $H = 3$ м. Высота рабочей поверхности над полом $h_p = 0,75$ м. Интегральным критерием оптимальности расположения светильников является величина $\lambda$, которая для люминесцентных светильников с защитной решёткой лежит в диапазоне $1,1 - 1,3$.

% В помещении установлены лампы дневного света ЛД-40, световой поток которой равен $\Phi_\text{ЛД} = 2300$ Лм.
Выбираем светильники с люминесцентными лампами типа ОДОР-2-40.
Этот светильник имеет две лампы мощностью 40 Вт каждая, длина светильника равна 1227 мм, ширина~--- 265 мм, высота подвеса~--- 300 мм.

На первом этапе определяется значение индекса освещённости $i$ по следующей формуле
\begin{equation}
    i = \frac{S}{(a + b) \cdot h},
\end{equation}
\begin{where}
    \item $S$~--- площадь помещения, $\text{м}^2$;
    \item $h$~--- расчётная высота подвеса светильника, м;
    \item $a$ и $b$~--- длина и ширина помещения, м.
\end{where}

Высота светильника над рабочей поверхностью
\begin{equation}
    h = H - h_p - h_c,
\end{equation}
\begin{where}
    \item $H$~--- высота помещения, м;
    \item $h_p$~--- высота рабочей поверхности, м;
    \item $h_c$~--- высота подвеса светильника, м.
\end{where}

% Расстояние между соседними рядами светильников
% \begin{equation}
%     L = \lambda \cdot h.
% \end{equation}
Произведём расчёт и определим рекомендуемое число светильников в помещении:
$$h = 3 - 0,75 - 0,3 = 1,95 \text{ (м)};$$
$$i = \frac{32}{(8+4) \cdot 1,95} \approx 1,37;$$
% $$L = \lambda \cdot h = 1,1 \cdot 1.95 = 2,145 \text{ (м)}.$$

Рассчитываем число рядов $N_b$ и светильников $N_a$ в каждом из них, а также общее число ламп $N$, учитывая, что число ламп в каждом светильнике равно двум:
$$N_a = \frac{a}{L} = \frac{8}{2,145} = 3,73 \approx 4;$$
$$N_b = \frac{b}{L} = \frac{4}{2,145} = 1,86 \approx 2;$$
$$N = 2 \cdot N_a \cdot N_b = 2 \cdot 4 \cdot 2 = 16.$$

Следовательно, светильники стоит разместить в два ряда по три в каждом.
Пусть $a_c = 0,265$ м, $b_c = 1,227$ м~--- ширина и длина светильника соответственно.
Расстояние от крайних светильников до стены должно быть в три раза меньше расстояния между ними.
Произведём расчёт расстояния между светильниками и рядами ($L_a$ и $L_b$), а также от крайних светильников до каждой из стен ($l_a$ и $l_b$):
$$a = N_a \cdot a_c + (N_a - 1) \cdot L_a + \frac{2}{3} \cdot L_a;$$
$$8 = 4 \cdot 0,265 + (4 - 1) \cdot L_a + \frac{2}{3} \cdot L_a = 1,06 + \frac{11}{3} \cdot L_a;$$
$$L_a = \frac{3}{11} \cdot (8 - 1,06) \approx 1,89 \text{ (м), \hspace{0.5cm}} l_a = \frac{1}{3} \cdot L_a = 0,63 \text{ (м), }$$
$$b = N_b \cdot b_c + (N_b - 1) \cdot L_b + \frac{2}{3} \cdot L_b$$
$$4 = 2 \cdot 1,227 + (2 - 1) \cdot L_b + \frac{2}{3} \cdot L_b = 2,454 + \frac{5}{3} \cdot L_b;$$
$$L_b = \frac{3}{5} \cdot (4 - 2,454) \approx 0,93 \text{ (м), \hspace{0.5cm}} l_b = \frac{1}{3} \cdot L_b = 0,31 \text{ (м), }$$

План размещения светильников с люминесцентными лампами представлен в приложении~\ref{appendix:lights}.

Световой поток лампы определяется по формуле
\begin{equation}
    \Phi = \frac{E_\text{н} \cdot S \cdot K_\text{з} \cdot Z}{N \cdot \eta},
\end{equation}
\begin{where}
    \item $E_\text{н}$~--- нормируемая минимальная освещённость по СНиП 23-05-95, лк~\cite{Snip2003}; при использовании ЭВМ и одновременной работе с документами она должна быть равна 600 лк;
    \item $S$~--- площадь освещаемого помещения, $\text{м}^2$;
    \item $K_\text{з}$~--- коэффициент запаса, учитывающий загрязнение источника света и отражающих поверхностей, наличие в атмосфере дыма, пыли;
    \item $Z$~--- коэффициент неравномерности освещения, отношение средней освещённости к минимальной ($E_\text{ср} / E_\text{мин}$). Для люминесцентных ламп при расчётах берётся равным 1,1;
    \item $N$~--- число ламп в помещении;
    \item $\eta$~--- коэффициент использования светового потока.
\end{where}

Помещение, в котором осуществлялась работа, относится к типу со слабым выделением пыли, в связи с этим имеем:
\begin{itemize}
    \item коэффициент запаса $K_\text{з} = 1,1$;
    \item состояние потолка~--- свежепобеленный, поэтому значение коэффициента отражения потолка $\rho_\text{п} = 0,7$;
    \item состояние стен~--- побеленные бетонные стены, поэтому значение коэффициента отражения стен $\rho_\text{с} = 0,5$.
\end{itemize}

Коэффициент использования светового потока, показывающий какая часть светового потока ламп попадает на рабочую поверхность, для светильников типа ОДОР с люминесцентными лампами при текущих значениях коэффициентов отражения и индексе помещения $i = 1,09$ равен $\eta = 0,6$.
Тогда световой поток лампы
$$\Phi = \frac{600 \cdot 32 \cdot 1,1 \cdot 1,1}{16 \cdot 0,6} = 2420 \text{ (Лм)}.$$

Для люминесцентных ламп ЛД-40 с мощностью 40~Вт и напряжением сети 220~В, стандартный световой поток ЛД равен 2300~Лм.
Определим, входит ли полученное значение в требуемый диапазон от $-10$ до $20\%$:
\begin{equation}
    \Delta\Phi = \frac{\Phi_\text{ЛД} - \Phi}{\Phi_\text{ЛД}}.
\end{equation}
$$\Delta\Phi = \frac{2300 - 2420}{2300} = -\frac{120}{2300} = -0,052.$$
$$-10\% \le -5,2\% \le 20\%.$$

Таким образом, необходимый световой поток светильника не выходит за пределы требуемого диапазона.


\subsubsection{Психофизиологические факторы}

Значительное умственное напряжение и другие нагрузки приводят к переутомлению функционального состояния центральной нервной системы, нервно\-мышечного аппарата рук. 
Нерациональное расположение элементов рабочего места вызывает необходимость поддержания вынужденной рабочей позы. 
Длительный дискомфорт вызывает повышенное напряжение позвоночных мышц и обуславливает развитие общего утомления и снижение работоспособности.
При длительной работе за экраном дисплея появляется выраженное напряжение зрительного аппарата с появлением жалоб на неудовлетворительность работы, головные боли, усталость и болезненное ощущение в глазах, в пояснице, в области шеи, руках.

Режим труда и отдыха работника: при вводе данных, редактировании программ, чтении информации с экрана непрерывная продолжительность работы не должна превышать 4-х часов при 8-часовом рабочем дне. 
Через каждый час работы необходимо делать перерыв на 5\,--\,10 минут, а через два часа~--- на 15 минут.

С целью снижения или устранения нервно-психологического, зрительного и мышечного напряжения, предупреждения переутомления необходимо проводить комплекс физических упражнений и сеансы психофизической разгрузки и снятия усталости во время регламентируемых перерывов и после окончания рабочего дня.


\subsection{Электробезопасность}

Электробезопасность представляет собой систему организационных и технических мероприятий и средств, обеспечивающих защиту людей от вредного и опасного воздействия электрического тока, дугового разряда, электромагнитного поля и статистического электричества.

В отношении опасности поражения людей электрическим током различают:
\begin{enumerate}
[leftmargin=0pt,itemindent=\parindent+\labelwidth+\labelsep]
    \item Помещения без повышенной опасности с номинальным напряжением не~более 1\,000~В, в которых отсутствуют условия, создающие повышенную или особую опасность.
    \item Помещения с повышенной опасностью с номинальным напряжением не~более 1\,000~В, которые характеризуются наличием в них одного из следующих условий, создающих повышенную опасность: сырость, токопроводящая пыль, токопроводящие полы (металлические, земляные, железобетонные, кирпичные и~т.\,п.), высокая температура, возможность одновременного прикосновения человека к имеющим соединение с землей металлоконструкциям, технологическим аппаратам, с одной стороны, и к металлическим корпусам электрооборудования~--- с другой;
    \item Особо опасные помещения, которые характеризуются наличием оборудования свыше 1\,000~В и одного из следующих условий, создающих особую опасность: особой сырости, химически активной или органической среды, одновременно двух или более условий повышенной опасности. 
    Территории размещения наружных электроустановок в отношении опасности поражения людей электрическим током приравниваются к особо опасным помещениям.
\end{enumerate}

К оборудованию предъявляются следующие требования:
\begin{itemize}
    \item экран монитора должен находиться на расстоянии не менее 50~см от пользователя;
    \item необходимо применение приэкранных фильтров, специальных экранов.
\end{itemize}

К СКЗ от электрического тока относятся:
\begin{itemize}
    \item защитное заземление;
    \item зануление;
    \item разделительные трансформаторы;
    \item защитное отключение;
    \item применение малых напряжений;
    \item изоляция;
    \item оградительные устройства;
    \item сигнализация, блокировка, знаки безопасности, плакаты.
\end{itemize}

К СИЗ от поражения электрическим током относятся диэлектрические перчатки, боты, резиновые коврики и дорожки, изолирующие подставки на фарфоровых изоляторах, переносные заземления и изолированные инструменты.

Перед началом работы следует убедиться в отсутствии свешивающихся со стола или висящих под столом проводов электропитания, в целостности вилки и провода электропитания, в отсутствии видимых повреждений аппаратуры и рабочей мебели, в отсутствии повреждений и наличии заземления приэкранного фильтра.

Помещение, в котором производилась данная работа, принадлежит к первой категории, т.\,е. к помещениям без повышенной опасности по степени вероятности поражения электрическим током с номинальным напряжением до~1\,000~В.
Значения тока, напряжения и сопротивления заземления в помещении не превышают предельно допустимые, определённые в ГОСТ 12.1.038-82~\cite{Gost1983}: 0,1 мА, 36 В и 4 Ом соответственно.


\subsection{Пожарная безопасность}

Для обеспечения безопасности людей и сохранения материальных ценностей существует пожарная безопасность, основными системами которой являются системы предотвращения пожара и противопожарной защиты, включая организационно\-технические мероприятия.
Основы противопожарной защиты предприятий определены в стандартах ГОСТ 12.1.004-91~\cite{Gost1992} и ГОСТ 12.1.010-76~\cite{Gost1976}.

Возникновение пожара при работе с электронной аппаратурой может быть по причинам как электрического, так и неэлектрического характера.
Причины возникновения пожара неэлектрического характера:
\begin{itemize}
    \item халатное неосторожное обращение с огнем (курение, оставленные без присмотра нагревательные приборы, использование открытого огня);
    \item самовоспламенение и самовозгорание веществ.
\end{itemize}

Причины возникновения пожара электрического характера:
\begin{itemize}
    \item короткое замыкание;
    \item перегрузки по току;
    \item повышение переходных сопротивлений в контактах;
    \item искрение и электрические дуги;
    \item статическое электричество и~т.\,п.;
\end{itemize}

Меры пожарной профилактики:
\begin{enumerate}[leftmargin=0pt,itemindent=\parindent+\labelwidth+\labelsep]
    \item Строительно-планировочные меры определяются огнестойкостью зданий и сооружений (выбор материалов конструкций по степени огнестойкости). 
    В зависимости от степени огнестойкости определяются наибольшие дополнительные расстояния от выходов для эвакуации при пожарах.
    \item Технические меры включают в себя соблюдение противопожарных норм для систем отопления, освещения, электрического обеспечения и~т.\,д., использование разнообразных защитных систем и соблюдение параметров технологических процессов и режимов работы оборудования.
\end{enumerate}

Помещение, в котором производилась данная работа, по степени пожаровзрывоопасности относится к категории В, т.\,е. к помещениям с твердыми сгораемыми веществами, такими как деревянные шкафы, столы, двери.
В помещении для тушения возгораний предусмотрено использование углекислотного огнетушителя ОУ-3 для тушения возгораний классов А, В и электроустановок до 10\,000~В при температуре воздуха от минус $40^\circ$\,C до $50^\circ$\,C. 
Таким образом, состояние помещения соответствует нормам пожаробезопасности. 
План эвакуации людей из помещения представлен в приложении~\ref{appendix:evacuation}.


\subsection{Охрана окружающей среды}

С точки зрения использования ресурсов компьютер потребляет сравнительно небольшое количество электроэнергии, что положительным образом сказывается на общей экономии потребления электроэнергии в целом.
При написании данной работы вредных выбросов в атмосферу, почву и водные источники не производилось, радиационного заражения не произошло, чрезвычайные ситуации не наблюдались, поэтому ущерба окружающей среде не было нанесено.

Основными отходами являются черновики бумаги, отработавшие люминесцентные лампы и картриджи.
Бумага сдавалась в пункт приёма макулатуры ООО <<Чистый мир>>.
Использованные лампы отправлялись в ООО НПП <<Экотом>>, а отработавшие картриджи~--- в ООО <<Томск Принт>>.


\subsection{Защита в чрезвычайных ситуациях}

Чрезвычайная ситуация (ЧС)~--- обстановка на определенной территории, сложившаяся в результате аварии, опасного природного явления, катастрофы, стихийного или иного бедствия, которая может повлечь или повлекла за собой человеческие жертвы, ущерб здоровью людей или окружающей среде, значительные материальные потери и нарушение условий жизнедеятельности людей.

Наиболее типичные для нашего региона вида ЧС, возможных на предприятии: пожар, сильный мороз, несанкционированное проникновение посторонних лиц.
Меры по предупреждению ЧС, возникших вследствие морозов: 
\begin{enumerate}
[leftmargin=0pt,itemindent=\parindent+\labelwidth+\labelsep]
    \item Повышение устойчивости системы электроснабжения. В первую очередь целесообразно заменить воздушные линии электропередач на кабельные (подземные) сети, использовать резервные сети для потребителей, предусмотреть автономные резервные источники электропитания объекта (передвижные электрогенераторы).
    \item Обеспечение устойчивости теплоснабжения за счет запасных автономных источников теплоснабжения, кольцевания системы, заглубления теплотрасс.
    \item Обеспечение устойчивости систем водоснабжения (устройство дублирования вод питания, кольцевание системы, заглубление водопроводов, обустройство резервных емкостей и водохранилищ, очистка воды от вредных веществ и т.п.).
    \item Обеспечение устойчивости системы водоотведения. 
    Повышение устойчивости системы канализации достигается созданием резервной сети труб, по которым может отводиться загрязненная вода при аварии основной сети. 
    Должна быть разработана схема аварийного выпуска сточных вод непосредственно в водоемы. Насосы, используемые для перекачки загрязненной воды, комплектуются надежными источниками электропитания.
\end{enumerate}

Для предупреждения ЧС, связанных с несанкционированным проникновением посторонних лиц, необходимы следующие меры безопасности: 
\begin{enumerate}
    \item Организовать контрольно-пропускной пункт.
    \item Установить системы видеонаблюдения во всех помещениях здания, а также на всех входах и выходах.
    \item Установить оповещающие системы безопасности при несанкционированном проникновении на предприятие в нерабочее время.
\end{enumerate}

Общие рекомендации при возникновении аварийной ситуации необходимо:
\begin{enumerate}
    \item Сообщить руководству (дежурному).
    \item Позвонить в соответствующую аварийную службу или МЧС~--- тел.~112.
    \item Принять меры по ликвидации аварии в соответствии с инструкцией.
\end{enumerate}


\subsection{Перечень научно-технической документации}

\begin{enumerate}[leftmargin=\labelwidth+\labelsep]
    \item ГОСТ 12.0.004-2015 Система стандартов безопасности труда (ССБТ). Организация обучения безопасности труда. Общие положения.
    \item ГОСТ 12.0.003-2015 Система стандартов безопасности труда (ССБТ). Опасные и вредные производственные факторы. Классификация.
    \item ГОСТ 12.1.013-78 Система стандартов безопасности труда (ССБТ). Строительство. Электробезопасность. Общие требования.
    \item ГОСТ 12.1.038-82 Система стандартов безопасности труда (ССБТ). Электробезопасность. Предельно допустимые значения напряжений прикосновения и токов.
    \item СанПиН 2.2.2/2.4.1340-03 Гигиенические требования к персональным электронно-вычислительным машинам и организации работы.
    \item ГОСТ 12.1.003-83 Система стандартов безопасности труда (ССБТ). Шум. Общие требования безопасности.
    \item СанПиН 2.2.4/2.1.8.10-32-2002 Шум на рабочих местах, в помещениях жилых, общественных зданий и на территории жилой застройки.
    \item СанПиН 2.2.4.548-96 Гигиенические требования к микроклимату производственных помещений.
    \item СНиП 23-05-95 Естественное и искусственное освещение.
    \item ГОСТ 12.1.004-91 Система стандартов безопасности труда (ССБТ). Пожарная безопасность. Общие требования.
    \item ГОСТ 12.1.010-76 Система стандартов безопасности труда (ССБТ). Взрывобезопасность. Общие требования.
\end{enumerate}


\anonsubsection{Заключение по разделу}

В ходе рассмотрения безопасности и охраны труда при осуществлении работ на ПЭВМ, были выявлены вредные и опасные факторы рабочей зоны, причины и средства защиты, рассмотрены чрезвычайные ситуации и поведение в них. 
Также были рассмотрены оптимальные условия для работы в данных условиях.