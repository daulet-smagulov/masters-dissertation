\newpage

\begin{abstract}
В данной курсовой работе я исследую и использую модели копула для представления и отображения многомерной зависимости в финансовых временных рядах. 
Я предлагаю алгоритм вычисления мер риска (\textit{Value-at-Risk} и \textit{Conditional-Value-at-Risk}) с использованием копул. 
Использую CVaR-оптимальный портфель. 
Далее я вычисляю P\&L портфеля и соответствующие кривые мер риска. 
Кривые \textit{VaR} и \textit{CVaR} были смоделированы с помощью трёх копул: Гауссовой, $t$\,--\,Стьюдента и вайн-копулы.
Эти модели позволяют учитывать корреляцию различных активов друг с другом.
Согласно полученным кривым риска, предложенная модель является более консервативной, чем историческое моделирование.
%Таким образом, все три модели позволяют 
Модель R-vine позволяет более адекватно оценить риски и предсказать поведение портфеля лучше, чем обычный эмпирический метод.
% Также описаны дальнейшие направления исследования. \\
\begin{center}
    \textbf{Abstract}
\end{center}
\begin{english}
In the paper, I use and investigate copulas models to represent multivariate dependence in financial time series. 
I propose the algorithm of risk measure computation using copula models.
Using the optimal CVaR portfolio, I compute portfolio's Profit \& Loss series and corresponded risk measures curves.
Value-at-risk and Conditional-Value-at-risk curves were simulated by three copula models: Gaussian, Student's~$t$ and regular vine copula. 
According to these risk curves, the proposed copula models are more conservative than a historical scenario method.
%Thus, all three models have superior 
In addition, R-vine copula model has better prediction ability than a usual empirical method. 
% Further directions of research are described.
\end{english}
\end{abstract}