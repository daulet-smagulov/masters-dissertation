\section*{Реферат}

Выпускная квалификационная работа содержит \formbytotal{TotPages}{страниц}{у}{ы}{}, \formbytotal{totalcount@figure}{рисун}{ок}{ка}{ков}, \formbytotal{totalcount@table}{таблиц}{у}{ы}{}, \formbytotal{citenum}{источник}{}{а}{ов} литературы и \formbytotal{appendixNum}{приложени}{е}{я}{й}.

\textbf{Ключевые слова:} копула, мера риска, моделирование.

\textbf{Объект исследования:} копулярные модели.

\textbf{Цель работы:} построение системы оценивания риска инвестиций с использованием копулярных моделей.

В работе использованы  копула-функции для моделирования многомерной зависимости на примере финансовых временных рядах. 
Предложен алгоритм вычисления точечных и интервальных оценок мер риска \textit{Value-at-Risk, VaR} и \textit{Conditional-Value-at-Risk, CVaR} с использованием копула-функций. 
Сформулирована и решена задача поиска CVaR-оптимального портфеля. 
Проведено моделирование рядов  \textit{VaR} и \textit{CVaR} с использованием трёх копула-функций: Гауссовой, $t$\,--\,Стьюдента и иерархической копулы.
Для оценки параметров копулярных моделей и проведения расчетов  использован язык программирования R. 

\textbf{Степень внедрения:} средняя; требуется оформление программного кода в виде пакета программ.

\textbf{Область применения:} мультидисциплинарная, в том числе финансовый риск-менеджмент, актуарная математика, эконометрика, система безопасности, обработка сигналов.


%\begin{abstracteng}
%In the research, copula models are used and investigated to represent multivariate dependencies in financial time series.  The algorithm of risk measure computation using copula models is proposed. Using the mean-CVaR optimization, portfolio's Profit \& Loss series and corresponded risk measures curves are computed.Value-at-risk and Conditional-Value-at-risk curves were simulated via three copula models: Gaussian, Student's~$t$ and regular vine copula. According to these risk curves, the proposed copula models are more conservative than a historical scenario method. In addition, R-vine copula model has better prediction ability than a usual empirical method.
%\end{abstracteng}